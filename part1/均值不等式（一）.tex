\chapter{均值不等式 (一)}

均值不等式是三大基本不等式之一,本讲介绍其在证明 $n$ 元不等式中的应用,需要体会均值不等式的适用情形、掌握“拆”的技巧、积累常见结构之间的关系。

\section{均值不等式的加强}

% 例 1
\begin{example}{}{}
设 $a_1, a_2, \dots, a_{n+1}$ 是正实数,记 $A_n, G_n$ 分别为 $a_1, \dots, a_n$ 的算术平均值和几何平均值。求证:
\begin{enumerate}
    \item $n(A_{n}-G_{n})\le(n+1)(A_{n+1}-G_{n+1})$;
    \item $\left(\frac{A_{n}}{G_{n}}\right)^{n}\le\left(\frac{A_{n+1}}{G_{n+1}}\right)^{n+1}$.
\end{enumerate}
\end{example}

\newpage 
% 例 2
\begin{example}{}{}
设 $a_{1},a_{2},\dots,a_{n}$ 是正实数,$b_{1},b_{2},\dots,b_{n}$ 是 $a_{1},a_{2},\dots,a_{n}$ 的一个排列。求证:
\begin{equation}
    A_{n}-G_{n}\ge\frac{1}{2n}\sum_{i=1}^{n}(\sqrt{b_{i+1}}-\sqrt{b_{i}})^{2},
\end{equation}
其中 $b_{n+1}=b_{1}$。
\end{example}

\newpage
\section{均值不等式的基本用法}

% 例 3
\begin{example}{}{}
设 $a_{1},a_{2},\dots,a_{n}$ 是正实数,记 $S=a_{1}+a_{2}+\dots+a_{n}$。求证:
\begin{equation}
    (1+a_{1})(1+a_{2})\dots(1+a_{n})\le1+\sum_{k=1}^{n}\left(1-\frac{k}{2n}\right)^{k-1}\frac{S^{k}}{k!}.
\end{equation}
\end{example}

\newpage 
% 例 4
\begin{example}{}{}
设整数 $n\ge2$,正实数 $a_{1},a_{2},\dots,a_{n}$ 满足 $a_{1}+a_{2}+\dots+a_{n}=1$。求证:
\begin{equation}
    \sum_{i=1}^{n}\frac{a_{i}}{a_{i+1}-a_{i+1}^{3}}\ge\frac{n^{3}}{n^{2}-1},
\end{equation}
其中 $a_{n+1}=a_{1}$。
\end{example}

\newpage 
% 例 5
\begin{example}{}{}
设 $a_{1},a_{2},\dots,a_{n},b_{1},b_{2},\dots,b_{n}$ 是非负实数,对 $1\le k\le n$,记 $c_{k}=\prod_{i=1}^{k}b_{i}^{\frac{1}{k}}$。求证:
\begin{equation}
    nc_{n}+\sum_{k=1}^{n}k(a_{k}-1)c_{k}\le\sum_{k=1}^{n}a_{k}^{k}b_{k}.
\end{equation}
\end{example}

\newpage 
% 例 6
\begin{example}{}{}
设 $a_{1},a_{2},\dots,a_{n}$ 是正实数,求证:
\begin{equation}
    \sum_{k=1}^{n}\left(\frac{a_{k}}{a_{k+1}}\right)^{n-1}\ge-n+2\left(\sum_{k=1}^{n}a_{k}\right)\prod_{k=1}^{n}{a_{k}}^{-\frac{1}{n}},
\end{equation}
其中 $a_{n+1}=a_{1}$。
\end{example}


\newpage 
% 例 7
\begin{example}{}{}
设整数 $n>2$,正实数 $a_{2},a_{3},\dots,a_{n}$ 满足 $a_{2}a_{3}\dots a_{n}=1$。求证:
\begin{equation}
    (1+a_{2})^{2}(1+a_{3})^{3}\dots(1+a_{n})^{n}>\frac{1}{4^{n-1}}n^{n}(n-1)^{n-1}.
\end{equation}
\end{example}


\newpage 
% 例 8
\begin{example}{}{}
给定正整数 $n$。设 $a_{1},a_{2},\dots,a_{n}$ 是正实数,求
\begin{equation}
    \frac{(1+a_{1})(1+a_{1}+a_{2})\dots(1+a_{1}+a_{2}+\dots+a_{n})}{\sqrt{a_{1}a_{2}\dots a_{n}}}
\end{equation}
的最小值。
\end{example}






\newpage 
\section{作业题}
% 作业题 1
\begin{homework}{}{}
设整数 $n\ge2$,$a_{1},a_{2},\dots,a_{n}$ 是正实数,求证:
\begin{equation}
    \frac{A_{n}}{G_{n}}\ge \max_{1\le i<j\le n}\left[\frac{1}{2}+\frac{1}{4}\left(\frac{a_{i}}{a_{j}}+\frac{a_{j}}{a_{i}}\right)\right]^{\frac{1}{n}}.
\end{equation}
\end{homework}

\newpage 
% 作业题 2
\begin{homework}{2016年高联P1}{}
给定整数 $n\ge2$。设实数 $a_{1},a_{2},\dots,a_{n}$ 满足 $9a_{i}>11a_{i+1}^{2}\ (1\le i\le n-1)$。求
\begin{equation}
    (a_{1}-a_{2}^{2})(a_{2}-a_{3}^{2})\dots(a_{n}-a_{1}^{2})
\end{equation}
的最大值。
\end{homework}

\newpage 
% 作业题 3
\begin{homework}{}{}
给定整数 $n\ge2$。求最大的实数 $\lambda$,使得对任意实数 $x_{1},x_{2},\dots,x_{n}\in(0,1]$,都有
\begin{equation}
    \frac{1}{\sum_{i=1}^{n}x_{i}}\ge\frac{1}{n}+\lambda\prod_{i=1}^{n}(1-x_{i}).
\end{equation}
\end{homework}

\newpage 
% 作业题 4
\begin{homework}{}{}
设 $n$ 是正整数,正实数 $a_{1},a_{2},\dots,a_{n}$ 满足 $a_{1}^{2}+2a_{2}^{3}+\dots+na_{n}^{n+1}\le1$。求证:
\begin{equation}
    2a_{1}+3a_{2}^{2}+\dots+(n+1)a_{n}^{n}<3.
\end{equation}
\end{homework}




