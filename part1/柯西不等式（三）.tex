\chapter{柯西不等式 (三)}

本节介绍柯西不等式的两种推广和加强:拉格朗日恒等式与 Hölder 不等式,需要注意“平移不变”技巧的运用。

\section{拉格朗日恒等式}
\begin{thm}{拉格朗日恒等式}{}
对于任意实数 $a_{1},a_{2},\dots,a_{n}$ 和 $b_{1},b_{2},\dots,b_{n}$,有
\begin{equation}
    \left(\sum_{i=1}^{n}a_{i}^{2}\right)\left(\sum_{i=1}^{n}b_{i}^{2}\right)-\left(\sum_{i=1}^{n}a_{i}b_{i}\right)^{2}=\frac{1}{2}\sum_{1\le i<j\le n}(a_{i}b_{j}-a_{j}b_{i})^{2}.
\end{equation}
\end{thm}


\newpage
% 例 1
\begin{example}{2010 中欧数学奥林匹克}{}
给定整数 $n\ge2$。求最大的实数 $\lambda$,使得对任意实数 $a_{1},a_{2},\dots,a_{n}$,有
\begin{equation}
    \frac{a_{1}^{2}+a_{2}^{2}+\dots+a_{n}^{2}}{n}\ge\left(\frac{a_{1}+a_{2}+\dots+a_{n}}{n}\right)^{2}+\lambda(a_{1}-a_{n})^{2}.
\end{equation}
\end{example}

\newpage 
% 例 2
\begin{example}{}{}
设 $a_{0},a_{1},\dots,a_{2n}$ 是实数,求证:
\begin{equation}
    \sum_{i=0}^{2n}a_{i}^{2}\ge\frac{1}{2n+1}\left(\sum_{i=0}^{2n}a_{i}\right)^{2}+\frac{3}{n(n+1)(2n+1)}\left(\sum_{i=0}^{2n}(i-n)a_{i}\right)^{2}.
\end{equation}
\end{example}

\newpage 
% 例 3
\begin{example}{}{}
设整数 $n\ge4$,正实数 $a_{1},a_{2},\dots,a_{n}$ 和 $t$ 满足 $\sum_{i=1}^{n}a_{i}=3t,\ \sum_{i=1}^{n}a_{i}^{2}=3t^{2},\ \sum_{i=1}^{n}a_{i}^{3}>3t^{3}+t$。求证:存在 $1\le i<j\le n$ 使得 $|a_{i}-a_{j}|>1$。
\end{example}

\newpage 
% 例 4
\begin{example}{}{}
设整数 $n\ge3$。正实数 $a_{1},a_{2},\dots,a_{n}$ 满足 $(\sum_{i=1}^{n}a_{i})(\sum_{i=1}^{n}\frac{1}{a_{i}})=n^{2}+1$。求证:
\begin{equation}
    \left(\sum_{i=1}^{n}a_{i}^{2}\right)\left(\sum_{i=1}^{n}\frac{1}{a_{i}^{2}}\right)\ge n^{2}+4+\frac{2}{n(n-1)}.
\end{equation}
\end{example}

\newpage 
% 例 5
\begin{example}{}{}
设 $a_{1},a_{2},\dots,a_{n}$ 是正实数,求证:
\begin{equation}
    \left(\sum_{i=1}^{n}a_{i}^{2n}\right)\left(\sum_{i=1}^{n}\frac{1}{a_{i}^{2n}}\right)-n^{2}\sum_{1\le i<j\le n}\left(\frac{a_{i}}{a_{j}}-\frac{a_{j}}{a_{i}}\right)^{2}\ge n^{2}.
\end{equation}
\end{example}

\newpage 
% 例 6
\begin{example}{}{}
设 $a_{1},\dots, a_{2n}, b_{1}, \dots, b_{2n}$ 是实数,求证:
\begin{equation}
    \left(\sum_{i=1}^{2n}a_{i}^{2}\right)\left(\sum_{i=1}^{2n}b_{i}^{2}\right)-\left(\sum_{i=1}^{2n}a_{i}b_{i}\right)^{2}\ge\left[\sum_{i=1}^{n}(a_{i}b_{n+i}-a_{n+i}b_{i})\right]^{2}.
\end{equation}
\end{example}

\newpage 
\section{Hölder 不等式}
\begin{thm}{Hölder 不等式}{sym_holder}
设 $m, n \ge 2$ 为整数。给定 $m$ 组非负实数:
\[
(a_{1,1}, \dots, a_{n,1}), \quad (a_{1,2}, \dots, a_{n,2}), \quad \dots, \quad (a_{1,m}, \dots, a_{n,m}).
\]
则有:
\begin{equation}
    \sum_{i=1}^n \left( \prod_{j=1}^m a_{i,j} \right) \le \prod_{j=1}^m \left( \sum_{i=1}^n a_{i,j}^m \right)^{\frac{1}{m}}
\end{equation}
即:
\begin{equation}
    \sum_{i=1}^n a_{i,1} a_{i,2} \cdots a_{i,m} \le \left( \sum_{i=1}^n a_{i,1}^m \right)^{\frac{1}{m}} \left( \sum_{i=1}^n a_{i,2}^m \right)^{\frac{1}{m}} \cdots \left( \sum_{i=1}^n a_{i,m}^m \right)^{\frac{1}{m}}
\end{equation}
当且仅当这 $m$ 组向量两两共线(即成比例)时,等号成立。
\end{thm}
\vspace{2cm}






\newpage 
% 例 7
\begin{example}{}{}
设 $a_{1},a_{2},\dots,a_{n}$ 是正实数,求证:
\begin{equation}
    \prod_{i=1}^{n}(a_{i}^{3}+1)\ge\prod_{i=1}^{n}(a_{i}^{2}a_{i+1}+1),
\end{equation}
其中 $a_{n+1}=a_{1}$。
\end{example}

\newpage 
% 例 8
\begin{example}{}{}
设正实数 $a_{1},a_{2},\dots,a_{n}$ 满足 $a_{1}+a_{2}+\dots+a_{n}=1$。求证:
\begin{equation}
    (a_{1}a_{2}+a_{2}a_{3}+\dots+a_{n}a_{1})\left(\frac{a_{1}}{a_{2}^{2}+a_{2}}+\frac{a_{2}}{a_{3}^{2}+a_{3}}+\dots+\frac{a_{n}}{a_{1}^{2}+a_{1}}\right)\ge\frac{n}{n+1}.
\end{equation}
\end{example}

\newpage 
\section{作业题}

% 作业 1
\begin{homework}{}{}
设 $a_{1},a_{2},\dots,a_{n}$ 是实数,求证:
\begin{equation}
    \left(\sum_{1\le i<j\le n}|a_{i}-a_{j}|\right)^{2}\le\frac{n^{2}-1}{3}\sum_{1\le i<j\le n}(a_{i}-a_{j})^{2}.
\end{equation}
\end{homework}

\newpage 
% 作业 2
\begin{homework}{}{}
设正实数 $a_{1},a_{2},\dots,a_{n},b_{1},b_{2},\dots,b_{n}$ 满足 $a_{i}>b_{i}\ (i=1,2,\dots,n)$ 且 $\prod_{i=1}^{n}a_{i}b_{i}=1$。求证:
\begin{equation}
    \prod_{i=1}^{n}a_{i}-\prod_{i=1}^{n}b_{i}\ge n\sqrt[n]{\prod_{i=1}^{n}(a_{i}-b_{i})}.
\end{equation}
\end{homework}



\newpage 
\section{柯西不等式补充练习题}


\begin{problem}{}
设 $a_i,b_i>0,\ i=1,2,\dots,n$,且 $a_1+a_2+\dots+a_n=b_1+b_2+\dots+b_n$,证明:
\begin{equation}
\frac{a_1^2}{a_1+b_1}+\frac{a_2^2}{a_2+b_2}+\dots+\frac{a_n^2}{a_n+b_n}
\ge\frac12(a_1+a_2+\dots+a_n).
\end{equation}
\end{problem}



\begin{problem}{10 浙大自招}
设整数 $n\ge 2$,且正实数 $x_1,x_2,\dots,x_n$ 满足 $x_1+x_2+\dots+x_n=1$,证明:
\begin{equation}
\frac{1}{x_1-x_1^3}+\frac{1}{x_2-x_2^3}+\dots+\frac{1}{x_n-x_n^3}>4.
\end{equation}
\end{problem}


\begin{problem}{02 女奥}
设整数 $n\ge 2$,且 $P_1,P_2,\dots,P_n$ 是 $1,2,\dots,n$ 的任意排列,证明:
\begin{equation}
\frac{1}{P_1+P_2}+\frac{1}{P_2+P_3}+\dots+\frac{1}{P_{n-1}+P_n}>\frac{n-1}{n+2}.
\end{equation}
\end{problem}


\begin{problem}{11 甘肃预赛}
已知正实数 $a_1,a_2,\dots,a_n$ 满足 $a_1+a_2+\dots+a_n=1$,证明:
\begin{equation}
\Bigl(a_1+\frac{1}{a_1}\Bigr)^2+\Bigl(a_2+\frac{1}{a_2}\Bigr)^2+\dots+\Bigl(a_n+\frac{1}{a_n}\Bigr)^2
\ge\frac{(n^2+1)^2}{n}.
\end{equation}
\end{problem}




\begin{problem}{}
设 $a_1,a_2,\dots,a_n$ 是实数,证明:
\begin{equation}
\sqrt[3]{a_1^3+a_2^3+\dots+a_n^3}
\le\sqrt{a_1^2+a_2^2+\dots+a_n^2}.
\end{equation}
\end{problem}


\begin{problem}{}
设 $n\ge 2,\ n\in\mathbb N^+$,证明:
\begin{equation}
1\cdot\sqrt{C_n^1}+2\cdot\sqrt{C_n^2}+\dots+n\cdot\sqrt{C_n^n}
<\sqrt{2^{n-1}\,n^3}.
\end{equation}
\end{problem}


\begin{problem}{}
设整数 $n\ge 3$,正实数 $a_1,a_2,\dots,a_n$ 满足 $a_n\ge a_1+a_2+\dots+a_{n-1}$,证明:
\begin{equation}
\Bigl(\frac{1}{a_1}+\frac{1}{a_2}+\dots+\frac{1}{a_n}\Bigr)(a_1+a_2+\dots+a_n)
\ge 2(n-1)^2+2.
\end{equation}
\end{problem}


\begin{problem}{}
已知正实数 $x_1,x_2,\dots,x_{n+1}$ 满足 $x_{n+1}=x_1+x_2+\dots+x_n$,证明:
\begin{equation}
\Bigl(\sum_{i=1}^n\sqrt{x_i(x_{n+1}-x_i)}\Bigr)^2
\le(n-1)\,x_{n+1}^2.
\end{equation}
\end{problem}




\begin{problem}{}
已知 $a_1,a_2,\dots,a_n$ 为正实数,证明:
\begin{equation}
\frac{(a_1+a_2+\dots+a_n)^2}{2(a_1^2+a_2^2+\dots+a_n^2)}
\le\frac{a_1}{a_2+a_3}+\frac{a_2}{a_3+a_4}+\dots+\frac{a_n}{a_1+a_2}.
\end{equation}
\end{problem}








\begin{problem}{}
给定整数 $n\ge 2$,非负实数 $x_1,x_2,\dots,x_n$ 满足
\begin{equation}
\sum_{i=1}^n x_i+\sum_{1\le i<j\le n}x_ix_j=n+C_n^2,
\end{equation}
其中 $C_a^b=\dfrac{a!}{b!(a-b)!}$。证明:$\sum_{i=1}^n x_i\ge n$。
\end{problem}




\begin{problem}{}
(17 HMMT)设 $x_1,x_2,\dots,x_{2017}$ 均为实数,求出最大的实数 $c$,使得下列不等式成立:
\begin{equation}
\sum_{i=1}^{2016}x_i(x_i+x_{i+1})\ge c\,x_{2017}^2.
\end{equation}
\end{problem}




\begin{problem}{19 欧洲杯}
已知数列 $\{x_n\}$ 满足 $x_1=\sqrt2,\ x_{n+1}=x_n+\dfrac{1}{x_n}\ (n\in\mathbb N^+)$,证明:
\begin{equation}
\sum_{k=1}^{2019}\frac{x_k^2}{2x_kx_{k+1}-1}>
\frac{2019^2}{x_{2019}^2+\dfrac{1}{x_{2019}^2}}.
\end{equation}
\end{problem}


\begin{problem}{}
已知正实数 $x_1,x_2,\dots,x_n$ 满足 $x_1+x_2+\dots+x_n=1$,证明:
\begin{equation}
\sum_{i=1}^n\frac{(n-1)\sum_{j\ne i}x_j^2+x_i}{1+\sum_{j\ne i}x_j}
\ge\frac{n^2-n+1}{2n-1}.
\end{equation}
\end{problem}


\begin{problem}{96 波兰}
设整数 $n\ge 2$,正数 $a_1,\dots,a_n,x_1,\dots,x_n$ 满足 $\sum_{i=1}^n a_i=1,\ \sum_{i=1}^n x_i=1$,证明:
\begin{equation}
2\sum_{1\le i<j\le n}x_ix_j\le\frac{n-2}{n-1}+\sum_{i=1}^n\frac{a_ix_i^2}{1-a_i},
\end{equation}
并指出等号成立的充要条件。
\end{problem}


\begin{problem}{}
设 $a_i>0,\,b_i>0,\ a_ib_i=c_i^2+d_i^2\ (i=1,2,\dots,n)$,证明:
\begin{equation}
\sum_{i=1}^n a_i\cdot\sum_{i=1}^n b_i
\ge\Bigl(\sum_{i=1}^n c_i\Bigr)^2+\Bigl(\sum_{i=1}^n d_i\Bigr)^2.
\end{equation}
\end{problem}


\begin{problem}{02 罗马尼亚}
设整数 $n\ge 4$,正实数 $a_1,a_2,\dots,a_n$ 满足 $a_1^2+a_2^2+\dots+a_n^2=1$,求证:
\begin{equation}
\frac{a_1}{a_2^2+1}+\frac{a_2}{a_3^2+1}+\dots+\frac{a_n}{a_1^2+1}
\ge\frac45\Bigl(a_1\sqrt{a_1}+a_2\sqrt{a_2}+\dots+a_n\sqrt{a_n}\Bigr)^2.
\end{equation}
\end{problem}


\begin{problem}{98 罗马尼亚}
已知正实数 $x_1,x_2,\dots,x_n$ 满足 $x_1x_2\dots x_n=1$,证明:
\begin{equation}
\frac{1}{n-1+x_1}+\frac{1}{n-1+x_2}+\dots+\frac{1}{n-1+x_n}\le 1.
\end{equation}
\end{problem}


\begin{problem}{14 东南}
设整数 $n\ge 2$,正实数 $x_1,x_2,\dots,x_n$ 满足 $x_1+\dots+x_n=1$,且 $x_{n+1}=x_1$,求证:
\begin{equation}
\frac{x_1}{x_2-x_2^3}+\frac{x_2}{x_3-x_3^3}+\dots+\frac{x_n}{x_{n+1}-x_{n+1}^3}
\ge\frac{n^3}{n^2-1}.
\end{equation}
\end{problem}


\begin{problem}{}
已知正实数 $x_1,x_2,\dots,x_n$ 满足 $\sum_{i=1}^n x_i=1$,证明:
\begin{equation}
\Bigl(\sum_{i=1}^n\sqrt{x_i}\Bigr)^2\cdot\sum_{i=1}^n\frac{1}{1+x_i}
\le\frac{n^3}{n+1}.
\end{equation}
\end{problem}




\begin{problem}{10 伊朗改}
设 $a_1,a_2,\dots,a_n$ 是正实数,证明:
\begin{equation}
\sum_{i=1}^n\frac{1}{a_i^2}+\frac{1}{(a_1+a_2+\dots+a_n)^2}
\ge\frac{n^3+1}{(n^2+1)^2}\Bigl(\sum_{i=1}^n\frac{1}{a_i}+\frac{1}{a_1+a_2+\dots+a_n}\Bigr)^2.
\end{equation}
\end{problem}


\begin{problem}{}
设 $x_1,x_2,\dots,x_n$ 是正实数,且 $x_{n+1}=x_1$,证明:
\begin{equation}
\frac{x_1^2}{x_2}+\frac{x_2^2}{x_3}+\dots+\frac{x_n^2}{x_{n+1}}
\ge(x_1+x_2+\dots+x_n)+\frac{4(x_1-x_n)^2}{x_1+x_2+\dots+x_n}.
\end{equation}
\end{problem}

