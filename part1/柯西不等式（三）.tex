\chapter{柯西不等式 (三)}

本节介绍柯西不等式的两种推广和加强:拉格朗日恒等式与 Hölder 不等式,需要注意“平移不变”技巧的运用。

\section{拉格朗日恒等式}
\begin{thm}{拉格朗日恒等式}{}
对于任意实数 $a_{1},a_{2},\dots,a_{n}$ 和 $b_{1},b_{2},\dots,b_{n}$,有
\begin{equation}
    \left(\sum_{i=1}^{n}a_{i}^{2}\right)\left(\sum_{i=1}^{n}b_{i}^{2}\right)-\left(\sum_{i=1}^{n}a_{i}b_{i}\right)^{2}= \sum_{1\le i<j\le n}(a_{i}b_{j}-a_{j}b_{i})^{2}.
\end{equation}
\end{thm}


\newpage
% 例 1
\begin{example}{2010 中欧数学奥林匹克}{}
给定整数 $n\ge2$。求最大的实数 $\lambda$,使得对任意实数 $a_{1},a_{2},\dots,a_{n}$,有
\begin{equation}
    \frac{a_{1}^{2}+a_{2}^{2}+\dots+a_{n}^{2}}{n}\ge\left(\frac{a_{1}+a_{2}+\dots+a_{n}}{n}\right)^{2}+\lambda(a_{1}-a_{n})^{2}.
\end{equation}
\end{example}

\newpage 
% 例 2
\begin{example}{1999年CTST}{}
设 $a_{0},a_{1},\dots,a_{2n}$ 是实数,求证:
\begin{equation}
    \sum_{i=0}^{2n}a_{i}^{2}\ge\frac{1}{2n+1}\left(\sum_{i=0}^{2n}a_{i}\right)^{2}+\frac{3}{n(n+1)(2n+1)}\left(\sum_{i=0}^{2n}(i-n)a_{i}\right)^{2}.
\end{equation}
\end{example}

\newpage 
% 例 3
\begin{example}{}{}
设整数 $n\ge4$,正实数 $a_{1},a_{2},\dots,a_{n}$ 和 $t$ 满足 $\sum_{i=1}^{n}a_{i}=3t,\ \sum_{i=1}^{n}a_{i}^{2}=3t^{2},\ \sum_{i=1}^{n}a_{i}^{3}>3t^{3}+t$。求证:存在 $1\le i<j\le n$ 使得 $|a_{i}-a_{j}|>1$。
\end{example}

\newpage 
% 例 4
\begin{example}{}{}
设整数 $n\ge3$。正实数 $a_{1},a_{2},\dots,a_{n}$ 满足 $(\sum_{i=1}^{n}a_{i})(\sum_{i=1}^{n}\frac{1}{a_{i}})=n^{2}+1$。求证:
\begin{equation}
    \left(\sum_{i=1}^{n}a_{i}^{2}\right)\left(\sum_{i=1}^{n}\frac{1}{a_{i}^{2}}\right)\ge n^{2}+4+\frac{2}{n(n-1)}.
\end{equation}
\end{example}

\newpage 
% 例 5
\begin{example}{}{}
设 $a_{1},a_{2},\dots,a_{n}$ 是正实数,求证:
\begin{equation}
    \left(\sum_{i=1}^{n}a_{i}^{2n}\right)\left(\sum_{i=1}^{n}\frac{1}{a_{i}^{2n}}\right)-n^{2}\sum_{1\le i<j\le n}\left(\frac{a_{i}}{a_{j}}-\frac{a_{j}}{a_{i}}\right)^{2}\ge n^{2}.
\end{equation}
\end{example}

\newpage 
% 例 6
\begin{example}{}{}
设 $a_{1},\dots, a_{2n}, b_{1}, \dots, b_{2n}$ 是实数,求证:
\begin{equation}
    \left(\sum_{i=1}^{2n}a_{i}^{2}\right)\left(\sum_{i=1}^{2n}b_{i}^{2}\right)-\left(\sum_{i=1}^{2n}a_{i}b_{i}\right)^{2}\ge\left[\sum_{i=1}^{n}(a_{i}b_{n+i}-a_{n+i}b_{i})\right]^{2}.
\end{equation}
\end{example}

\newpage 
\section{Hölder 不等式}
\begin{thm}{Hölder 不等式}{sym_holder}
设 $m, n \ge 2$ 为整数。给定 $m$ 组非负实数:
\[
(a_{1,1}, \dots, a_{n,1}), \quad (a_{1,2}, \dots, a_{n,2}), \quad \dots, \quad (a_{1,m}, \dots, a_{n,m}).
\]
则有:
\begin{equation}
    \sum_{i=1}^n \left( \prod_{j=1}^m a_{i,j} \right) \le \prod_{j=1}^m \left( \sum_{i=1}^n a_{i,j}^m \right)^{\frac{1}{m}}
\end{equation}
即:
\begin{equation}
    \sum_{i=1}^n a_{i,1} a_{i,2} \cdots a_{i,m} \le \left( \sum_{i=1}^n a_{i,1}^m \right)^{\frac{1}{m}} \left( \sum_{i=1}^n a_{i,2}^m \right)^{\frac{1}{m}} \cdots \left( \sum_{i=1}^n a_{i,m}^m \right)^{\frac{1}{m}}
\end{equation}
当且仅当这 $m$ 组向量两两共线(即成比例)时,等号成立。
\end{thm}
\vspace{2cm}






\newpage 
% 例 7
\begin{example}{}{}
设 $a_{1},a_{2},\dots,a_{n}$ 是正实数,求证:
\begin{equation}
    \prod_{i=1}^{n}(a_{i}^{3}+1)\ge\prod_{i=1}^{n}(a_{i}^{2}a_{i+1}+1),
\end{equation}
其中 $a_{n+1}=a_{1}$。
\end{example}

\newpage 
% 例 8
\begin{example}{}{}
设正实数 $a_{1},a_{2},\dots,a_{n}$ 满足 $a_{1}+a_{2}+\dots+a_{n}=1$。求证:
\begin{equation}
    (a_{1}a_{2}+a_{2}a_{3}+\dots+a_{n}a_{1})\left(\frac{a_{1}}{a_{2}^{2}+a_{2}}+\frac{a_{2}}{a_{3}^{2}+a_{3}}+\dots+\frac{a_{n}}{a_{1}^{2}+a_{1}}\right)\ge\frac{n}{n+1}.
\end{equation}
\end{example}

\newpage 
\section{作业题}

% 作业 1
\begin{homework}{}{}
设 $a_{1},a_{2},\dots,a_{n}$ 是实数,求证:
\begin{equation}
    \left(\sum_{1\le i<j\le n}|a_{i}-a_{j}|\right)^{2}\le\frac{n^{2}-1}{3}\sum_{1\le i<j\le n}(a_{i}-a_{j})^{2}.
\end{equation}
\end{homework}

\newpage 
% 作业 2
\begin{homework}{}{}
设正实数 $a_{1},a_{2},\dots,a_{n},b_{1},b_{2},\dots,b_{n}$ 满足 $a_{i}>b_{i}\ (i=1,2,\dots,n)$ 且 $\prod_{i=1}^{n}a_{i}b_{i}=1$。求证:
\begin{equation}
    \prod_{i=1}^{n}a_{i}-\prod_{i=1}^{n}b_{i}\ge n\sqrt[n]{\prod_{i=1}^{n}(a_{i}-b_{i})}.
\end{equation}
\end{homework}



