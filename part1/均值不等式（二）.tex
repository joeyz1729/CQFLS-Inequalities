\chapter{均值不等式 (二)}
\section{基础知识}
\begin{thm}{对称交叉项}{}
设整数 $n\ge2$,$a_{1},a_{2},\dots,a_{n}$ 是实数,则
\begin{equation}
    \sum_{1\le i<j\le n}a_{i}a_{j}\le\frac{n-1}{2n}\left(\sum_{i=1}^{n}a_{i}\right)^{2}.
\end{equation}
\end{thm}


\newpage 
\begin{thm}{轮换交叉项}{}
(四分之一引理)设整数 $n\ge 4$,$a_{1},a_{2},\dots,a_{n}$ 是非负实数,则
\begin{equation}
    \sum_{i=1}^{n}a_{i}a_{i+1}\le\frac{1}{4}\left(\sum_{i=1}^{n}a_{i}\right)^{2},
\end{equation}
其中 $a_{n+1}=a_{1}$。
\end{thm}


\newpage 
\section{典型例题}
% 例 1
\begin{example}{}{}
给定整数 $n\ge2$。设 $a_{1},a_{2},\dots,a_{n}$ 是实数,求
\begin{equation}
    \sum_{i=1}^{n}a_{i}^{2}+\sum_{1\le i<j\le n}a_{i}a_{j}+\sum_{i=1}^{n}a_{i}
\end{equation}
的最小值。
\end{example}


\newpage
% 例 2
\begin{example}{}{}
给定整数 $n\ge2$。设 $a_{1},a_{2},\dots,a_{n}$ 是实数,求
\begin{equation}
    \sum_{i=1}^{n}a_{i}^{2}+\sum_{i=1}^{n-1}a_{i}a_{i+1}+\sum_{i=1}^{n}a_{i}
\end{equation}
的最小值。
\end{example}


\newpage
% 例 3
\begin{example}{}{}
给定整数 $n\ge3$。设 $a_{1},a_{2},\dots,a_{2n},b_{1},b_{2},\dots,b_{2n}$ 是 $4n$ 个非负实数,满足
\begin{equation}
    a_{1}+a_{2}+\dots+a_{2n}=b_{1}+b_{2}+\dots+b_{2n}>0
\end{equation}
且对任意 $i=1,2,\dots,2n$,有 $a_{i}a_{i+2}\ge b_{i}+b_{i+1}$(这里 $a_{2n+1}=a_{1},a_{2n+2}=a_{2},b_{2n+1}=b_{1}$)。
求 $a_{1}+a_{2}+\dots+a_{2n}$ 的最小值。
\end{example}


\newpage
% 例 4
\begin{example}{}{}
给定整数 $n\ge3$。设实数 $a_{1},a_{2},\dots,a_{n}\ge-1$ 且满足 $a_{1}+a_{2}+\dots+a_{n}=0$,求
\begin{equation}
    a_{1}a_{2}+a_{2}a_{3}+\dots+a_{n-1}a_{n}
\end{equation}
的最大值。
\end{example}


\newpage
% 例 5
\begin{example}{}{}
给定整数 $n\ge4$。求最大的实数 $\lambda$,使得对任意非负实数 $a_{1},a_{2},\dots,a_{n}$,均有
\begin{equation}
    \sum_{i=1}^{n}a_{i}a_{i+1}+\lambda mM\le\frac{1}{4}\left(\sum_{i=1}^{n}a_{i}\right)^{2},
\end{equation}
其中 $a_{n+1}=a_{1}$,$m=\min\{a_{1},a_{2},\dots,a_{n}\}$,$M=\max\{a_{1},a_{2},\dots,a_{n}\}$。
\end{example}


\newpage
% 例 6
\begin{example}{}{}
设正实数 $a_{1},a_{2},\dots,a_{100}$ 满足 $a_{i}+a_{i+1}+a_{i+2}\le1,1\le i\le100$,其中脚标按模 100 理解。求 $\sum_{i=1}^{100}a_{i}a_{i+1}$ 的最大值。
\end{example}


\newpage
% 例 7
\begin{example}{}{}
给定整数 $n\ge3,\ \lambda\in[\frac{1}{2},2]$。设非负实数 $a_{1},a_{2},\dots,a_{n},b_{1},b_{2},\dots,b_{n}$ 满足 $\sum a_i = \sum b_i = 1$。
对 $1\le i\le n$,记 $c_{i}=(\lambda a_{i}+b_{i+1})(\lambda a_{i+1}+b_{i})$,其中 $a_{n+1}=a_{1},b_{n+1}=b_{1}$。求 $c_{1}+c_{2}+\dots+c_{n}$ 的最大值。
\end{example}


\newpage
\section{作业题}

% 作业题 1
\begin{homework}{}{}
设 $a_{1},a_{2},\dots,a_{100}$ 是非负实数,满足:
\begin{enumerate}
    \item $a_{1}+a_{2}+\dots+a_{100}=2$;
    \item $a_{1}a_{2}+a_{2}a_{3}+\dots+a_{100}a_{1}=1$.
\end{enumerate}
求 $a_{1}^{2}+a_{2}^{2}+\dots+a_{100}^{2}$ 的最大值和最小值。
\end{homework}


\newpage
% 作业题 2
\begin{homework}{}{}
给定整数 $n\ge4$。设非负实数 $a_{1},a_{2},\dots,a_{n}$ 满足 $a_{1}+a_{2}+\dots+a_{n}=2$,求
\begin{equation}
    \frac{a_{1}}{a_{2}^{2}+1}+\frac{a_{2}}{a_{3}^{2}+1}+\dots+\frac{a_{n}}{a_{1}^{2}+1}
\end{equation}
的最小值。
\end{homework}


\newpage
% 作业题 3
\begin{homework}{}{}
给定整数 $n\ge4$。设非负实数 $a_{1},a_{2},\dots,a_{n}$ 满足 $a_{1}+a_{2}+\dots+a_{n}=1$,求
\begin{equation}
    a_{1}a_{2}a_{3}+a_{2}a_{3}a_{4}+\dots+a_{n}a_{1}a_{2}
\end{equation}
的最大值。
\end{homework}


\newpage
% 作业题 4
\begin{homework}{}{}
给定整数 $n\ge2$。设集合 $T=\{(i,j) \mid 1\le i < j \le n,\ i \mid j\}$。
对任意满足 $x_{1}+x_{2}+\dots+x_{n}=1$ 的非负实数 $x_{1},x_{2},\dots,x_{n}$,求
\begin{equation}
    \sum_{(i,j)\in T} x_i x_j
\end{equation}
的最大值。
\end{homework}




\newpage 
\section{补充练习题}

 
\begin{problem}{}
已知正实数 $a_1,a_2,\dots,a_n$ 满足 $a_1a_2\dots a_n=1$,证明:
\begin{equation}
(2+a_1)(2+a_2)\dots(2+a_n)\ge 3^n.
\end{equation}
\end{problem}

 
\begin{problem}{}
已知正实数 $a_1,a_2,\dots,a_n$ 满足 $a_1+a_2+\dots+a_n=S$,证明:
\begin{equation}
1+S+\frac{S^2}{2!}+\dots+\frac{S^n}{n!}
\ge
(1+a_1)(1+a_2)\dots(1+a_n).
\end{equation}
\end{problem}


\begin{problem}{17 北大挑战赛}
给定整数 $n$ 和正实数 $a_1,a_2,\dots,a_n$,且满足 $\prod_{i=1}^{k}a_i\ge k!\ (1\le k\le n)$,证明:
\begin{equation}
\frac{2!}{1+a_1}+\frac{3!}{(1+a_1)(2+a_2)}+\dots+
\frac{(n+1)!}{(1+a_1)(2+a_2)\dots(n+a_n)}<3.
\end{equation}
\end{problem}


\begin{problem}{14 北约}
已知正数 $x_1,x_2,\dots,x_n$ 满足 $x_1x_2\dots x_n=1$,证明:
\begin{equation}
(\sqrt2+x_1)(\sqrt2+x_2)\dots(\sqrt2+x_n)\ge(\sqrt2+1)^n.
\end{equation}
\end{problem}


\begin{problem}{18 北大综合营}
设非负实数 $x_1,x_2,\dots,x_n$ 满足 $x_1+x_2+\dots+x_n=1$,求证:对任意 $r>0$ 有
\begin{equation}
(1+rx_1)(1+rx_2)\dots(1+rx_n)\ge(n+r)^n x_1x_2\dots x_n.
\end{equation}
\end{problem}



\begin{problem}{07 女奥}
设整数 $n\ge 4$,非负实数 $a_1,a_2,\dots,a_n$ 满足 $a_1+a_2+\dots+a_n=2$,记
\begin{equation}
P=\frac{a_1}{a_2^2+1}+\frac{a_2}{a_3^2+1}+\dots+\frac{a_{n-1}}{a_n^2+1}+\frac{a_n}{a_1^2+1}.
\end{equation}
求 $P$ 的最小值,并给出相应的取等条件。
\end{problem}


\begin{problem}{}
设整数 $n\ge 3$,且正实数 $a_1,a_2,\dots,a_n$ 满足
\[
\frac{1}{1+a_1^4}+\frac{1}{1+a_2^4}+\dots+\frac{1}{1+a_n^4}=1.
\]
求证:
\begin{equation}
a_1a_2\dots a_n\ge(n-1)^{n/4}.
\end{equation}
\end{problem}



\begin{problem}{}
已知正实数 $x_1,x_2,\dots,x_n$ 满足 $\sum_{i=1}^{n}x_i^2=n$,证明:
\begin{equation}
\frac12\Bigl(\sum_{i=1}^{n}x_i+\sum_{i=1}^{n}\frac{1}{x_i}\Bigr)
\ge n-1+\frac{n}{\sum_{i=1}^{n}x_i}.
\end{equation}
\end{problem}


\begin{problem}{07 保加利亚}
设整数 $n\ge 2$,求常数 $C(n)$ 的最大值,使得对所有满足 $x_i\in(0,1)\ (i=1,2,\dots,n)$ 且 $(1-x_i)(1-x_j)\ge\frac14\ (1\le i<j\le n)$ 的实数 $x_1,x_2,\dots,x_n$,均有
\begin{equation}
\sum_{i=1}^{n}x_i
\ge C(n)\sum_{1\le i<j\le n}\bigl(2x_ix_j+\sqrt{x_ix_j}\bigr).
\end{equation}
\end{problem}


\begin{problem}{}
设 $n$ 为正整数,正实数 $a_1,a_2,\dots,a_n$ 满足
$a_1^2+2a_2^3+\dots+na_n^{n+1}\le 1$,证明:
\begin{equation}
2a_1+3a_2^2+\dots+na_{n-1}^{n-1}+(n+1)a_n^n<3.
\end{equation}
\end{problem}


\begin{problem}{}
设 $n$ 为正整数,$x_1,x_2,\dots,x_n$ 为正实数,证明:
\begin{equation}
1+\sum_{k=1}^{n}(k+1)x_k^k
<\Bigl(1+\sqrt{\sum_{k=1}^{n}kx_k^{k+1}}\Bigr)^2.
\end{equation}
\end{problem}