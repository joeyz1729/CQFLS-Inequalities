\chapter{补充练习题}
% 第 1 题
\begin{problem}{}{}
已知正实数 $a_1,a_2,\dots,a_n$ 满足 $a_1a_2\dots a_n=1$,证明:
\begin{equation}
    (2+a_1)(2+a_2)\dots(2+a_n)\ge 3^n
\end{equation}
\end{problem}

% 第 2 题
\vspace{5cm}
\begin{problem}{}{}
已知正实数 $a_1,a_2,\dots,a_n$ 满足 $a_1+a_2+\dots+a_n=S$,证明:
\begin{equation}
    1+S+\frac{S^2}{2!}+\dots+\frac{S^n}{n!} \ge (1+a_1)(1+a_2)\dots(1+a_n)
\end{equation}
\end{problem}


% 第 3 题
\vspace{5cm}
\begin{problem}{2017 北大挑战赛}{}
给定整数 $n$ 和正实数 $a_1,a_2,\dots,a_n$,且满足 $\prod_{i=1}^{k}a_i\ge k!\ (1\le k\le n)$,证明:
\begin{equation}
    \frac{2!}{1+a_1}+\frac{3!}{(1+a_1)(2+a_2)}+\dots+\frac{(n+1)!}{(1+a_1)(2+a_2)\dots(n+a_n)}<3
\end{equation}
\end{problem}

% 第 4 题
\vspace{5cm}
\begin{problem}{2014 北约自主招生}{}
已知正数 $x_1,x_2,\dots,x_n$ 满足 $x_1x_2\dots x_n=1$,证明:
\begin{equation}
    (\sqrt2+x_1)(\sqrt2+x_2)\dots(\sqrt2+x_n)\ge(\sqrt2+1)^n
\end{equation}
\end{problem}
% 第 5 题
\vspace{5cm}
\begin{problem}{2018 北大综合营}{}
设非负实数 $x_1,x_2,\dots,x_n$ 满足 $x_1+x_2+\dots+x_n=1$,求证:对任意 $r>0$ 有
\begin{equation}
    (1+rx_1)(1+rx_2)\dots(1+rx_n)\ge(n+r)^n x_1x_2\dots x_n
\end{equation}
\end{problem}

% % 第 6 题
% \vspace{5cm}
% \begin{problem}{}{}
% 设整数 $n\ge 4$,非负实数 $a_1,a_2,\dots,a_n$ 满足 $\sum_{i=1}^{n}a_i=1$,求
% $a_1a_2+a_2a_3+\dots+a_{n-1}a_n+a_na_1$ 的最大值。
% \end{problem}


% 第 7 题
\vspace{5cm}
\begin{problem}{2007 女奥 (CGMO)}{}
设整数 $n\ge 4$,非负实数 $a_1,a_2,\dots,a_n$ 满足 $a_1+a_2+\dots+a_n=2$,记
\begin{equation}
    P=\frac{a_1}{a_2^2+1}+\frac{a_2}{a_3^2+1}+\dots+\frac{a_{n-1}}{a_n^2+1}+\frac{a_n}{a_1^2+1}
\end{equation}
求 $P$ 的最小值,并给出相应的取等条件。
\end{problem}


% 第 8 题
\vspace{5cm}
\begin{problem}{}{}
设整数 $n\ge 3$,且正实数 $a_1,a_2,\dots,a_n$ 满足
\[
\frac{1}{1+a_1^4}+\frac{1}{1+a_2^4}+\dots+\frac{1}{1+a_n^4}=1
\]
求证:
\begin{equation}
    a_1a_2\dots a_n\ge(n-1)^{n/4}
\end{equation}
\end{problem}


% 第 9 题
\vspace{5cm}
\begin{problem}{1998 IMO 预选}{}
设 $a_1,a_2,\dots,a_n\ (n\ge 2)$ 均为正实数,且 $a_1+a_2+\dots+a_n<1$,证明:
\begin{equation}
    \frac{a_1a_2\dots a_n\bigl[1-(a_1+a_2+\dots+a_n)\bigr]}
    {(a_1+a_2+\dots+a_n)(1-a_1)(1-a_2)\dots(1-a_n)}
    \le\frac{1}{n^{n+1}}
\end{equation}
\end{problem}


% 第 10 题
\vspace{5cm}
\begin{problem}{}{}
已知正实数 $x_1,x_2,\dots,x_n$ 满足 $\sum_{i=1}^{n}x_i^2=n$,证明:
\begin{equation}
    \frac12\Bigl(\sum_{i=1}^{n}x_i+\sum_{i=1}^{n}\frac{1}{x_i}\Bigr)
    \ge n-1+\frac{n}{\sum_{i=1}^{n}x_i}
\end{equation}
\end{problem}


% 第 11 题
\vspace{5cm}
\begin{problem}{2007 保加利亚}{}
设整数 $n\ge 2$,求常数 $C(n)$ 的最大值,使得对所有满足 $x_i\in(0,1)\ (i=1,2,\dots,n)$ 且 $(1-x_i)(1-x_j)\ge\frac14\ (1\le i<j\le n)$ 的实数 $x_1,x_2,\dots,x_n$,均有
\begin{equation}
    \sum_{i=1}^{n}x_i \ge C(n)\sum_{1\le i<j\le n}\bigl(2x_ix_j+\sqrt{x_ix_j}\bigr)
\end{equation}
\end{problem}





% 第 1 题
\vspace{5cm}
\begin{problem}{}{}
设 $a_i,b_i>0,\ i=1,2,\dots,n$,且 $a_1+a_2+\dots+a_n=b_1+b_2+\dots+b_n$,证明:
\begin{equation}
    \frac{a_1^2}{a_1+b_1}+\frac{a_2^2}{a_2+b_2}+\dots+\frac{a_n^2}{a_n+b_n} \ge\frac12(a_1+a_2+\dots+a_n)
\end{equation}
\end{problem}


% 第 2 题

\vspace{5cm}
\begin{problem}{2010 浙大自招}{}
设整数 $n\ge 2$,且正实数 $x_1,x_2,\dots,x_n$ 满足 $x_1+x_2+\dots+x_n=1$,证明:
\begin{equation}
    \frac{1}{x_1-x_1^3}+\frac{1}{x_2-x_2^3}+\dots+\frac{1}{x_n-x_n^3}>4
\end{equation}
\end{problem}


% 第 3 题

\vspace{5cm}
\begin{problem}{2002 女奥 (CGMO)}{}
设整数 $n\ge 2$,且 $P_1,P_2,\dots,P_n$ 是 $1,2,\dots,n$ 的任意排列,证明:
\begin{equation}
    \frac{1}{P_1+P_2}+\frac{1}{P_2+P_3}+\dots+\frac{1}{P_{n-1}+P_n}>\frac{n-1}{n+2}
\end{equation}
\end{problem}


% 第 4 题

\vspace{5cm}
\begin{problem}{2011 甘肃预赛}{}
已知正实数 $a_1,a_2,\dots,a_n$ 满足 $a_1+a_2+\dots+a_n=1$,证明:
\begin{equation}
    \Bigl(a_1+\frac{1}{a_1}\Bigr)^2+\Bigl(a_2+\frac{1}{a_2}\Bigr)^2+\dots+\Bigl(a_n+\frac{1}{a_n}\Bigr)^2 \ge\frac{(n^2+1)^2}{n}
\end{equation}
\end{problem}


% % 第 5 题

% \vspace{5cm}
% \begin{problem}{}{}
% 给定整数 $n\ge 2$,非负实数 $a_1,a_2,\dots,a_n$ 满足 $\sum_{i=1}^n a_i=\sum_{i=1}^n a_i^3$,且 $a_{n+1}=a_1$,证明:
% \begin{equation}
%     \frac{1}{a_1^2-a_2+n}+\frac{1}{a_2^2-a_3+n}+\dots+\frac{1}{a_n^2-a_{n+1}+n}\ge 1
% \end{equation}
% \end{problem}


% 第 6 题

\vspace{5cm}
\begin{problem}{}{}
设 $a_1,a_2,\dots,a_n$ 是实数,证明:
\begin{equation}
    \sqrt[3]{a_1^3+a_2^3+\dots+a_n^3} \le\sqrt{a_1^2+a_2^2+\dots+a_n^2}
\end{equation}
\end{problem}


% 第 7 题

\vspace{5cm}
\begin{problem}{}{}
设 $n\ge 2,\ n\in\mathbb N^+$,证明:
\begin{equation}
    1\cdot\sqrt{C_n^1}+2\cdot\sqrt{C_n^2}+\dots+n\cdot\sqrt{C_n^n} <\sqrt{2^{n-1}\,n^3}
\end{equation}
\end{problem}


% 第 8 题

\vspace{5cm}
\begin{problem}{}{}
设整数 $n\ge 3$,正实数 $a_1,a_2,\dots,a_n$ 满足 $a_n\ge a_1+a_2+\dots+a_{n-1}$,证明:
\begin{equation}
    \Bigl(\frac{1}{a_1}+\frac{1}{a_2}+\dots+\frac{1}{a_n}\Bigr)(a_1+a_2+\dots+a_n) \ge 2(n-1)^2+2
\end{equation}
\end{problem}


% 第 9 题

\vspace{5cm}
\begin{problem}{}{}
已知正实数 $x_1,x_2,\dots,x_{n+1}$ 满足 $x_{n+1}=x_1+x_2+\dots+x_n$,证明:
\begin{equation}
    \Bigl(\sum_{i=1}^n\sqrt{x_i(x_{n+1}-x_i)}\Bigr)^2 \le(n-1)\,x_{n+1}^2
\end{equation}
\end{problem}


% 第 10 题

\vspace{5cm}
\begin{problem}{}{}
已知 $a_1,a_2,\dots,a_n$ 为正实数,证明:
\begin{equation}
    \frac{(a_1+a_2+\dots+a_n)^2}{2(a_1^2+a_2^2+\dots+a_n^2)} \le\frac{a_1}{a_2+a_3}+\frac{a_2}{a_3+a_4}+\dots+\frac{a_n}{a_1+a_2}
\end{equation}
\end{problem}


% 第 11 题

\vspace{5cm}
\begin{problem}{}{}
给定整数 $n\ge 2$,非负实数 $x_1,x_2,\dots,x_n$ 满足
\begin{equation}
    \sum_{i=1}^n x_i+\sum_{1\le i<j\le n}x_ix_j=n+C_n^2,
\end{equation}
其中 $C_a^b=\dfrac{a!}{b!(a-b)!}$。证明:$\sum_{i=1}^n x_i\ge n$。
\end{problem}


% 第 12 题

\vspace{5cm}
\begin{problem}{2017 HMMT}{}
设 $x_1,x_2,\dots,x_{2017}$ 均为实数,求出最大的实数 $c$,使得下列不等式成立:
\begin{equation}
    \sum_{i=1}^{2016}x_i(x_i+x_{i+1})\ge c\,x_{2017}^2
\end{equation}
\end{problem}


% 第 13 题

\vspace{5cm}
\begin{problem}{2019 欧洲杯}{}
已知数列 $\{x_n\}$ 满足 $x_1=\sqrt2,\ x_{n+1}=x_n+\dfrac{1}{x_n}\ (n\in\mathbb N^+)$,证明:
\begin{equation}
    \sum_{k=1}^{2019}\frac{x_k^2}{2x_kx_{k+1}-1}> \frac{2019^2}{x_{2019}^2+\dfrac{1}{x_{2019}^2}}
\end{equation}
\end{problem}


% 第 14 题

\vspace{5cm}
\begin{problem}{}{}
已知正实数 $x_1,x_2,\dots,x_n$ 满足 $x_1+x_2+\dots+x_n=1$,证明:
\begin{equation}
    \sum_{i=1}^n\frac{(n-1)\sum_{j\ne i}x_j^2+x_i}{1+\sum_{j\ne i}x_j} \ge\frac{n^2-n+1}{2n-1}
\end{equation}
\end{problem}


% 第 15 题

\vspace{5cm}
\begin{problem}{1996 波兰}{}
设整数 $n\ge 2$,正数 $a_1,\dots,a_n,x_1,\dots,x_n$ 满足 $\sum_{i=1}^n a_i=1,\ \sum_{i=1}^n x_i=1$,证明:
\begin{equation}
    2\sum_{1\le i<j\le n}x_ix_j\le\frac{n-2}{n-1}+\sum_{i=1}^n\frac{a_ix_i^2}{1-a_i},
\end{equation}
并指出等号成立的充要条件。
\end{problem}


% 第 16 题

\vspace{5cm}
\begin{problem}{}{}
设 $a_i>0,\,b_i>0,\ a_ib_i=c_i^2+d_i^2\ (i=1,2,\dots,n)$,证明:
\begin{equation}
    \sum_{i=1}^n a_i\cdot\sum_{i=1}^n b_i \ge\Bigl(\sum_{i=1}^n c_i\Bigr)^2+\Bigl(\sum_{i=1}^n d_i\Bigr)^2
\end{equation}
\end{problem}


% 第 17 题

\vspace{5cm}
\begin{problem}{2002 罗马尼亚}{}
设整数 $n\ge 4$,正实数 $a_1,a_2,\dots,a_n$ 满足 $a_1^2+a_2^2+\dots+a_n^2=1$,求证:
\begin{equation}
    \frac{a_1}{a_2^2+1}+\frac{a_2}{a_3^2+1}+\dots+\frac{a_n}{a_1^2+1} \ge\frac45\Bigl(a_1\sqrt{a_1}+a_2\sqrt{a_2}+\dots+a_n\sqrt{a_n}\Bigr)^2
\end{equation}
\end{problem}


% 第 18 题

\vspace{5cm}
\begin{problem}{1998 罗马尼亚}{}
已知正实数 $x_1,x_2,\dots,x_n$ 满足 $x_1x_2\dots x_n=1$,证明:
\begin{equation}
    \frac{1}{n-1+x_1}+\frac{1}{n-1+x_2}+\dots+\frac{1}{n-1+x_n}\le 1
\end{equation}
\end{problem}


% 第 19 题

\vspace{5cm}
\begin{problem}{2014 东南}{}
设整数 $n\ge 2$,正实数 $x_1,x_2,\dots,x_n$ 满足 $x_1+\dots+x_n=1$,且 $x_{n+1}=x_1$,求证:
\begin{equation}
    \frac{x_1}{x_2-x_2^3}+\frac{x_2}{x_3-x_3^3}+\dots+\frac{x_n}{x_{n+1}-x_{n+1}^3} \ge\frac{n^3}{n^2-1}
\end{equation}
\end{problem}





% 第 22 题

\vspace{5cm}
\begin{problem}{2010 伊朗改}{}
设 $a_1,a_2,\dots,a_n$ 是正实数,证明:
\begin{equation}
    \sum_{i=1}^n\frac{1}{a_i^2}+\frac{1}{(a_1+a_2+\dots+a_n)^2} \ge\frac{n^3+1}{(n^2+1)^2}\Bigl(\sum_{i=1}^n\frac{1}{a_i}+\frac{1}{a_1+a_2+\dots+a_n}\Bigr)^2
\end{equation}
\end{problem}


% 第 23 题

\vspace{5cm}
\begin{problem}{}{}
设 $x_1,x_2,\dots,x_n$ 是正实数,且 $x_{n+1}=x_1$,证明:
\begin{equation}
    \frac{x_1^2}{x_2}+\frac{x_2^2}{x_3}+\dots+\frac{x_n^2}{x_{n+1}} \ge(x_1+x_2+\dots+x_n)+\frac{4(x_1-x_n)^2}{x_1+x_2+\dots+x_n}
\end{equation}
\end{problem}
