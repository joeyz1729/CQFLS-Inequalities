\chapter{补充练习题}

\begin{problem}{}
(18 俄罗斯)给定正实数 $x_1,x_2,\dots,x_n$,其中整数 $n\ge 2$,证明:
\begin{equation}
\frac{1+x_1^2}{1+x_1x_2}+\frac{1+x_2^2}{1+x_2x_3}+\dots+\frac{1+x_n^2}{1+x_nx_1}\ge n.
\end{equation}
\end{problem}


\begin{problem}{}
给定正实数 $a_1,a_2,\dots,a_n$,其中整数 $n\ge 2$,证明:
\begin{equation}
\frac{a_1^2+1}{a_1a_2+1}+\frac{a_2^2+1}{a_2a_3+1}+\dots+\frac{a_n^2+1}{a_na_1+1}
\le\frac{a_1}{a_2}+\frac{a_2}{a_3}+\dots+\frac{a_n}{a_1}.
\end{equation}
\end{problem}


\begin{problem}{}
(16 新加坡)设 $a_1,a_2,\dots,a_n$ 为正实数,且 $a_{n+1}=a_1$,证明:
\begin{equation}
\frac{a_2}{a_1}+\frac{a_3}{a_2}+\dots+\frac{a_{n+1}}{a_n}
\ge\sqrt{\frac{1+a_2^2}{1+a_1^2}}+\sqrt{\frac{1+a_3^2}{1+a_2^2}}+\dots+\sqrt{\frac{1+a_{n+1}^2}{1+a_n^2}}.
\end{equation}
\end{problem}


\begin{problem}{}
设正实数 $a_1,a_2,\dots,a_n$ 满足 $a_1+a_2+\dots+a_n=1$,证明:
\begin{equation}
\frac{a_1}{a_2}+\frac{a_2}{a_3}+\dots+\frac{a_n}{a_1}
\ge\frac{1-a_2}{1-a_1}+\frac{1-a_3}{1-a_2}+\dots+\frac{1-a_1}{1-a_n}.
\end{equation}
\end{problem}


\begin{problem}{}
设整数 $n\ge 3$,证明:对于正实数 $x_1\le x_2\le\dots\le x_n$,有
\begin{equation}
\frac{x_1x_2}{x_3}+\frac{x_2x_3}{x_4}+\dots+\frac{x_{n-1}x_n}{x_1}+\frac{x_nx_1}{x_2}
\ge x_1+x_2+\dots+x_n.
\end{equation}
\end{problem}


\begin{problem}{}
已知正实数 $a_1,a_2,\dots,a_n$ 满足 $a_1+a_2+\dots+a_n=1$,记
\[
b_k=\frac{a_k}{a_k^2+a_ka_{k+1}+a_{k+1}^2},\quad k=1,2,\dots,n,
\]
其中 $a_{n+1}=a_1$,证明:
\begin{equation}
(a_1^4+a_2^4+\dots+a_n^4)(b_1^2+b_2^2+\dots+b_n^2)\ge\frac19.
\end{equation}
\end{problem}


\begin{problem}{}
设正整数 $n\ge 2$,且 $a_1,a_2,\dots,a_n,b_1,b_2,\dots,b_n$ 是正实数,证明:
\begin{equation}
\sum_{k=1}^n\frac{a_k}{\sum_{\substack{1\le j\le n\\ j\ne k}}a_jb_j}
\ge\frac{4}{b_1+b_2+\dots+b_n}.
\end{equation}
\end{problem}


\begin{problem}{}
已知 $a_1,a_2,\dots,a_n$ 为正实数,证明:
\begin{equation}
\frac{1}{1+a_1}+\frac{2}{1+a_1+a_2}+\dots+\frac{n}{1+a_1+a_2+\dots+a_n}
\le\frac{n}{2}\sqrt{\frac{1}{a_1}+\frac{1}{a_2}+\dots+\frac{1}{a_n}}.
\end{equation}
\end{problem}


\begin{problem}{}
已知实数 $a_1,a_2,\dots,a_n$ 满足 $0\le a_i\le 1\ (i=1,2,\dots,n)$,证明:
\begin{equation}
\sum_{i=1}^n\frac{a_i}{1+\sum_{\substack{j=1\\ j\ne i}}^n a_j}+\prod_{i=1}^n(1-a_i)\le 1.
\end{equation}
\end{problem}


\begin{problem}{}
设 $a_1,a_2,\dots,a_n$ 均大于 1,且 $x_0,x_1,\dots,x_n$ 满足 $x_0=1,\ x_k=\dfrac{1}{1+a_kx_{k-1}}\ (1\le k\le n)$,证明:
\begin{equation}
\sum_{k=1}^n x_k>\frac{n^2\bigl(1+\sum_{k=1}^n a_k\bigr)}{n^2+\bigl(1+\sum_{k=1}^n a_k\bigr)^2}.
\end{equation}
\end{problem}


\begin{problem}{}
(17 女奥)设 $a_i\ge 0,\ x_i\in\mathbb R,\ i=1,2,\dots,n$,证明:
\begin{equation}
\Bigl[\Bigl(1-\sum_{i=1}^n a_i\cos x_i\Bigr)^2+\Bigl(1-\sum_{i=1}^n a_i\sin x_i\Bigr)^2\Bigr]^2
\ge 4\Bigl(1-\sum_{i=1}^n a_i\Bigr)^3.
\end{equation}
\end{problem}


\begin{problem}{}
设 $n$ 为正整数,实数 $a_1,a_2,\dots,a_n$ 满足 $\sum_{i=1}^n a_i^2=1$,证明:
\begin{equation}
\sum_{1\le i\cdot j \le n}a_ia_j<2\sqrt{n}.
\end{equation}
\end{problem}



\begin{problem}{}
设正整数 $n\ge 2$,$a_1,a_2,\dots,a_n$ 是实数,证明:
\begin{equation}
\frac{3}{n^2-1}\Bigl[\sum_{k=1}^n(2k-n-1)a_k\Bigr]^2
+\Bigl(\sum_{k=1}^n a_k\Bigr)^2
\le n\sum_{k=1}^n a_k^2.
\end{equation}
\end{problem}


\begin{problem}{}
给定正整数 $n$,设正实数 $a_1,a_2,\dots,a_n$ 满足
\begin{equation}
\sum_{i=1}^n a_i=\frac{2}{n-1}\sum_{1\le i<j\le n}a_ia_j,
\end{equation}
记 $x_i=\sum_{j=1}^n a_j-a_i\ (1\le i\le n)$,证明:
\begin{equation}
\sum_{i=1}^n\frac{1}{1+x_i}\le 1.
\end{equation}
\end{problem}


\begin{problem}{}
(19 浙江预赛)设 $a_i,b_i>0\ (1\le i\le n+1)$,$b_{i+1}-b_i\ge\delta>0$($\delta$ 为常数),若 $\sum_{i=1}^n a_i=1$,证明:
\begin{equation}
\sum_{i=1}^n\frac{i\,\sqrt[i]{a_1a_2\dotsm a_ib_1b_2\dotsm b_i}}{b_ib_{i+1}}
<\frac{1}{\delta}.
\end{equation}
\end{problem}


\begin{problem}{}
设 $x_1,x_2,\dots,x_n$ 为非负实数,证明
\begin{equation}
\frac{x_1}{\left(1+x_1+x_2+\cdots+x_n\right)^2}+\frac{x_2}{\left(1+x_2+x_3+\cdots+x_n\right)^2}+\cdots+\frac{x_n}{\left(1+x_n\right)^2} \leq k_n^2
\end{equation}
其中数列 $\{k_n\}$ 满足 $k_1=\frac12,\ k_{n+1}=\dfrac{k_n^2+1}{2}$。
\end{problem}




\begin{problem}{}
已知 $a_1,a_2,\dots,a_n$ 是正实数,且 $n\ge 3,\ S=\sum_{i=1}^n a_i$,证明:
\begin{equation}
\sum_{i=1}^n\Bigl[\frac{a_i}{S-a_i}+\sqrt[n-1]{\Bigl(\frac{(n-1)a_i}{S-a_i}\Bigr)^{n-2}}\Bigr]
\ge\frac{n^2}{n-1}.
\end{equation}
\end{problem}


\begin{problem}{11 西班牙改}
已知 $a_1,a_2,\dots,a_n$ 是正实数,且 $n\ge 3$,设 $S=\sum_{i=1}^n a_i$,证明:
\begin{equation}
\sum_{i=1}^n\frac{a_i}{S-a_i}+\frac{2}{n-1}\sqrt{\frac{\sum_{1\le i<j\le n}2a_ia_j}{(n-1)\sum_{i=1}^n a_i^2}}
\ge\frac{n+2}{n-1}.
\end{equation}
\end{problem}


\begin{problem}{10 地中海}
已知 $n>2$,正实数 $a_1,a_2,\dots,a_n$ 满足 $a_1+a_2+\dots+a_n=1$,证明:
\begin{equation}
\frac{a_2a_3\dotsm a_n}{a_1+n-2}+\frac{a_1a_3\dotsm a_n}{a_2+n-2}+\dots+\frac{a_1a_2\dotsm a_{n-1}}{a_n+n-2}
\le\frac{1}{(n-1)^2},
\end{equation}
\end{problem}


\begin{problem}{}
已知 $a_1,a_2,\dots,a_n$ 是正实数,且 $n\ge 3$,设 $S=\sum_{i=1}^n a_i^2$,证明:
\begin{equation}
\Bigl(\sum_{i=1}^n a_i\Bigr)^2\sum_{i=1}^n\frac{1}{S+a_i^2}
\le\frac{n^3}{n+1}.
\end{equation}
\end{problem}


\begin{problem}{}
已知实数 $a_1,a_2,\dots,a_n$ 满足 $\sum_{i=1}^n a_i=0$,证明:
\begin{equation}
\sum_{i=1}^n\frac{(a_i+1)^2}{a_i^2+n-1}
\ge\frac{n}{n-1}.
\end{equation}
\end{problem}


\begin{problem}{}
已知实数 $x_1,x_2,\dots,x_n$ 满足 $\sum_{i=1}^n x_i=0$,证明:
\begin{equation}
\sum_{i=1}^n\frac{(n-2)x_i^2+2x_i}{(n-1)x_i^2+1}
\ge 0.
\end{equation}
\end{problem}


\begin{problem}{}
(86 苏联)设 $a_1,a_2,\dots,a_n$ 均为正实数,证明:
\begin{equation}
\frac{1}{a_1}+\frac{2}{a_1+a_2}+\dots+\frac{n}{a_1+a_2+\dots+a_n}
<4\Bigl(\frac{1}{a_1}+\frac{1}{a_2}+\dots+\frac{1}{a_n}\Bigr).
\end{equation}
\end{problem}


\begin{problem}{}
设 $x_1,x_2,\dots,x_n$ 为非负实数,证明:
\begin{equation}
\sum_{k=1}^n\Bigl(\frac{x_1+x_2+\dots+x_k}{k}\Bigr)^2
\le\sum_{k=1}^n(k+1)x_k^2.
\end{equation}
\end{problem}


\begin{problem}{}
(05 韩国)设 $x_1,x_2,\dots,x_n$ 为非负实数,证明:
\begin{equation}
\sum_{k=1}^n\Bigl(\frac{x_1+x_2+\dots+x_k}{k}\Bigr)^2
\le 4\sum_{k=1}^n x_k^2.
\end{equation}
\end{problem}


\begin{problem}{}
设 $0<x_n<x_{n-1}<\dots<x_1<x_0<1$,证明:
\begin{equation}
\sum_{i=1}^n\frac{x_i^2}{x_{i-1}-x_i}
>\frac12\sum_{i=1}^n i x_i-1.
\end{equation}
\end{problem}


\begin{problem}{}
给定实数 $p_1,p_2,\dots,p_n$,对于满足 $p_1x_1+p_2x_2+\dots+p_nx_n=1$ 的实数 $x_1,x_2,\dots,x_n$,记
\begin{equation}
P=\sum_{i=1}^n x_i^2+\Bigl(\sum_{i=1}^n x_i\Bigr)^2,
\end{equation}
求 $P$ 的最小值,并给出相应的取等条件。
\end{problem}


\begin{problem}{}
设 $x_1,x_2,\dots,x_n$ 为正实数,证明:
\begin{equation}
\sum_{i=1}^n x_i^2
\ge\frac{1}{n+1}\Bigl(\sum_{i=1}^n x_i\Bigr)^2
+\frac{12}{n(n+1)(n+2)(3n+1)}\Bigl(\sum_{i=1}^n i x_i\Bigr)^2.
\end{equation}
\end{problem}


\begin{problem}{}
给定整数 $n\ge 3$,求最大的实数 $\lambda$,使得只要正实数 $a_1,a_2,\dots,a_n$ 满足
\[
a_1^2+a_2^2+\dots+a_n^2<\lambda\,(a_1+a_2+\dots+a_n)^2,
\]
就有 $a_1,a_2,\dots,a_n$ 中任意三个数均可作为某个三角形的边长。
\end{problem}