% chapters/18_induction_1.tex

\chapter{归纳法}

\section{归纳法与不等式 (一)}
凡是与正整数有关的命题都可以尝试用归纳法证明,在不等式中也是如此,本讲主要介绍第一数学归纳法的一些例子。

% === 例题部分 ===

% 例 1
\begin{problem}{}{}
设整数 $n\ge2$,$a_{1},a_{2},\dots,a_{n}\in[0,1]$。求证:
\begin{equation}
    \sum_{i=1}^{n}a_{i}-\sum_{1\le i<j\le n}a_{i}a_{j}\le1.
\end{equation}
\end{problem}
\newpage

% 例 2
\begin{problem}{}{}
设整数 $n\ge2$,实数 $a_{1},a_{2},\dots,a_{n}\in[0,1]$。求证:
\begin{equation}
    \prod_{i=1}^{n}(a_{i}^{2}-a_{i}a_{i+1}+1)\ge1,
\end{equation}
其中 $a_{n+1}=a_{1}$。
\end{problem}
\newpage

% 例 3
\begin{problem}{}{}
设整数 $n\ge2$,实数 $a_{1},a_{2},\dots,a_{n}\ge1$,且满足 $|a_{i}-a_{i+1}|\le1\ (1\le i\le n-1)$。求证:
\begin{equation}
    \sum_{i=1}^{n}\frac{a_{i}}{a_{i+1}}\le 2n-H_{n},
\end{equation}
其中 $a_{n+1}=a_{1}$,$H_{n}=1+\frac{1}{2}+\dots+\frac{1}{n}$。
\end{problem}
\newpage

% 例 4
\begin{problem}{}{}
设整数 $n\ge2$,正实数 $a_{1}\ge a_{2}\ge\dots\ge a_{n}$。求证:
\begin{equation}
    \sum_{i=1}^{n}\frac{a_{i}}{a_{i+1}}-n\le\frac{1}{2a_{1}a_{n}}\sum_{i=1}^{n}(a_{i}-a_{i+1})^{2},
\end{equation}
其中 $a_{n+1}=a_{1}$。
\end{problem}
\newpage

% 例 5
\begin{problem}{}{}
设整数 $n\ge2$,正实数 $a_{1}\le a_{2}\le\dots\le a_{n}$。求证:
\begin{equation}
    \sum_{1\le i<j\le n}(a_{i}+a_{j})^{2}\left(\frac{1}{i^{2}}+\frac{1}{j^{2}}\right)\ge4(n-1)\sum_{i=1}^{n}\frac{a_{i}^{2}}{i^{2}}.
\end{equation}
\end{problem}
\newpage

% 例 6
\begin{problem}{}{}
设整数 $n\ge3$,非负实数 $a_{1},a_{2},\dots,a_{n}$ 满足 $a_{1}+a_{2}+\dots+a_{n}=1$。求证:
\begin{equation}
    a_{1}^{2}a_{2}+a_{2}^{2}a_{3}+\dots+a_{n}^{2}a_{1}\le\frac{4}{27}.
\end{equation}
\end{problem}
\newpage

% 例 7
\begin{problem}{}{}
设正实数 $a_{1}\le a_{2}\le\dots\le a_{n}$ 满足 $a_{1}a_{2}\dots a_{n}=1$。求证:
\begin{equation}
    \sum_{i=1}^{n}\frac{1}{2^{i}(1+a_{i}^{2^{i}})}\ge\frac{1}{2}-\frac{1}{2^{n+1}}.
\end{equation}
\end{problem}
\newpage

% 例 8
\begin{problem}{}{}
设整数 $n\ge2$,正实数 $a_{1},a_{2},\dots,a_{n}$ 满足 $a_{i}a_{j}\le t^{|i-j|}$ 对任意 $1\le i,j\le n$ 成立,其中 $t\in(0,1)$。求证:
\begin{equation}
    a_{1}+a_{2}+\dots+a_{n}<\frac{1}{1-\sqrt{t}}.
\end{equation}
\end{problem}
\newpage

% === 作业题部分 ===

% 作业题 1
\begin{problem}{作业题 1}{}
设整数 $n\ge2$,$a_{1},a_{2},\dots,a_{n}\in[0,1]$。求证:
\begin{equation}
    \sum_{i=1}^{n}a_{i}-\sum_{i=1}^{n}a_{i}a_{i+1}\le\left[\frac{n}{2}\right],
\end{equation}
其中 $a_{n+1}=a_{1}$。
\end{problem}
\newpage

% 作业题 2
\begin{problem}{作业题 2}{}
设整数 $n\ge2$,$a_{1},a_{2},\dots,a_{n}\in(0,1)$。求证:
\begin{equation}
    \frac{\sqrt{1-a_{1}}}{a_{1}}+\frac{\sqrt{1-a_{2}}}{a_{2}}+\dots+\frac{\sqrt{1-a_{n}}}{a_{n}}<\frac{\sqrt{n-1}}{a_{1}a_{2}\dots a_{n}}.
\end{equation}
\end{problem}
\newpage

% 作业题 3
\begin{problem}{作业题 3}{}
设整数 $n\ge4$,正实数 $a_{1},a_{2},\dots,a_{n}$ 满足 $a_{1}a_{2}\dots a_{n}=1$。求证:
\begin{equation}
    \frac{1}{\sqrt{1+a_{1}}}+\frac{1}{\sqrt{1+a_{2}}}+\dots+\frac{1}{\sqrt{1+a_{n}}}<n-1.
\end{equation}
\end{problem}
\newpage

% 作业题 4
\begin{problem}{作业题 4}{}
设整数 $n\ge2$,正实数 $a_{1},a_{2},\dots,a_{n}$ 满足 $a_{i}a_{j}\le 4^{-|i-j|}$ 对任意 $1\le i,j\le n$ 成立。求证:
\begin{equation}
    a_{1}+a_{2}+\dots+a_{n}<\frac{5}{3}.
\end{equation}
\end{problem}
\newpage


% chapters/19_induction_2.tex

\section{归纳法与不等式 (二)}

本讲继续介绍用归纳法证明不等式,包括第二数学归纳法、加强数学归纳法、反向数学归纳法,以及几个经典的不等式。

% === 例题部分 ===

% 例 1
\begin{problem}{}{}
设实数 $a_{i},b_{i}\ (i=0,1,\dots,2n)$ 满足:
\begin{enumerate}
    \item 对 $i=0,1,\dots,2n-1$,有 $a_{i}+a_{i+1}\ge0$;
    \item 对 $j=0,1,\dots,n-1$,有 $a_{2j+1}\le0$;
    \item 对 $0\le p\le q\le n$,有 $\sum_{k=2p}^{2q}b_{k}>0$.
\end{enumerate}
求证:$\sum_{i=0}^{2n}(-1)^{i}a_{i}b_{i}\ge0$.
\end{problem}
\newpage

% 例 2
\begin{problem}{}{}
设 $a_{1},a_{2},\dots,a_{n}$ 是不全为零的非负实数,对 $1\le k\le n$,记
\begin{equation}
    m_{k}=\max_{1\le l\le k}\frac{a_{k-l+1}+a_{k-l+2}+\dots+a_{k}}{l}.
\end{equation}
求证:对任意正实数 $\alpha$,满足 $m_{k}>\alpha$ 的 $k$ 少于 $\frac{a_{1}+a_{2}+\dots+a_{n}}{\alpha}$ 个。
\end{problem}
\newpage

% 例 3
\begin{problem}{}{}
给定整数 $n\ge2$。设非负实数 $a_{1},a_{2},\dots,a_{n}$ 满足 $a_{1}\ge a_{2}\ge\dots\ge a_{n},\ a_{1}+a_{2}+\dots+a_{n}\le n$。求
\begin{equation}
    a_{1}+a_{1}a_{2}+a_{1}a_{2}a_{3}+\dots+a_{1}a_{2}\dots a_{n}
\end{equation}
的最小值。
\end{problem}
\newpage

% 例 4
\begin{problem}{}{}
设 $a_{1},a_{2},a_{3},\dots$ 是实数列,满足存在正整数 $N$,使得对任意 $n\ge N$ 都有 $a_{n}=1$。已知对任意整数 $n\ge2$ 都有 $a_{n}\le a_{n-1}+\frac{1}{2^{n}}a_{2n}$。
求证:对任意正整数 $k$,都有 $a_{k}>1-\frac{1}{2^{k}}$。
\end{problem}
\newpage

% 例 5
\begin{problem}{}{}
对实数列 $\{a_{n}\}$,定义数列 $\{b_{n}\}$ 如下:
\begin{equation}
    b_{1}=a_{1},\quad b_{n+1}=a_{n+1}-\left(\sum_{i=1}^{n}a_{i}^{2}\right)^{\frac{1}{2}},\quad n\ge1.
\end{equation}
求最小的正实数 $\lambda$,使得对任意实数列 $\{a_{n}\}$ 以及任意正整数 $n$,都有
\begin{equation}
    \frac{1}{n}\sum_{i=1}^{n}a_{i}^{2}\le\sum_{i=1}^{n}\lambda^{n-i}b_{i}^{2}.
\end{equation}
\end{problem}
\newpage

% 例 6
\begin{problem}{}{}
求最大的正实数 $\lambda$,使得对任意正整数 $n$ 以及任意正实数 $a_{1},a_{2},\dots,a_{n}$,都有
\begin{equation}
    1+\sum_{k=1}^{n}\frac{1}{a_{k}^{2}}\ge\lambda\sum_{k=1}^{n}\frac{1}{(1+\sum_{i=1}^{k}a_{i})^{2}}.
\end{equation}
\end{problem}
\newpage

% 例 7
\begin{problem}{牛顿不等式 (Newton's Inequality)}{}
设 $a_{1},a_{2},\dots,a_{n}$ 是实数,对 $1\le k\le n$ 记
\begin{equation}
    S_{k}=\frac{\sum_{1\le i_{1}<i_{2}<\dots<i_{k}\le n}a_{i_{1}}a_{i_{2}}\dots a_{i_{k}}}{C_{n}^{k}}.
\end{equation}
则 $S_{k-1}S_{k+1}\le S_{k}^{2}$,其中 $S_{0}=1$。当且仅当 $a_{1}=a_{2}=\dots=a_{n}$ 时等号成立。
\end{problem}
\newpage

% 例 8
\begin{problem}{麦克劳林不等式 (Maclaurin's Inequality)}{}
设 $a_{1},a_{2},\dots,a_{n}$ 是正实数,对 $1\le k\le n$,记
\begin{equation}
    S_{k}=\frac{\sum_{1\le i_{1}<i_{2}<\dots<i_{k}\le n}a_{i_{1}}a_{i_{2}}\dots a_{i_{k}}}{C_{n}^{k}}.
\end{equation}
则
\begin{equation}
    S_{1}\ge\sqrt{S_{2}}\ge\sqrt[3]{S_{3}}\ge\dots\ge\sqrt[n]{S_{n}}.
\end{equation}
当且仅当 $a_{1}=a_{2}=\dots=a_{n}$ 时等号成立。
\end{problem}
\newpage

% === 作业题部分 ===

% 作业题 1
\begin{problem}{作业题 1}{}
设 $n$ 是正整数,$a_{1},a_{2},\dots,a_{n},b_{1},b_{2},\dots,b_{n},A,B$ 是正实数,满足 $b_i \le a_i \le A,\ i = 1,2,\dots,n$,且 $\frac{b_{1}b_{2}\dots b_{n}}{a_{1}a_{2}\dots a_{n}}\le\frac{B}{A}$。
求证:
\begin{equation}
    \frac{(b_{1}+1)(b_{2}+1)\dots(b_{n}+1)}{(a_{1}+1)(a_{2}+1)\dots(a_{n}+1)}\le\frac{B+1}{A+1}.
\end{equation}
\end{problem}
\newpage

% 作业题 2
\begin{problem}{作业题 2}{}
设 $x_1, x_2, \dots, x_n$ 和 $y_1, y_2, \dots, y_n$ 均为不减的正数数列,满足 $\sum_{i=1}^{n}x_{i}=\sum_{i=1}^{n}y_{i}$。
求证:
\begin{equation}
    \sum_{\emptyset \ne S\subseteq\{1,2,\dots,n\}}\frac{\sum_{i\in S}x_{i}}{\sum_{i\in S}y_{i}}\le 2^{n}-1.
\end{equation}
\end{problem}
\newpage

% 作业题 3
\begin{problem}{作业题 3}{}
设 $0<a_{1}\le a_{2}\le\dots\le a_{n},\ b_{1}\ge b_{2}\ge\dots\ge b_{n}>0$,且对 $1 \le i \le n-1$,有 $\frac{a_{i+1}}{a_{i}}\le\frac{b_{i}}{b_{i+1}}$。
求证:
\begin{equation}
    \frac{A_{n}(a)}{G_{n}(a)}\le\left(\frac{A_{n}(b)}{G_{n}(b)}\right)^{n-1},
\end{equation}
其中 $A_n, G_n$ 分别表示算术平均值和几何平均值。
\end{problem}
\newpage

% 作业题 4
\begin{problem}{Suranyi 不等式}{}
设 $a_1, a_2, \dots, a_n$ 是正实数,则
\begin{equation}
    (n-1)\sum_{i=1}^{n}a_{i}^{n}+n\prod_{i=1}^{n}a_{i}\ge\left(\sum_{i=1}^{n}a_{i}\right)\left(\sum_{i=1}^{n}a_{i}^{n-1}\right).
\end{equation}
\end{problem}
\newpage