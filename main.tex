% main.tex
\documentclass[12pt, a4paper, openany, oneside]{ctexbook}

% ==========================================
% 1. 基础宏包配置
% ==========================================
\usepackage{amsmath, amssymb, amsthm, amsfonts}
\usepackage{mathrsfs}
\usepackage[margin=2.5cm]{geometry} % 更简洁的写法
\usepackage{fancyhdr}
\usepackage{hyperref}
\usepackage{xcolor} % 需要加载 xcolor 来定义自定义颜色

% --- 定义一套清新、低饱和度的色彩方案 (Material Design风格) ---
\definecolor{mainBlue}{HTML}{1976D2}    % 定理 - 海蓝
\definecolor{mainGreen}{HTML}{388E3C}   % 例题 - 森绿
\definecolor{mainAmber}{HTML}{FBC02D}   % 注意 - 琥珀
\definecolor{paleGray}{HTML}{F5F5F5}    % 通用极淡背景

\hypersetup{
    colorlinks=true,
    linkcolor=mainBlue,
    filecolor=magenta,      
    urlcolor=cyan,
    citecolor=mainGreen,
}

% ==========================================
% 2. 美化宏包 (tcolorbox) - 核心部分
% ==========================================
\usepackage[most]{tcolorbox}
\tcbuselibrary{theorems, skins, breakable}

% --- 定义一个通用的“清新风格”基础样式 ---
% 特点:无边框,左侧有粗色条,背景极淡,直角
\tcbset{
    cleanstyle/.style={
        enhanced,
        breakable,
        frame hidden,      % 隐藏四周所有边框
        leftrule=3pt,      % 只保留左侧3pt粗的线条
        boxrule=0pt,       % 其他边框宽度为0
        arc=0pt, outer arc=0pt, % 直角,不圆润
        titlerule=0pt,     % 标题和内容之间不要分割线
        left=8pt, right=8pt, top=8pt, bottom=8pt, % 内部边距
        toptitle=8pt, bottomtitle=4pt, % 标题边距
        fonttitle=\bfseries\large,
        coltitle=black,    % 标题文字颜色默认黑色
    }
}

% --- 定义【定理】环境 (海蓝色风格) ---
\newtcbtheorem[number within=chapter]{thm}{定理}{
    cleanstyle, % 调用上面定义的基础风格
    colframe=mainBlue,          % 左侧条颜色
    colback=mainBlue!4!white,   % 背景色 (极淡的蓝白)
    colbacktitle=mainBlue!4!white, % 标题背景色与内容一致
    coltitle=mainBlue,          % 标题文字颜色变蓝
    separator sign={\ $\cdot$\ },
    description font=\bfseries\color{mainBlue!80!black}
}{th}


% --- 定义【提示/注意】环境 (琥珀色风格) ---
% 注意环境不需要编号,所以用 newtcolorbox
\newtcolorbox{note}{
    cleanstyle,
    colframe=mainAmber,
    colback=mainAmber!8!white,
    colbacktitle=mainAmber!8!white,
    title={\textbf{注意}},
    coltitle=mainAmber!70!black,
    fontupper=\small % 内容字体稍微小一点点
}

% --- 定义【证明/解答】环境 ---
% 回归最简洁的 amsthm 风格,或者用一个极简的 tcolorbox
% 这里推荐用极简的 tcolorbox,带有缩进和灰色左条,区分度好
\newtcolorbox{proofbox}{
    enhanced, breakable, frame hidden, boxrule=0pt,
    leftrule=2pt, colframe=gray!30!white, % 浅灰色左条
    colback=white,
    left=8pt, right=0pt, top=0pt, bottom=0pt,
    arc=0pt, outer arc=0pt,
    parbox=false, % 允许段落正常换行
    before upper={\noindent\textbf{证明:}\ } % 开头加粗的“证明:”
}
\renewenvironment{proof}{\begin{proofbox}}{\qed\end{proofbox}}

\newtcolorbox{solutionbox}{
    enhanced, breakable, frame hidden, boxrule=0pt,
    leftrule=2pt, colframe=gray!30!white,
    colback=white,
    left=8pt, right=0pt, top=0pt, bottom=0pt,
    arc=0pt, outer arc=0pt,
    parbox=false,
    before upper={\noindent\textbf{解:}\ }
}
\newenvironment{solution}{\begin{solutionbox}}{\qed\end{solutionbox}}

% 在 main.tex 中添加:
\definecolor{mainPurple}{HTML}{7B1FA2} % 习题 - 紫色

% --- 定义颜色 ---
\definecolor{colPreview}{HTML}{009688}   % 预习 - 青色 (Teal)
\definecolor{colExample}{HTML}{2E7D32}   % 例题 - 森绿 (Green)
\definecolor{colProblem}{HTML}{673AB7}   % 练习 - 紫色 (Deep Purple)
\definecolor{colHomework}{HTML}{E65100}  % 习题 - 暖橙 (Orange)

% --- 1. 预习题 (Preview) ---
\newtcbtheorem[number within=chapter]{preview}{预习}{
    cleanstyle,
    colframe=colPreview,
    colback=colPreview!4!white,
    colbacktitle=colPreview!4!white,
    coltitle=colPreview,
    separator sign={\ $\cdot$\ },
    description font=\bfseries\color{colPreview!80!black}
}{pre}

% --- 2. 典型例题 (Example) ---
\newtcbtheorem[number within=chapter]{example}{例题}{
    cleanstyle,
    colframe=colExample,
    colback=colExample!5!white,
    colbacktitle=colExample!5!white,
    coltitle=colExample!90!black,
    separator sign={\ $\cdot$\ },
    description font=\bfseries\color{colExample!80!black}
}{ex}

% --- 3. 课堂练习/问题 (Problem) ---
\newtcbtheorem[number within=chapter]{problem}{练习}{
    cleanstyle,
    colframe=colProblem,
    colback=colProblem!4!white,
    colbacktitle=colProblem!4!white,
    coltitle=colProblem,
    separator sign={\ $\cdot$\ },
    description font=\bfseries\color{colProblem!80!black}
}{prob}

% --- 4. 课后习题 (Homework) ---
\newtcbtheorem[number within=chapter]{homework}{作业}{
    cleanstyle,
    colframe=colHomework,
    colback=colHomework!5!white,
    colbacktitle=colHomework!5!white,
    coltitle=colHomework,
    separator sign={\ $\cdot$\ },
    description font=\bfseries\color{colHomework!80!black}
}{hw}
% ==========================================
% 3. 页眉页脚设置 (保持简洁)
% ==========================================
\pagestyle{fancy}
\fancyhf{}
\fancyhead[L]{\small \leftmark} 
\fancyhead[R]{\small 代数不等式讲义} 
\fancyfoot[C]{\small \thepage} 
\renewcommand{\headrulewidth}{0.4pt} % 页眉线细一点

% ==========================================
% 4. 文档主体
% ==========================================
\begin{document}

% --- 封面 (稍微现代化一点) ---
\begin{titlepage}
    \centering
    \vspace*{4cm}
    {\Huge \bfseries \sffamily 重庆外国语学校-数学竞赛 \par} % 使用无衬线字体做大标题更现代
    \vspace{1.5cm}
    {\Huge \bfseries \sffamily 代数不等式 \par}
    \vspace{0.5cm}
    \rule{0.4\textwidth}{1pt} % 一条横线
    \vspace{5cm}
    
    % {\Large \sffamily \textbf{Joey} \par}
    \vspace{1cm}
    {\large \sffamily 2026.1.1 \par}
\end{titlepage}

\frontmatter
{
  \hypersetup{linkcolor=black} % 目录链接用黑色,不花哨
  \tableofcontents
}
\newpage

\mainmatter
% 引入章节

\part{基本不等式}
\chapter{均值不等式 (一)}

均值不等式是三大基本不等式之一,本讲介绍其在证明 $n$ 元不等式中的应用,需要体会均值不等式的适用情形、掌握“拆”的技巧、积累常见结构之间的关系。

\section{均值不等式的加强}

% 例 1
\begin{example}{}{}
设 $a_1, a_2, \dots, a_{n+1}$ 是正实数,记 $A_n, G_n$ 分别为 $a_1, \dots, a_n$ 的算术平均值和几何平均值。求证:
\begin{enumerate}
    \item $n(A_{n}-G_{n})\le(n+1)(A_{n+1}-G_{n+1})$;
    \item $\left(\frac{A_{n}}{G_{n}}\right)^{n}\le\left(\frac{A_{n+1}}{G_{n+1}}\right)^{n+1}$.
\end{enumerate}
\end{example}

\newpage 
% 例 2
\begin{example}{}{}
设 $a_{1},a_{2},\dots,a_{n}$ 是正实数,$b_{1},b_{2},\dots,b_{n}$ 是 $a_{1},a_{2},\dots,a_{n}$ 的一个排列。求证:
\begin{equation}
    A_{n}-G_{n}\ge\frac{1}{2n}\sum_{i=1}^{n}(\sqrt{b_{i+1}}-\sqrt{b_{i}})^{2},
\end{equation}
其中 $b_{n+1}=b_{1}$。
\end{example}

\newpage
\section{均值不等式的基本用法}

% 例 3
\begin{example}{}{}
设 $a_{1},a_{2},\dots,a_{n}$ 是正实数,记 $S=a_{1}+a_{2}+\dots+a_{n}$。求证:
\begin{equation}
    (1+a_{1})(1+a_{2})\dots(1+a_{n})\le1+\sum_{k=1}^{n}\left(1-\frac{k}{2n}\right)^{k-1}\frac{S^{k}}{k!}.
\end{equation}
\end{example}

\newpage 
% 例 4
\begin{example}{}{}
设整数 $n\ge2$,正实数 $a_{1},a_{2},\dots,a_{n}$ 满足 $a_{1}+a_{2}+\dots+a_{n}=1$。求证:
\begin{equation}
    \sum_{i=1}^{n}\frac{a_{i}}{a_{i+1}-a_{i+1}^{3}}\ge\frac{n^{3}}{n^{2}-1},
\end{equation}
其中 $a_{n+1}=a_{1}$。
\end{example}

\newpage 
% 例 5
\begin{example}{}{}
设 $a_{1},a_{2},\dots,a_{n},b_{1},b_{2},\dots,b_{n}$ 是非负实数,对 $1\le k\le n$,记 $c_{k}=\prod_{i=1}^{k}b_{i}^{\frac{1}{k}}$。求证:
\begin{equation}
    nc_{n}+\sum_{k=1}^{n}k(a_{k}-1)c_{k}\le\sum_{k=1}^{n}a_{k}^{k}b_{k}.
\end{equation}
\end{example}

\newpage 
% 例 6
\begin{example}{}{}
设 $a_{1},a_{2},\dots,a_{n}$ 是正实数,求证:
\begin{equation}
    \sum_{k=1}^{n}\left(\frac{a_{k}}{a_{k+1}}\right)^{n-1}\ge-n+2\left(\sum_{k=1}^{n}a_{k}\right)\prod_{k=1}^{n}{a_{k}}^{-\frac{1}{n}},
\end{equation}
其中 $a_{n+1}=a_{1}$。
\end{example}


\newpage 
% 例 7
\begin{example}{}{}
设整数 $n>2$,正实数 $a_{2},a_{3},\dots,a_{n}$ 满足 $a_{2}a_{3}\dots a_{n}=1$。求证:
\begin{equation}
    (1+a_{2})^{2}(1+a_{3})^{3}\dots(1+a_{n})^{n}>\frac{1}{4^{n-1}}n^{n}(n-1)^{n-1}.
\end{equation}
\end{example}


\newpage 
% 例 8
\begin{example}{}{}
给定正整数 $n$。设 $a_{1},a_{2},\dots,a_{n}$ 是正实数,求
\begin{equation}
    \frac{(1+a_{1})(1+a_{1}+a_{2})\dots(1+a_{1}+a_{2}+\dots+a_{n})}{\sqrt{a_{1}a_{2}\dots a_{n}}}
\end{equation}
的最小值。
\end{example}






\newpage 
\section{作业题}
% 作业题 1
\begin{homework}{}{}
设整数 $n\ge2$,$a_{1},a_{2},\dots,a_{n}$ 是正实数,求证:
\begin{equation}
    \frac{A_{n}}{G_{n}}\ge \max_{1\le i<j\le n}\left[\frac{1}{2}+\frac{1}{4}\left(\frac{a_{i}}{a_{j}}+\frac{a_{j}}{a_{i}}\right)\right]^{\frac{1}{n}}.
\end{equation}
\end{homework}

\newpage 
% 作业题 2
\begin{homework}{2016年高联P1}{}
给定整数 $n\ge2$。设实数 $a_{1},a_{2},\dots,a_{n}$ 满足 $9a_{i}>11a_{i+1}^{2}\ (1\le i\le n-1)$。求
\begin{equation}
    (a_{1}-a_{2}^{2})(a_{2}-a_{3}^{2})\dots(a_{n}-a_{1}^{2})
\end{equation}
的最大值。
\end{homework}

\newpage 
% 作业题 3
\begin{homework}{}{}
给定整数 $n\ge2$。求最大的实数 $\lambda$,使得对任意实数 $x_{1},x_{2},\dots,x_{n}\in(0,1]$,都有
\begin{equation}
    \frac{1}{\sum_{i=1}^{n}x_{i}}\ge\frac{1}{n}+\lambda\prod_{i=1}^{n}(1-x_{i}).
\end{equation}
\end{homework}

\newpage 
% 作业题 4
\begin{homework}{}{}
设 $n$ 是正整数,正实数 $a_{1},a_{2},\dots,a_{n}$ 满足 $a_{1}^{2}+2a_{2}^{3}+\dots+na_{n}^{n+1}\le1$。求证:
\begin{equation}
    2a_{1}+3a_{2}^{2}+\dots+(n+1)a_{n}^{n}<3.
\end{equation}
\end{homework}





\chapter{均值不等式 (二)}
\section{基础知识}
\begin{thm}{对称交叉项}{}
设整数 $n\ge2$,$a_{1},a_{2},\dots,a_{n}$ 是实数,则
\begin{equation}
    \sum_{1\le i<j\le n}a_{i}a_{j}\le\frac{n-1}{2n}\left(\sum_{i=1}^{n}a_{i}\right)^{2}.
\end{equation}
\end{thm}


\newpage 
\begin{thm}{轮换交叉项}{}
(四分之一引理)设整数 $n\ge 4$,$a_{1},a_{2},\dots,a_{n}$ 是非负实数,则
\begin{equation}
    \sum_{i=1}^{n}a_{i}a_{i+1}\le\frac{1}{4}\left(\sum_{i=1}^{n}a_{i}\right)^{2},
\end{equation}
其中 $a_{n+1}=a_{1}$。
\end{thm}


\newpage 
\section{典型例题}
% 例 1
\begin{example}{}{}
给定整数 $n\ge2$。设 $a_{1},a_{2},\dots,a_{n}$ 是实数,求
\begin{equation}
    \sum_{i=1}^{n}a_{i}^{2}+\sum_{1\le i<j\le n}a_{i}a_{j}+\sum_{i=1}^{n}a_{i}
\end{equation}
的最小值。
\end{example}


\newpage
% 例 2
\begin{example}{}{}
给定整数 $n\ge2$。设 $a_{1},a_{2},\dots,a_{n}$ 是实数,求
\begin{equation}
    \sum_{i=1}^{n}a_{i}^{2}+\sum_{i=1}^{n-1}a_{i}a_{i+1}+\sum_{i=1}^{n}a_{i}
\end{equation}
的最小值。
\end{example}


\newpage
% 例 3
\begin{example}{}{}
给定整数 $n\ge3$。设 $a_{1},a_{2},\dots,a_{2n},b_{1},b_{2},\dots,b_{2n}$ 是 $4n$ 个非负实数,满足
\begin{equation}
    a_{1}+a_{2}+\dots+a_{2n}=b_{1}+b_{2}+\dots+b_{2n}>0
\end{equation}
且对任意 $i=1,2,\dots,2n$,有 $a_{i}a_{i+2}\ge b_{i}+b_{i+1}$(这里 $a_{2n+1}=a_{1},a_{2n+2}=a_{2},b_{2n+1}=b_{1}$)。
求 $a_{1}+a_{2}+\dots+a_{2n}$ 的最小值。
\end{example}


\newpage
% 例 4
\begin{example}{}{}
给定整数 $n\ge3$。设实数 $a_{1},a_{2},\dots,a_{n}\ge-1$ 且满足 $a_{1}+a_{2}+\dots+a_{n}=0$,求
\begin{equation}
    a_{1}a_{2}+a_{2}a_{3}+\dots+a_{n-1}a_{n}
\end{equation}
的最大值。
\end{example}


\newpage
% 例 5
\begin{example}{}{}
给定整数 $n\ge4$。求最大的实数 $\lambda$,使得对任意非负实数 $a_{1},a_{2},\dots,a_{n}$,均有
\begin{equation}
    \sum_{i=1}^{n}a_{i}a_{i+1}+\lambda mM\le\frac{1}{4}\left(\sum_{i=1}^{n}a_{i}\right)^{2},
\end{equation}
其中 $a_{n+1}=a_{1}$,$m=\min\{a_{1},a_{2},\dots,a_{n}\}$,$M=\max\{a_{1},a_{2},\dots,a_{n}\}$。
\end{example}


\newpage
% 例 6
\begin{example}{}{}
设正实数 $a_{1},a_{2},\dots,a_{100}$ 满足 $a_{i}+a_{i+1}+a_{i+2}\le1,1\le i\le100$,其中脚标按模 100 理解。求 $\sum_{i=1}^{100}a_{i}a_{i+1}$ 的最大值。
\end{example}


\newpage
% 例 7
\begin{example}{}{}
给定整数 $n\ge3,\ \lambda\in[\frac{1}{2},2]$。设非负实数 $a_{1},a_{2},\dots,a_{n},b_{1},b_{2},\dots,b_{n}$ 满足 $\sum a_i = \sum b_i = 1$。
对 $1\le i\le n$,记 $c_{i}=(\lambda a_{i}+b_{i+1})(\lambda a_{i+1}+b_{i})$,其中 $a_{n+1}=a_{1},b_{n+1}=b_{1}$。求 $c_{1}+c_{2}+\dots+c_{n}$ 的最大值。
\end{example}


\newpage
\section{作业题}

% 作业题 1
\begin{homework}{}{}
设 $a_{1},a_{2},\dots,a_{100}$ 是非负实数,满足:
\begin{enumerate}
    \item $a_{1}+a_{2}+\dots+a_{100}=2$;
    \item $a_{1}a_{2}+a_{2}a_{3}+\dots+a_{100}a_{1}=1$.
\end{enumerate}
求 $a_{1}^{2}+a_{2}^{2}+\dots+a_{100}^{2}$ 的最大值和最小值。
\end{homework}


\newpage
% 作业题 2
\begin{homework}{}{}
给定整数 $n\ge4$。设非负实数 $a_{1},a_{2},\dots,a_{n}$ 满足 $a_{1}+a_{2}+\dots+a_{n}=2$,求
\begin{equation}
    \frac{a_{1}}{a_{2}^{2}+1}+\frac{a_{2}}{a_{3}^{2}+1}+\dots+\frac{a_{n}}{a_{1}^{2}+1}
\end{equation}
的最小值。
\end{homework}


\newpage
% 作业题 3
\begin{homework}{}{}
给定整数 $n\ge4$。设非负实数 $a_{1},a_{2},\dots,a_{n}$ 满足 $a_{1}+a_{2}+\dots+a_{n}=1$,求
\begin{equation}
    a_{1}a_{2}a_{3}+a_{2}a_{3}a_{4}+\dots+a_{n}a_{1}a_{2}
\end{equation}
的最大值。
\end{homework}


\newpage
% 作业题 4
\begin{homework}{}{}
给定整数 $n\ge2$。设集合 $T=\{(i,j) \mid 1\le i < j \le n,\ i \mid j\}$。
对任意满足 $x_{1}+x_{2}+\dots+x_{n}=1$ 的非负实数 $x_{1},x_{2},\dots,x_{n}$,求
\begin{equation}
    \sum_{(i,j)\in T} x_i x_j
\end{equation}
的最大值。
\end{homework}


\chapter{柯西不等式 (一)}
柯西不等式 (Cauchy-Schwarz Inequality) 是现代数学各个分支中应用最为广泛的不等式之一。本讲将系统介绍柯西不等式的多种形式、证明技巧(如换元、待定系数、裂项)以及相关的推广(如拉格朗日恒等式、Hölder 不等式)。

本节介绍柯西不等式的几种形式,以及在证明分式不等式中的应用。

\begin{thm}{柯西不等式}{}
\end{thm}

\vspace{5cm}
\begin{thm}{积分形式}{}
设 $f,g$ 是区间 $[a,b]$ 上的可积函数,求证:
\begin{equation}
    \left(\int_{a}^{b}f(x)^{2}dx\right)\left(\int_{a}^{b}g(x)^{2}dx\right)\ge\left(\int_{a}^{b}f(x)g(x)dx\right)^{2}.
\end{equation}
\end{thm}




\newpage 
% 例 2
\begin{thm}{Wagner 不等式}{}
设 $a_{1},a_{2},\dots,a_{n},b_{1},b_{2},\dots,b_{n}$ 是实数,$x\in[0,1]$。求证:
\begin{equation}
    \left(\sum_{k=1}^{n}a_{k}^{2}+2x\sum_{1\le i<j\le n}a_{i}a_{j}\right)\left(\sum_{k=1}^{n}b_{k}^{2}+2x\sum_{1\le i<j\le n}b_{i}b_{j}\right)\ge\left(\sum_{k=1}^{n}a_{k}b_{k}+x\sum_{i\ne j}a_{i}b_{j}\right)^{2}.
\end{equation}
\end{thm}

\newpage 
% 例 3
\begin{thm}{Aczel 不等式}{}
设整数 $n\ge2$,$a_{1},a_{2},\dots,a_{n},b_{1},b_{2},\dots,b_{n}$ 是实数,满足 $a_{1}^{2}>\sum_{i=2}^{n}a_{i}^{2}$。求证:
\begin{equation}
    \left(a_{1}^{2}-\sum_{i=2}^{n}a_{i}^{2}\right)\left(b_{1}^{2}-\sum_{i=2}^{n}b_{i}^{2}\right)\le\left(a_{1}b_{1}-\sum_{i=2}^{n}a_{i}b_{i}\right)^{2}.
\end{equation}
\end{thm}

\newpage 
\section{分式型柯西不等式}

% 例 4
\begin{example}{}{}
设整数 $n\ge2$,正实数 $a_{1},a_{2},\dots,a_{n}$ 满足 $\sum_{i=1}^{n}a_{i}=\sum_{i=1}^{n}a_{i}^{3}$。求证:
\begin{equation}
    \sum_{i=1}^{n}\frac{1}{a_{i}^{2}-a_{i+1}+n}\ge1,
\end{equation}
其中 $a_{n+1}=a_{1}$。
\end{example}
\newpage

% 例 5
\begin{example}{}{}
设 $a_{1},a_{2},\dots,a_{n}$ 是给定的正实数,求证:存在和为 1 的正实数 $x_{1},x_{2},\dots,x_{n}$,使得对任意和为 1 的正实数 $y_{1},y_{2},\dots,y_{n}$,都有
\begin{equation}
    \sum_{i=1}^{n}\frac{a_{i}x_{i}}{x_{i}+y_{i}}\ge\frac{1}{2}\sum_{i=1}^{n}a_{i}.
\end{equation}
\end{example}
\newpage

% 例 6
\begin{example}{}{}
设整数 $m<n$,$a_{1},a_{2},\dots,a_{n}$ 是正实数。对集合 $\{1,2,\dots,n\}$ 的子集 $A$,记 $S_{A}=\sum_{i\in A}a_{i}$。求证:
\begin{equation}
    \sum_{|A|=m}\frac{S_{A}}{S_{A^{c}}}\ge\frac{m}{n-m}C_{n}^{m}.
\end{equation}
\end{example}
\newpage

% 例 7
\begin{example}{}{}
设正实数 $a_{1},a_{2},\dots,a_{n}$ 满足 $\sum_{i=1}^{n}\frac{1}{1+a_{i}}=\frac{n}{2}$。求证:
\begin{equation}
    \sum_{1\le i,j\le n}\frac{1}{a_{i}+a_{j}}\ge\frac{n^{2}}{2}.
\end{equation}
\end{example}
\newpage

% 例 8
\begin{example}{}{}
设正实数 $a_{1},a_{2},\dots,a_{n}$ 满足 $\sum_{i=1}^{n}a_{i}=\frac{2}{n-1}\sum_{1\le i<j\le n}a_{i}a_{j}$。
对 $1\le i\le n$,记 $x_{i}=\sum_{j=1}^{n}a_{j}-a_{i}$。求证:
\begin{equation}
    \sum_{i=1}^{n}\frac{1}{1+x_{i}}\le1.
\end{equation}
\end{example}
\newpage

% 例 9
\begin{example}{2006 CTST}{}
设整数 $n\ge2$,正实数 $a_{1},a_{2},\dots,a_{n}$ 满足 $\sum_{i=1}^{n}a_{i}=1$。求证:
\begin{equation}
    \left(\sum_{i=1}^{n}\sqrt{a_{i}}\right)\left(\sum_{i=1}^{n}\frac{1}{\sqrt{1+a_{i}}}\right)\le\frac{n^{2}}{\sqrt{n+1}}.
\end{equation}
\end{example}




\newpage
\section{作业题}

% 作业 1
\begin{homework}{}{}
设 $a_{1},a_{2},\dots,a_{n},b_{1},b_{2},\dots,b_{n}$ 是实数,求证:
\begin{equation}
    \left(\sum_{i=1}^{n}a_{i}b_{i}\right)^{2}\le\left(\sum_{i=1}^{n}\max\{a_{i}^{2},b_{i}^{2}\}\right)\left(\sum_{i=1}^{n}\min\{a_{i}^{2},b_{i}^{2}\}\right)\le\left(\sum_{i=1}^{n}a_{i}^{2}\right)\left(\sum_{i=1}^{n}b_{i}^{2}\right).
\end{equation}
\end{homework}
\newpage

% 作业 2
\begin{homework}{}{}
给定整数 $n\ge3$。求最小的实数 $\lambda$,使得对任意正实数 $a_{1},a_{2},\dots,a_{n}$,都有
\begin{equation}
    \sum_{i=1}^{n-1}\frac{a_{i}}{S-a_{i}}+\frac{\lambda a_{n}}{S-a_{n}}\ge\frac{n-1}{n-2},
\end{equation}
其中 $S = a_1+a_2+\dots+a_n$。
\end{homework}
\newpage

% 作业 3
\begin{homework}{}{}
设整数 $n\ge2$,正实数 $a_{1}\ge a_{2}\ge\dots\ge a_{n}$。求证:
\begin{equation}
    \frac{a_{1}}{a_{1}+a_{2}}+\frac{a_{2}}{a_{2}+a_{3}}+\dots+\frac{a_{n}}{a_{n}+a_{1}}\ge\frac{n}{2}.
\end{equation}
\end{homework}
\newpage

% 作业 4
\begin{homework}{}{}
设正实数 $a_{1},a_{2},\dots,a_{n}$ 满足 $\sum_{i=1}^{n}a_{i}^{2}=n$。求证:
\begin{equation}
    \left(\sum_{i=1}^{n}a_{i}\right)^{2}\left(\sum_{i=1}^{n}\frac{1}{a_{i}^{2}+1}\right)\le\frac{n^{3}}{2}.
\end{equation}
\end{homework}


% % 第 20 题
% \begin{problem}{}{}
% 已知正实数 $x_1,x_2,\dots,x_n$ 满足 $\sum_{i=1}^n x_i=1$,证明:
% \begin{equation}
%     \Bigl(\sum_{i=1}^n\sqrt{x_i}\Bigr)^2\cdot\sum_{i=1}^n\frac{1}{1+x_i} \le\frac{n^3}{n+1}
% \end{equation}
% \end{problem}

\chapter{柯西不等式 (二)}

本节介绍柯西不等式与换元、待定系数、裂项等方法综合运用的问题。

\section{换元法}

% 例 1
\begin{example}{}{}
给定正整数 $n$。设实数 $a_{1},a_{2},\dots,a_{2n}$ 满足 $\sum_{i=1}^{2n-1}(a_{i+1}-a_{i})^{2}=1$,求
\begin{equation}
    (a_{n+1}+a_{n+2}+\dots+a_{2n})-(a_{1}+a_{2}+\dots+a_{n})
\end{equation}
的最大值。
\end{example}

\newpage 
% 例 2
\begin{example}{}{}
设实数 $a_{1},a_{2},\dots,a_{n}$ 满足 $a_{1}+a_{2}+\dots+a_{n}=0$,求证:
\begin{equation}
    \max_{1\le k\le n}a_{k}^{2}\le\frac{n}{3}\sum_{i=1}^{n-1}(a_{i+1}-a_{i})^{2}.
\end{equation}
\end{example}

\newpage 
% 例 3
\begin{example}{}{}
设 $a_{1},a_{2},\dots,a_{n}$ 是正实数,求证:
\begin{equation}
    \sum_{k=1}^{n}\sum_{j=1}^{k}\sum_{i=1}^{j}a_{i}\le 2\sum_{j=1}^{n}\frac{1}{a_{j}}\left(\sum_{i=1}^{j}a_{i}\right)^{2}.
\end{equation}
\end{example}

\newpage 
% 例 4
\begin{example}{}{}
给定整数 $n\ge2$。设非负实数 $a_{1},a_{2},\dots,a_{n}$ 满足 $\sum_{i=1}^{n}a_{i}^{2}+2\sum_{1\le i<j\le n}\sqrt{\frac{i}{j}}a_{i}a_{j}=1$,求 $\sum_{i=1}^{n}a_{i}$ 的最大值和最小值。
\end{example}

\newpage 
\section{待定系数法}

% 例 5
\begin{example}{Ostrowski 不等式}{}
设实数 $a_{1},a_{2},\dots,a_{n},b_{1},b_{2},\dots,b_{n},x_{1},x_{2},\dots,x_{n}$ 满足 $\sum_{i=1}^{n}a_{i}x_{i}=0,\ \sum_{i=1}^{n}b_{i}x_{i}=1$。求证:
\begin{equation}
    \sum_{i=1}^{n}x_{i}^{2}\ge\frac{\sum_{i=1}^{n}a_{i}^{2}}{(\sum_{i=1}^{n}a_{i}^{2})(\sum_{i=1}^{n}b_{i}^{2})-(\sum_{i=1}^{n}a_{i}b_{i})^{2}}.
\end{equation}
\end{example}

\newpage 
% 例 6
\begin{example}{}{}
给定正整数 $n$。设实数 $a_{1},a_{2},\dots,a_{n}$ 满足 $\sum_{i=1}^{n}ia_{i}=1$。求
\begin{equation}
    \sum_{i=1}^{n}a_{i}^{2}+\sum_{1\le i<j\le n}a_{i}a_{j}
\end{equation}
的最小值。
\end{example}

\newpage 
\section{裂项法}

% 例 7
\begin{example}{}{}
设 $a_{1},a_{2},\dots,a_{n}$ 是实数,求证:
\begin{equation}
    \frac{a_{1}}{1+a_{1}^{2}}+\frac{a_{2}}{1+a_{1}^{2}+a_{2}^{2}}+\dots+\frac{a_{n}}{1+a_{1}^{2}+\dots+a_{n}^{2}}<\sqrt{n}.
\end{equation}
\end{example}

\newpage 
% 例 8
\begin{example}{2004年CMO P5}{}
设整数 $n\ge2$,正整数 $a_{1}<a_{2}<\dots<a_{n}$ 满足 $\sum_{i=1}^{n}\frac{1}{a_{i}}\le1$。求证:对任意实数 $x$,有
\begin{equation}
    \left(\sum_{i=1}^{n}\frac{1}{a_{i}^{2}+x^{2}}\right)^{2}\le\frac{1}{2}\cdot\frac{1}{a_{1}(a_{1}-1)+x^{2}}.
\end{equation}
\end{example}



\newpage 
\section{作业题}

% 作业 1
\begin{homework}{}{}
设实数 $1=a_{1}\ge a_{2}\ge\dots\ge a_{n}\ge a_{n+1}=0$,求证:
\begin{equation}
    \sqrt{\sum_{i=1}^{n}a_{i}}\ge\sum_{i=1}^{n}\sqrt{i}(a_{i}-a_{i+1}).
\end{equation}
\end{homework}

\newpage 
% 作业 2
\begin{homework}{}{}
设整数 $n\ge2$,$a_{1},a_{2},\dots,a_{2n-1}$ 是实数,求证:
\begin{equation}
    (a_{1}+a_{3}+\dots+a_{2n-1})^{2}\le\sum_{1\le i\le j\le2n-1}(a_{i}+\dots+a_{j})^{2}.
\end{equation}
\end{homework}

\newpage 
% 作业 3
\begin{homework}{}{}
给定整数 $n\ge2$。求最小的实数 $\lambda$,使得对任意满足 $\sum_{i=1}^{n}ia_{i}=0$ 的实数 $a_{1},a_{2},\dots,a_{n}$,都有
\begin{equation}
    \left(\sum_{i=1}^{n}a_{i}\right)^{2}\le\lambda\sum_{i=1}^{n}a_{i}^{2}.
\end{equation}
\end{homework}

\newpage 
% 作业 4
\begin{homework}{}{}
设 $a_{1},a_{2},\dots,a_{n}$ 是正实数,求证:
\begin{equation}
    \frac{1}{1+a_{1}}+\frac{1}{1+a_{1}+a_{2}}+\dots+\frac{1}{1+a_{1}+\dots+a_{n}}<\sqrt{\frac{1}{a_{1}}+\frac{1}{a_{2}}+\dots+\frac{1}{a_{n}}}.
\end{equation}
\end{homework}


\chapter{柯西不等式 (三)}

本节介绍柯西不等式的两种推广和加强:拉格朗日恒等式与 Hölder 不等式,需要注意“平移不变”技巧的运用。

\section{拉格朗日恒等式}
\begin{thm}{拉格朗日恒等式}{}
对于任意实数 $a_{1},a_{2},\dots,a_{n}$ 和 $b_{1},b_{2},\dots,b_{n}$,有
\begin{equation}
    \left(\sum_{i=1}^{n}a_{i}^{2}\right)\left(\sum_{i=1}^{n}b_{i}^{2}\right)-\left(\sum_{i=1}^{n}a_{i}b_{i}\right)^{2}= \sum_{1\le i<j\le n}(a_{i}b_{j}-a_{j}b_{i})^{2}.
\end{equation}
\end{thm}


\newpage
% 例 1
\begin{example}{2010 中欧数学奥林匹克}{}
给定整数 $n\ge2$。求最大的实数 $\lambda$,使得对任意实数 $a_{1},a_{2},\dots,a_{n}$,有
\begin{equation}
    \frac{a_{1}^{2}+a_{2}^{2}+\dots+a_{n}^{2}}{n}\ge\left(\frac{a_{1}+a_{2}+\dots+a_{n}}{n}\right)^{2}+\lambda(a_{1}-a_{n})^{2}.
\end{equation}
\end{example}

\newpage 
% 例 2
\begin{example}{1999年CTST}{}
设 $a_{0},a_{1},\dots,a_{2n}$ 是实数,求证:
\begin{equation}
    \sum_{i=0}^{2n}a_{i}^{2}\ge\frac{1}{2n+1}\left(\sum_{i=0}^{2n}a_{i}\right)^{2}+\frac{3}{n(n+1)(2n+1)}\left(\sum_{i=0}^{2n}(i-n)a_{i}\right)^{2}.
\end{equation}
\end{example}

\newpage 
% 例 3
\begin{example}{}{}
设整数 $n\ge4$,正实数 $a_{1},a_{2},\dots,a_{n}$ 和 $t$ 满足 $\sum_{i=1}^{n}a_{i}=3t,\ \sum_{i=1}^{n}a_{i}^{2}=3t^{2},\ \sum_{i=1}^{n}a_{i}^{3}>3t^{3}+t$。求证:存在 $1\le i<j\le n$ 使得 $|a_{i}-a_{j}|>1$。
\end{example}

\newpage 
% 例 4
\begin{example}{}{}
设整数 $n\ge3$。正实数 $a_{1},a_{2},\dots,a_{n}$ 满足 $(\sum_{i=1}^{n}a_{i})(\sum_{i=1}^{n}\frac{1}{a_{i}})=n^{2}+1$。求证:
\begin{equation}
    \left(\sum_{i=1}^{n}a_{i}^{2}\right)\left(\sum_{i=1}^{n}\frac{1}{a_{i}^{2}}\right)\ge n^{2}+4+\frac{2}{n(n-1)}.
\end{equation}
\end{example}

\newpage 
% 例 5
\begin{example}{}{}
设 $a_{1},a_{2},\dots,a_{n}$ 是正实数,求证:
\begin{equation}
    \left(\sum_{i=1}^{n}a_{i}^{2n}\right)\left(\sum_{i=1}^{n}\frac{1}{a_{i}^{2n}}\right)-n^{2}\sum_{1\le i<j\le n}\left(\frac{a_{i}}{a_{j}}-\frac{a_{j}}{a_{i}}\right)^{2}\ge n^{2}.
\end{equation}
\end{example}

\newpage 
% 例 6
\begin{example}{}{}
设 $a_{1},\dots, a_{2n}, b_{1}, \dots, b_{2n}$ 是实数,求证:
\begin{equation}
    \left(\sum_{i=1}^{2n}a_{i}^{2}\right)\left(\sum_{i=1}^{2n}b_{i}^{2}\right)-\left(\sum_{i=1}^{2n}a_{i}b_{i}\right)^{2}\ge\left[\sum_{i=1}^{n}(a_{i}b_{n+i}-a_{n+i}b_{i})\right]^{2}.
\end{equation}
\end{example}

\newpage 
\section{Hölder 不等式}
\begin{thm}{Hölder 不等式}{sym_holder}
设 $m, n \ge 2$ 为整数。给定 $m$ 组非负实数:
\[
(a_{1,1}, \dots, a_{n,1}), \quad (a_{1,2}, \dots, a_{n,2}), \quad \dots, \quad (a_{1,m}, \dots, a_{n,m}).
\]
则有:
\begin{equation}
    \sum_{i=1}^n \left( \prod_{j=1}^m a_{i,j} \right) \le \prod_{j=1}^m \left( \sum_{i=1}^n a_{i,j}^m \right)^{\frac{1}{m}}
\end{equation}
即:
\begin{equation}
    \sum_{i=1}^n a_{i,1} a_{i,2} \cdots a_{i,m} \le \left( \sum_{i=1}^n a_{i,1}^m \right)^{\frac{1}{m}} \left( \sum_{i=1}^n a_{i,2}^m \right)^{\frac{1}{m}} \cdots \left( \sum_{i=1}^n a_{i,m}^m \right)^{\frac{1}{m}}
\end{equation}
当且仅当这 $m$ 组向量两两共线(即成比例)时,等号成立。
\end{thm}
\vspace{2cm}






\newpage 
% 例 7
\begin{example}{}{}
设 $a_{1},a_{2},\dots,a_{n}$ 是正实数,求证:
\begin{equation}
    \prod_{i=1}^{n}(a_{i}^{3}+1)\ge\prod_{i=1}^{n}(a_{i}^{2}a_{i+1}+1),
\end{equation}
其中 $a_{n+1}=a_{1}$。
\end{example}

\newpage 
% 例 8
\begin{example}{}{}
设正实数 $a_{1},a_{2},\dots,a_{n}$ 满足 $a_{1}+a_{2}+\dots+a_{n}=1$。求证:
\begin{equation}
    (a_{1}a_{2}+a_{2}a_{3}+\dots+a_{n}a_{1})\left(\frac{a_{1}}{a_{2}^{2}+a_{2}}+\frac{a_{2}}{a_{3}^{2}+a_{3}}+\dots+\frac{a_{n}}{a_{1}^{2}+a_{1}}\right)\ge\frac{n}{n+1}.
\end{equation}
\end{example}

\newpage 
\section{作业题}

% 作业 1
\begin{homework}{}{}
设 $a_{1},a_{2},\dots,a_{n}$ 是实数,求证:
\begin{equation}
    \left(\sum_{1\le i<j\le n}|a_{i}-a_{j}|\right)^{2}\le\frac{n^{2}-1}{3}\sum_{1\le i<j\le n}(a_{i}-a_{j})^{2}.
\end{equation}
\end{homework}

\newpage 
% 作业 2
\begin{homework}{}{}
设正实数 $a_{1},a_{2},\dots,a_{n},b_{1},b_{2},\dots,b_{n}$ 满足 $a_{i}>b_{i}\ (i=1,2,\dots,n)$ 且 $\prod_{i=1}^{n}a_{i}b_{i}=1$。求证:
\begin{equation}
    \prod_{i=1}^{n}a_{i}-\prod_{i=1}^{n}b_{i}\ge n\sqrt[n]{\prod_{i=1}^{n}(a_{i}-b_{i})}.
\end{equation}
\end{homework}




% chapters/20_sorting_chebyshev.tex

\chapter{排序不等式与切比雪夫不等式}

排序不等式是除均值不等式和柯西不等式之外的第三种基本不等式,有别于前两种通过平方和得到大小关系,排序不等式是通过序来得到的。切比雪夫 (Chebyshev) 不等式可以认为是排序不等式的特殊情况,但因其结构整齐,特别在近年出现的频率很高。

本讲介绍排序不等式和切比雪夫不等式的证明及应用。需要注意的是,有的题目虽然条件中没有给出序,但如果变量地位相同,我们可以不妨设一个序。

\section{基础知识}

\begin{thm}{排序不等式 (Rearrangement Inequality)}{rearrangement}
设实数组 $a_1 \le a_2 \le \dots \le a_n$ 和 $b_1 \le b_2 \le \dots \le b_n$。
若 $c_1, c_2, \dots, c_n$ 是 $b_1, b_2, \dots, b_n$ 的任一排列,则有:
\begin{equation}
    \sum_{i=1}^n a_i b_{n+1-i} \le \sum_{i=1}^n a_i c_i \le \sum_{i=1}^n a_i b_i
\end{equation}
即:\textbf{反序和 $\le$ 乱序和 $\le$ 同序和}。
当且仅当 $a_1=\dots=a_n$ 或 $b_1=\dots=b_n$ 时等号成立(严格递增时,仅当排列对应相同时取等)。
\end{thm}
\vspace{1.5cm}

\begin{thm}{切比雪夫不等式 (Chebyshev's Inequality)}{chebyshev}
\begin{enumerate}
    \item 若 $a_1 \le a_2 \le \dots \le a_n$ 且 $b_1 \le b_2 \le \dots \le b_n$(同序),则
    \begin{equation}
        n \sum_{i=1}^n a_i b_i \ge \left(\sum_{i=1}^n a_i\right) \left(\sum_{i=1}^n b_i\right)
    \end{equation}
    \item 若 $a_1 \le a_2 \le \dots \le a_n$ 且 $b_1 \ge b_2 \ge \dots \ge b_n$(反序),则
    \begin{equation}
        n \sum_{i=1}^n a_i b_i \le \left(\sum_{i=1}^n a_i\right) \left(\sum_{i=1}^n b_i\right)
    \end{equation}
\end{enumerate}
\end{thm}
\newpage

\section{典型例题}

% 例 1
\begin{problem}{}{}
设实数 $x_{1},x_{2},\dots,x_{n},y_{1},y_{2},\dots,y_{n}$ 满足 $x_{1}\ge x_{2}\ge\dots\ge x_{n},\ y_{1}\ge y_{2}\ge\dots\ge y_{n}$。$z_{1},z_{2},\dots,z_{n}$ 是 $y_1, y_2, \dots, y_n$ 的一个排列,求证:
\begin{equation}
    \sum_{i=1}^{n}(x_{i}-y_{i})^{2}\le\sum_{i=1}^{n}(x_{i}-z_{i})^{2}.
\end{equation}
\end{problem}
\newpage

% 例 2
\begin{problem}{}{}
设 $\theta_{1},\theta_{2},\dots,\theta_{n}\in(0,\frac{\pi}{2})$,求证:
\begin{equation}
    \sum_{i=1}^{n}\theta_{i}\ge\sum_{i=1}^{n}\theta_{i}\cdot\frac{\sin \theta_{i+1}}{\sin \theta_{i}},
\end{equation}
其中 $\theta_{n+1}=\theta_{1}$。
\end{problem}
\newpage

% 例 3
\begin{problem}{}{}
设 $a_{1},a_{2},\dots,a_{n}$ 是两两不同的正整数,求证:
\begin{equation}
    \sum_{k=1}^{n}\frac{a_{k}}{k^{2}}\ge\sum_{k=1}^{n}\frac{1}{k}.
\end{equation}
\end{problem}
\newpage

% 例 4
\begin{problem}{}{}
设正实数 $a_{1},a_{2},\dots,a_{n},b_{1},b_{2},\dots,b_{n}$ 满足 $a_{1}\ge a_{2}\ge\dots\ge a_{n},\ b_{1}\le b_{2}\le\dots\le b_{n}$。求证:
\begin{equation}
    \sum_{i=1}^{n}\frac{a_{i}}{b_{i}}\ge\frac{a_{1}+a_{2}+\dots+a_{n}}{\sqrt[n]{b_{1}b_{2}\dots b_{n}}}.
\end{equation}
\end{problem}
\newpage

% 例 5
\begin{problem}{}{}
设 $a_{1},a_{2},\dots,a_{n}$ 是正实数,求证:
\begin{equation}
    \frac{1}{\sum_{i=1}^n \frac{1}{1+a_i}} - \frac{1}{\sum_{i=1}^n \frac{1}{a_i}} \ge \frac{1}{n}.
\end{equation}
\end{problem}
\newpage

% 例 6
\begin{problem}{}{}
设整数 $n\ge2$,正实数 $a_{1},a_{2},\dots,a_{n}$ 满足 $\sum_{i=1}^{n}a_{i}=1$。求证:
\begin{equation}
    \left(\sum_{i=1}^{n}\frac{1}{1-a_{i}}\right)\left(\sum_{1\le i<j\le n}a_{i}a_{j}\right)\le\frac{n}{2}.
\end{equation}
\end{problem}
\newpage

% 例 7
\begin{problem}{}{}
给定整数 $n\ge3$。设非负实数 $a_{1}\le a_{2}\le\dots\le a_{n}$,且满足 $\sum_{i=1}^{n}a_{i}=1$。求 $a_{n}\sum_{i=1}^{n}(n+1-i)a_{i}$ 的最大值。
\end{problem}
\newpage

% 例 8
\begin{problem}{}{}
给定整数 $n\ge2$。设非负实数 $a_{1}\ge a_{2}\ge\dots\ge a_{n},\ b_{1}\le b_{2}\le\dots\le b_{n}$,满足
\begin{equation}
    \sum_{i=1}^n a_i a_{n+1-i} = \sum_{i=1}^n b_i b_{n+1-i} = 1.
\end{equation}
求 $\sum_{1\le i<j\le n}a_{i}b_{j}$ 的最小值。
\end{problem}
\newpage

\section{课后练习}

% 作业题 1
\begin{problem}{作业题 1}{}
设 $a_{1},a_{2},\dots,a_{n}$ 是 $1,2,\dots,n$ 的一个排列,求证:
\begin{equation}
    \sum_{k=2}^{n}\frac{a_{k-1}}{a_{k}}\ge\sum_{k=2}^{n}\frac{k-1}{k}.
\end{equation}
\end{problem}
\newpage

% 作业题 2
\begin{problem}{作业题 2}{}
设 $a_{1},a_{2},\dots,a_{n},b_{1},b_{2},\dots,b_{n}$ 是非负实数。对 $1\le k \le n$,定义 
\[
C_k = \max\{a_{1}b_{k},a_{2}b_{k},\dots,a_{k}b_{k},a_{k}b_{k-1},\dots,a_{k}b_{1}\}.
\]
设 $\sigma$ 是 $1 \sim n$ 的一个置换,求证:
\begin{equation}
    a_{1}b_{\sigma(1)}+a_{2}b_{\sigma(2)}+\dots+a_{n}b_{\sigma(n)}\le C_{1}+C_{2}+\dots+C_{n}.
\end{equation}
\end{problem}
\newpage

% 作业题 3
\begin{problem}{作业题 3}{}
给定整数 $n\ge2$。求最大的正实数 $\lambda$,使得对任意满足 $x_{1}\le x_{2}\le\dots\le x_{n},\ y_{1}\le y_{2}\le\dots\le y_{n}$ 且 $\sum_{i=1}^{n}x_{i}=\sum_{i=1}^{n}y_{i}=0$ 的实数 $x_{1},x_{2},\dots,x_{n},y_{1},y_{2},\dots,y_{n}$,都有
\begin{equation}
    \sum_{i=1}^{n}x_{i}y_{i}\ge\lambda \max_{1\le i\le n}x_{i}y_{i}.
\end{equation}
\end{problem}
\newpage

% 作业题 4
\begin{problem}{作业题 4}{}
设整数 $n\ge2$,正实数 $a_{1},a_{2},\dots,a_{n}$ 满足 $\sum_{i=1}^{n}a_{i}=1$。求证:
\begin{equation}
    (n-1)\sum_{i=1}^{n}\frac{1}{1-a_{i}}\ge(n+1)\sum_{i=1}^{n}\frac{1}{1+a_{i}}.
\end{equation}
\end{problem}
\newpage
\chapter{补充练习题}

\begin{problem}{}
(18 俄罗斯)给定正实数 $x_1,x_2,\dots,x_n$,其中整数 $n\ge 2$,证明:
\begin{equation}
\frac{1+x_1^2}{1+x_1x_2}+\frac{1+x_2^2}{1+x_2x_3}+\dots+\frac{1+x_n^2}{1+x_nx_1}\ge n.
\end{equation}
\end{problem}


\begin{problem}{}
给定正实数 $a_1,a_2,\dots,a_n$,其中整数 $n\ge 2$,证明:
\begin{equation}
\frac{a_1^2+1}{a_1a_2+1}+\frac{a_2^2+1}{a_2a_3+1}+\dots+\frac{a_n^2+1}{a_na_1+1}
\le\frac{a_1}{a_2}+\frac{a_2}{a_3}+\dots+\frac{a_n}{a_1}.
\end{equation}
\end{problem}


\begin{problem}{}
(16 新加坡)设 $a_1,a_2,\dots,a_n$ 为正实数,且 $a_{n+1}=a_1$,证明:
\begin{equation}
\frac{a_2}{a_1}+\frac{a_3}{a_2}+\dots+\frac{a_{n+1}}{a_n}
\ge\sqrt{\frac{1+a_2^2}{1+a_1^2}}+\sqrt{\frac{1+a_3^2}{1+a_2^2}}+\dots+\sqrt{\frac{1+a_{n+1}^2}{1+a_n^2}}.
\end{equation}
\end{problem}


\begin{problem}{}
设正实数 $a_1,a_2,\dots,a_n$ 满足 $a_1+a_2+\dots+a_n=1$,证明:
\begin{equation}
\frac{a_1}{a_2}+\frac{a_2}{a_3}+\dots+\frac{a_n}{a_1}
\ge\frac{1-a_2}{1-a_1}+\frac{1-a_3}{1-a_2}+\dots+\frac{1-a_1}{1-a_n}.
\end{equation}
\end{problem}


\begin{problem}{}
设整数 $n\ge 3$,证明:对于正实数 $x_1\le x_2\le\dots\le x_n$,有
\begin{equation}
\frac{x_1x_2}{x_3}+\frac{x_2x_3}{x_4}+\dots+\frac{x_{n-1}x_n}{x_1}+\frac{x_nx_1}{x_2}
\ge x_1+x_2+\dots+x_n.
\end{equation}
\end{problem}


\begin{problem}{}
已知正实数 $a_1,a_2,\dots,a_n$ 满足 $a_1+a_2+\dots+a_n=1$,记
\[
b_k=\frac{a_k}{a_k^2+a_ka_{k+1}+a_{k+1}^2},\quad k=1,2,\dots,n,
\]
其中 $a_{n+1}=a_1$,证明:
\begin{equation}
(a_1^4+a_2^4+\dots+a_n^4)(b_1^2+b_2^2+\dots+b_n^2)\ge\frac19.
\end{equation}
\end{problem}


\begin{problem}{}
设正整数 $n\ge 2$,且 $a_1,a_2,\dots,a_n,b_1,b_2,\dots,b_n$ 是正实数,证明:
\begin{equation}
\sum_{k=1}^n\frac{a_k}{\sum_{\substack{1\le j\le n\\ j\ne k}}a_jb_j}
\ge\frac{4}{b_1+b_2+\dots+b_n}.
\end{equation}
\end{problem}


\begin{problem}{}
已知 $a_1,a_2,\dots,a_n$ 为正实数,证明:
\begin{equation}
\frac{1}{1+a_1}+\frac{2}{1+a_1+a_2}+\dots+\frac{n}{1+a_1+a_2+\dots+a_n}
\le\frac{n}{2}\sqrt{\frac{1}{a_1}+\frac{1}{a_2}+\dots+\frac{1}{a_n}}.
\end{equation}
\end{problem}


\begin{problem}{}
已知实数 $a_1,a_2,\dots,a_n$ 满足 $0\le a_i\le 1\ (i=1,2,\dots,n)$,证明:
\begin{equation}
\sum_{i=1}^n\frac{a_i}{1+\sum_{\substack{j=1\\ j\ne i}}^n a_j}+\prod_{i=1}^n(1-a_i)\le 1.
\end{equation}
\end{problem}


\begin{problem}{}
设 $a_1,a_2,\dots,a_n$ 均大于 1,且 $x_0,x_1,\dots,x_n$ 满足 $x_0=1,\ x_k=\dfrac{1}{1+a_kx_{k-1}}\ (1\le k\le n)$,证明:
\begin{equation}
\sum_{k=1}^n x_k>\frac{n^2\bigl(1+\sum_{k=1}^n a_k\bigr)}{n^2+\bigl(1+\sum_{k=1}^n a_k\bigr)^2}.
\end{equation}
\end{problem}


\begin{problem}{}
(17 女奥)设 $a_i\ge 0,\ x_i\in\mathbb R,\ i=1,2,\dots,n$,证明:
\begin{equation}
\Bigl[\Bigl(1-\sum_{i=1}^n a_i\cos x_i\Bigr)^2+\Bigl(1-\sum_{i=1}^n a_i\sin x_i\Bigr)^2\Bigr]^2
\ge 4\Bigl(1-\sum_{i=1}^n a_i\Bigr)^3.
\end{equation}
\end{problem}


\begin{problem}{}
设 $n$ 为正整数,实数 $a_1,a_2,\dots,a_n$ 满足 $\sum_{i=1}^n a_i^2=1$,证明:
\begin{equation}
\sum_{1\le i\cdot j \le n}a_ia_j<2\sqrt{n}.
\end{equation}
\end{problem}



\begin{problem}{}
设正整数 $n\ge 2$,$a_1,a_2,\dots,a_n$ 是实数,证明:
\begin{equation}
\frac{3}{n^2-1}\Bigl[\sum_{k=1}^n(2k-n-1)a_k\Bigr]^2
+\Bigl(\sum_{k=1}^n a_k\Bigr)^2
\le n\sum_{k=1}^n a_k^2.
\end{equation}
\end{problem}


\begin{problem}{}
给定正整数 $n$,设正实数 $a_1,a_2,\dots,a_n$ 满足
\begin{equation}
\sum_{i=1}^n a_i=\frac{2}{n-1}\sum_{1\le i<j\le n}a_ia_j,
\end{equation}
记 $x_i=\sum_{j=1}^n a_j-a_i\ (1\le i\le n)$,证明:
\begin{equation}
\sum_{i=1}^n\frac{1}{1+x_i}\le 1.
\end{equation}
\end{problem}


\begin{problem}{}
(19 浙江预赛)设 $a_i,b_i>0\ (1\le i\le n+1)$,$b_{i+1}-b_i\ge\delta>0$($\delta$ 为常数),若 $\sum_{i=1}^n a_i=1$,证明:
\begin{equation}
\sum_{i=1}^n\frac{i\,\sqrt[i]{a_1a_2\dotsm a_ib_1b_2\dotsm b_i}}{b_ib_{i+1}}
<\frac{1}{\delta}.
\end{equation}
\end{problem}


\begin{problem}{}
设 $x_1,x_2,\dots,x_n$ 为非负实数,证明
\begin{equation}
\frac{x_1}{\left(1+x_1+x_2+\cdots+x_n\right)^2}+\frac{x_2}{\left(1+x_2+x_3+\cdots+x_n\right)^2}+\cdots+\frac{x_n}{\left(1+x_n\right)^2} \leq k_n^2
\end{equation}
其中数列 $\{k_n\}$ 满足 $k_1=\frac12,\ k_{n+1}=\dfrac{k_n^2+1}{2}$。
\end{problem}




\begin{problem}{}
已知 $a_1,a_2,\dots,a_n$ 是正实数,且 $n\ge 3,\ S=\sum_{i=1}^n a_i$,证明:
\begin{equation}
\sum_{i=1}^n\Bigl[\frac{a_i}{S-a_i}+\sqrt[n-1]{\Bigl(\frac{(n-1)a_i}{S-a_i}\Bigr)^{n-2}}\Bigr]
\ge\frac{n^2}{n-1}.
\end{equation}
\end{problem}


\begin{problem}{11 西班牙改}
已知 $a_1,a_2,\dots,a_n$ 是正实数,且 $n\ge 3$,设 $S=\sum_{i=1}^n a_i$,证明:
\begin{equation}
\sum_{i=1}^n\frac{a_i}{S-a_i}+\frac{2}{n-1}\sqrt{\frac{\sum_{1\le i<j\le n}2a_ia_j}{(n-1)\sum_{i=1}^n a_i^2}}
\ge\frac{n+2}{n-1}.
\end{equation}
\end{problem}


\begin{problem}{10 地中海}
已知 $n>2$,正实数 $a_1,a_2,\dots,a_n$ 满足 $a_1+a_2+\dots+a_n=1$,证明:
\begin{equation}
\frac{a_2a_3\dotsm a_n}{a_1+n-2}+\frac{a_1a_3\dotsm a_n}{a_2+n-2}+\dots+\frac{a_1a_2\dotsm a_{n-1}}{a_n+n-2}
\le\frac{1}{(n-1)^2},
\end{equation}
\end{problem}


\begin{problem}{}
已知 $a_1,a_2,\dots,a_n$ 是正实数,且 $n\ge 3$,设 $S=\sum_{i=1}^n a_i^2$,证明:
\begin{equation}
\Bigl(\sum_{i=1}^n a_i\Bigr)^2\sum_{i=1}^n\frac{1}{S+a_i^2}
\le\frac{n^3}{n+1}.
\end{equation}
\end{problem}


\begin{problem}{}
已知实数 $a_1,a_2,\dots,a_n$ 满足 $\sum_{i=1}^n a_i=0$,证明:
\begin{equation}
\sum_{i=1}^n\frac{(a_i+1)^2}{a_i^2+n-1}
\ge\frac{n}{n-1}.
\end{equation}
\end{problem}


\begin{problem}{}
已知实数 $x_1,x_2,\dots,x_n$ 满足 $\sum_{i=1}^n x_i=0$,证明:
\begin{equation}
\sum_{i=1}^n\frac{(n-2)x_i^2+2x_i}{(n-1)x_i^2+1}
\ge 0.
\end{equation}
\end{problem}


\begin{problem}{}
(86 苏联)设 $a_1,a_2,\dots,a_n$ 均为正实数,证明:
\begin{equation}
\frac{1}{a_1}+\frac{2}{a_1+a_2}+\dots+\frac{n}{a_1+a_2+\dots+a_n}
<4\Bigl(\frac{1}{a_1}+\frac{1}{a_2}+\dots+\frac{1}{a_n}\Bigr).
\end{equation}
\end{problem}


\begin{problem}{}
设 $x_1,x_2,\dots,x_n$ 为非负实数,证明:
\begin{equation}
\sum_{k=1}^n\Bigl(\frac{x_1+x_2+\dots+x_k}{k}\Bigr)^2
\le\sum_{k=1}^n(k+1)x_k^2.
\end{equation}
\end{problem}


\begin{problem}{}
(05 韩国)设 $x_1,x_2,\dots,x_n$ 为非负实数,证明:
\begin{equation}
\sum_{k=1}^n\Bigl(\frac{x_1+x_2+\dots+x_k}{k}\Bigr)^2
\le 4\sum_{k=1}^n x_k^2.
\end{equation}
\end{problem}


\begin{problem}{}
设 $0<x_n<x_{n-1}<\dots<x_1<x_0<1$,证明:
\begin{equation}
\sum_{i=1}^n\frac{x_i^2}{x_{i-1}-x_i}
>\frac12\sum_{i=1}^n i x_i-1.
\end{equation}
\end{problem}


\begin{problem}{}
给定实数 $p_1,p_2,\dots,p_n$,对于满足 $p_1x_1+p_2x_2+\dots+p_nx_n=1$ 的实数 $x_1,x_2,\dots,x_n$,记
\begin{equation}
P=\sum_{i=1}^n x_i^2+\Bigl(\sum_{i=1}^n x_i\Bigr)^2,
\end{equation}
求 $P$ 的最小值,并给出相应的取等条件。
\end{problem}


\begin{problem}{}
设 $x_1,x_2,\dots,x_n$ 为正实数,证明:
\begin{equation}
\sum_{i=1}^n x_i^2
\ge\frac{1}{n+1}\Bigl(\sum_{i=1}^n x_i\Bigr)^2
+\frac{12}{n(n+1)(n+2)(3n+1)}\Bigl(\sum_{i=1}^n i x_i\Bigr)^2.
\end{equation}
\end{problem}


\begin{problem}{}
给定整数 $n\ge 3$,求最大的实数 $\lambda$,使得只要正实数 $a_1,a_2,\dots,a_n$ 满足
\[
a_1^2+a_2^2+\dots+a_n^2<\lambda\,(a_1+a_2+\dots+a_n)^2,
\]
就有 $a_1,a_2,\dots,a_n$ 中任意三个数均可作为某个三角形的边长。
\end{problem}


\part{基本方法}
\chapter{恒等变形}
\section{基础知识}

\begin{thm}{}{basic_split}
\begin{gather}
    a_n - a_1 = \sum_{i=1}^{n-1} (a_{i+1} - a_i) \\
    \left(\sum_{i=1}^n a_i\right)^2 = \sum_{i=1}^n a_i^2 + 2 \sum_{1 \leq i < j \leq n} a_i a_j = \sum_{i=1}^n a_i^2 + \sum_{i \neq j} a_i a_j
\end{gather}
\end{thm}
% 留白供笔记
\vspace{2.5cm}


\begin{thm}{}{cyclic_diff}
当 $a_{n+1} = a_1$ 时:
\begin{equation}
    \sum_{i=1}^n a_i^2 - \sum_{i=1}^n a_i a_{i+1} = \frac{1}{2} \sum_{i=1}^n (a_i - a_{i+1})^2
\end{equation}
\end{thm}
% 留白供笔记
\vspace{2.5cm}

\begin{thm}{拉格朗日恒等式 (Lagrange)}{lagrange}
\begin{gather}
    \left(a_i a_j + b_i b_j\right) - \left(a_i b_j + a_j b_i\right) = \left(a_i - b_i\right)\left(a_j - b_j\right), \quad 1 \leq i, j \leq n \\[8pt]
    \sum_{i=1}^n a_i^2 \cdot \sum_{i=1}^n b_i^2 = \left(\sum_{i=1}^n a_i b_i\right)^2 + \sum_{1 \leq i < j \leq n} (a_i b_j - a_j b_i)^2
\end{gather}
\end{thm}
% 留白供笔记
\vspace{2.5cm}


\begin{thm}{}{variance}
\begin{gather}
    \sum_{1 \leq i < j \leq n} (a_i - a_j)^2 = n \sum_{i=1}^n a_i^2 - \left(\sum_{i=1}^n a_i\right)^2 \\
    = (n - 1) \sum_{i=1}^n a_i^2 - \sum_{1 \leq i < j \leq n} 2 a_i a_j \\[8pt]
    \sum_{i=1}^n \left(2 \sum_{k=1}^n b_k - n b_i\right)^2 = n^2 \sum_{i=1}^n b_i^2
\end{gather}
\end{thm}
% 留白供笔记
\vspace{2.5cm}





\begin{thm}{}{double_sum}
\begin{gather}
    \sum_{1 \leq i \leq j \leq n} a_i a_j = \sum_{j=1}^n \sum_{i=1}^j a_i a_j = \sum_{i=1}^n \sum_{j=i}^n a_i a_j \\
    = \sum_{i=1}^n a_i^2 + \sum_{1 \leq i < j \leq n} a_i a_j \\[10pt]
    \sum_{i=1}^n a_i \cdot \sum_{i=1}^n b_i = \sum_{i=1}^n \sum_{j=1}^n a_i b_j = \sum_{i=1}^n \sum_{j=1}^n a_j b_i = \frac{1}{2} \sum_{i=1}^n \sum_{j=1}^n (a_i b_j + a_j b_i)
\end{gather}
\end{thm}
% 留白供笔记
\vspace{2.5cm}


\begin{thm}{}{prefix_sum}
\begin{gather}
    \sum_{k=1}^n\left(\sum_{i=1}^k a_i\right)^2 = \sum_{i=1}^n(n + 1 - i)a_i^2 + 2\sum_{1 \leq j < k \leq n}(n + 1 - k)a_j a_k \\[6pt]
    (n + 1)\left(\sum_{k=1}^n a_k\right)^2 - \sum_{k=1}^n\left(\sum_{i=1}^k a_i\right)^2 = \sum_{i=1}^n i a_i^2 + 2\sum_{1 \leq j < k \leq n} k a_j a_k \\[6pt]
    \sum_{k=1}^{n-1}\left(\sum_{i=1}^k a_i\right)\left(\sum_{i=k+1}^n a_i\right) = \sum_{1 \leq i < j \leq n}(j - i)a_i a_j \\[6pt]
    \sum_{k=1}^n\left(\sum_{i=k}^n a_i\right)^2 = \sum_{i=1}^n i a_i^2 + 2\sum_{1 \leq i < k \leq n} i a_i a_k
\end{gather}
\end{thm}
% 留白供笔记
\vspace{2.5cm}


\begin{thm}{}{special_construct}
\begin{gather}
    \sum_{k=1}^n \frac{a_{k+1}}{a_k(a_k + a_{k+1})} = \sum_{k=1}^n \frac{a_k}{a_{k+1}(a_k + a_{k+1})} \\[10pt]
    \sum_{i=1}^n \frac{1}{1 + \sum_{k=0}^{n-2} \left(\prod_{j=i}^{i+k} a_j\right)} = 1 \quad (\text{其中} \prod_{i=1}^n a_i = 1) \\[10pt]
    \prod_{k=1}^n(a_k^2 + 1) = \left[\sum_{k=0}^{\left[\frac{n}{2}\right]}(-1)^k \sigma_{2k}\right]^2 + \left[\sum_{k=0}^{\left[\frac{n-1}{2}\right]}(-1)^k \sigma_{2k+1}\right]^2
\end{gather}
\end{thm}
% 最后也留一点
\vspace{2.5cm}


% chapters/ch02.tex


\section{预习题}
% === 第 1 题 ===
\begin{preview}{}{p1}
求下列各式的值:
\begin{enumerate}
    \item $S_1=\sum_{1 \leq i<j \leq n}(i-j)^2$
    \item $S_2=\sum_{1 \leq i<j \leq n} i \cdot j$
    \item $S_3=\sum_{1 \leq i<j \leq n}\left(i^2+j^2\right)$
\end{enumerate}
\end{preview}

% === 第 13 题 ===
\begin{preview}{2001 韩国}{p13}
已知实数 \( x_1, x_2, \cdots, x_n, y_1, y_2, \cdots, y_n \) 满足 \( \sum_{i=1}^n x_i^2=\sum_{i=1}^n y_i^2=1 \),证明:
\begin{equation}
    1-\left(x_1 y_1+x_2 y_2+\cdots+x_n y_n\right) \geq \frac{\left(x_1 y_2-x_2 y_1\right)^2}{2}
\end{equation}
\end{preview}



\newpage
\section{例题}



% === 第 2 题 ===
\begin{example}{}{p2}
设 \( x_1, x_2, \cdots, x_n \) 为正实数,且 \( x_{n+1}=x_1 \),证明:
\begin{equation}
    \frac{x_1 x_2}{x_1+x_2}+\frac{x_2 x_3}{x_2+x_3}+\cdots+\frac{x_n x_{n+1}}{x_n+x_{n+1}} \leq \frac{1}{2} \cdot\left(x_1+x_2+\cdots+x_n\right)
\end{equation}
\end{example}
\newpage

% === 第 3 题 ===
\begin{example}{}{p3}
设 \( x_1, x_2, \cdots, x_n \) 为正实数,且 \( x_{n+1}=x_1 \),证明:
\begin{equation}
    \frac{x_1^2}{x_2}+\frac{x_2^2}{x_3}+\cdots+\frac{x_n^2}{x_{n+1}} \geq \frac{2 x_1^2}{x_1+x_2}+\frac{2 x_2^2}{x_2+x_3}+\cdots+\frac{2 x_n^2}{x_n+x_{n+1}}
\end{equation}
\end{example}
\newpage


% === 第 8 题 ===
\begin{example}{1998 前南斯拉夫}{p8}
设正整数 \( n \geq 2 \),且 \( a_1, a_2, \cdots, a_n, b_1, b_2, \cdots, b_n \) 是正实数,证明:
\begin{equation}
    \left(\sum_{i \neq j} a_i b_j\right)^2 \geq \sum_{i \neq j} a_i a_j \cdot \sum_{i \neq j} b_i b_j
\end{equation}
\end{example}
\newpage










% 带系数

% === 第 6 题 ===
\begin{example}{2016 西部赛}{p6}
设 \( a_1, a_2, \cdots, a_n \) 为 \( n \) 个非负实数,记 \( S_k=\sum_{i=1}^k a_i(1 \leq k \leq n) \),证明:
\begin{equation}
    \sum_{i=1}^n\left(a_i S_i \cdot \sum_{j=i}^n a_j^2\right) \leq \sum_{i=1}^n\left(a_i S_i\right)^2
\end{equation}
\end{example}
\newpage

% === 第 7 题 ===
\begin{example}{}{p7}
设整数 \( n \geq 2, a_1, a_2, \cdots, a_n \) 是正实数,设 \( M=\max \left\{a_1, a_2, \cdots, a_n\right\} \),证明:
\begin{equation}
    M \cdot \sum_{i=1}^n i a_i \geq \frac{n+1}{n-1} \cdot \sum_{1 \leq i<j \leq n} a_i a_j
\end{equation}
\end{example}
\newpage


% 增量 换元



% === 第 9 题 ===
\begin{example}{2018 西部赛}{p9}
设整数 \( n \geq 2 \),正实数 \( a_1 \geq a_2 \geq \cdots \geq a_n \),且 \( a_1=a_{n+1} \),证明:
\begin{equation}
    \sum_{i=1}^n \frac{a_i}{a_{i+1}}-n \leq \frac{1}{2 a_1 a_n} \cdot \sum_{i=1}^n\left(a_i-a_{i+1}\right)^2
\end{equation}
\end{example}
\newpage


% lagrange 恒等式

% === 第 10 题 ===
\begin{example}{1991 IMO 预选}{p10}
给定整数 \( n \geq 2 \),且非负实数 \( x_1, x_2, \cdots, x_n \) 满足 \( x_1+x_2+\cdots+x_n=1 \),求
\begin{equation}
    P=\sum_{1 \leq i<j \leq n} x_i x_j \cdot\left(x_i+x_j\right)
\end{equation}
的最大值与最小值,并给出相应的取等条件。
\end{example}
\newpage






% === 第 14 题 ===
\begin{example}{2006 IMO 预选}{p14}
设 \( a_1, a_2, \cdots, a_n \) 是正实数,证明:
\begin{equation}
    \sum_{1 \leq i<j \leq n} \frac{a_i a_j}{a_i+a_j} \leq \frac{n}{2\left(a_1+a_2+\cdots+a_n\right)} \cdot \sum_{1 \leq i<j \leq n} a_i a_j
\end{equation}
\end{example}
\newpage

% === 第 15 题 ===
\begin{example}{}{p15}
设 \( a_i, b_i, c_i \) 均为实数,其中 \( i=1,2, \cdots, n \).且 \( \sum_{i=1}^n b_i^2=\sum_{i=1}^n b_i c_i=1, \sum_{i=1}^n a_i b_i=0 \),证明:
\begin{equation}
    \sum_{1 \leq i<j \leq n}\left(a_i c_j-a_j c_i\right)^2 \geq a_1^2+a_2^2+\cdots+a_n^2
\end{equation}
\end{example}
\newpage



% === 第 17 题 ===
\begin{example}{}{p17}
设 \( a_1, a_2, \cdots, a_n, b_1, b_2, \cdots, b_n \) 是实数,证明:
\begin{equation}
    \sum_{i=1}^n a_i b_i+\sqrt{\sum_{i=1}^n a_i^2 \cdot \sum_{i=1}^n b_i^2} \geq \frac{2}{n} \cdot \sum_{i=1}^n a_i \cdot \sum_{i=1}^n b_i
\end{equation}
\end{example}
\newpage

% 第 6 题
\begin{example}{}{}
设整数 $n\ge2$,$z_1, z_2, \dots, z_n$ 是复数,求证:
\begin{equation}
    \Bigl(\sum_{1\le i<j\le n}|z_{i}-z_{j}|\Bigr)^{2}\ge(n-1)\sum_{1\le i<j\le n}|z_{i}-z_{j}|^{2}.
\end{equation}
\end{example}
\newpage


% 第 4 题
\begin{example}{}{}
设 $n\geq 3$,记正实数 $a_1,a_2,\dots,a_n$ 的和为 $S$,证明:
\begin{equation}
    \sum_{i=1}^n a_i^2\sum_{i=1}^n\frac{1}{a_i}+n(n-2)S \leq S\sum_{i=1}^n\frac{S-a_i}{a_i}
\end{equation}
\end{example}
\newpage


% === 第 19 题 ===
\begin{example}{}{p19}
设正整数 $n\geq 2$,非负实数 $x_1,x_2,\dots,x_n$ 的和为 1,求
\begin{equation}
S=\sum_{1\leq i<j\leq n}(j-i)x_i x_j
\end{equation}
的最大值。
\end{example}
\newpage

% === 第 20 题 ===
\begin{example}{2004 俄罗斯}
设整数 \( n \geq 4 \),正实数 \( x_1, x_2, \cdots, x_n \) 满足 \( x_1 x_2 \cdots x_n=1 \),证明:
\begin{equation}
    \frac{1}{1+x_1+x_1 x_2}+\frac{1}{1+x_2+x_2 x_3}+\cdots+\frac{1}{1+x_n+x_n x_1}>1
\end{equation}
\end{example}
\newpage



% 第 1 题
\begin{example}{}{}
设正整数 $n\geq 3$,正实数 $x_1,x_2,\dots,x_n$ 满足 $\sum_{k=1}^n\frac{1}{1+x_k}=n-1$,证明:
\begin{equation}
    \sum_{1\leq i<j<k\leq n}\sqrt[3]{x_i x_j x_k} \leq\frac{n(n-2)}{6}
\end{equation}
\end{example}
\newpage









\section{作业题}
% === 第 4 题 ===
\begin{homework}{}{p4}
证明:
\begin{equation}
    \frac{4 \cdot \sum_{i=1}^n a_i \cdot \sum_{i=1}^n a_{i+1}}{\sum_{i=1}^n a_i+\sum_{i=1}^n a_{i+1}}\geq \sum_{i=1}^n \frac{4 a_i a_{i+1}}{a_i+a_{i+1}}
\end{equation}
\end{homework}


\newpage 
% === 第 5 题 ===
\begin{homework}{2016 IMC}{p5}
设实数 \( a_1, a_2, \cdots, a_n \) 和 \( b_1, b_2, \cdots, b_n \) 满足 \( a_i+b_i>0(i=1,2, \cdots, n) \),证明:
\begin{equation}
    \frac{\sum_{i=1}^n a_i \cdot \sum_{i=1}^n b_i-\left(\sum_{i=1}^n b_i\right)^2}{\sum_{i=1}^n\left(a_i+b_i\right)} \geq \sum_{i=1}^n \frac{a_i b_i-b_i^2}{a_i+b_i}
\end{equation}
\end{homework}



\newpage 
% === 第 11 题 ===
\begin{homework}{}{p11}
给定整数 \( n \geq 2 \),且非负实数 \( x_1, x_2, \cdots, x_n \) 满足 \( x_1+x_2+\cdots+x_n=1 \),求
\begin{equation}
    Q=\sum_{1 \leq i<j \leq n}\left(1+\sqrt{x_i x_j}\right) \cdot\left(\sqrt{x_i}+\sqrt{x_j}\right)
\end{equation}
的最大值,并给出相应的取等条件。
\end{homework}



\newpage 
% === 第 16 题 ===
\begin{homework}{}{p16}
已知实数 \( a_1, a_2, \cdots, a_n \) 满足 \( 0<a_i \leq \frac{1}{2}(i=1,2, \cdots, n) \),证明:
\begin{equation}
    \frac{\sum_{i=1}^n a_i^2}{\left(\sum_{i=1}^n a_i\right)^2} \geq \frac{\sum_{i=1}^n\left(1-a_i\right)^2}{\left[\sum_{i=1}^n\left(1-a_i\right)\right]^2}
\end{equation}
\end{homework}


\newpage 
% === 第 18 题 ===
\begin{homework}{}{p18}
设实数 $a_1, a_2, \cdots, a_n, b_1, b_2, \cdots, b_n$ 满足 $a_1 b_1+a_2 b_2+\cdots+a_n b_n=0$ ,求证:
\begin{equation}
 \sum_{k=1}^n a_k^2 \cdot \sum_{k=1}^n b_k^2 \geq \frac{4}{n^2} \cdot\left(\sum_{k=1}^n a_k\right)^2 \cdot\left(\sum_{k=1}^n b_k\right)^2 .
\end{equation}
\end{homework}



\newpage 
% 第 6 题
\begin{homework}{}{}
设 $a_1,a_2,\dots,a_n$ 为正实数,证明:
\begin{equation}
    \sum_{k=1}^n a_k\sum_{k=1}^n\frac{1}{a_k} \geq\frac{(n-1)^2\sum_{k=1}^n a_k^2}{\sum_{1\leq k<j\leq n}a_k a_j} +n^2-2n+2
\end{equation}
\end{homework}
% chapters/03_abel.tex

\chapter{Abel 变换}

\section{基础知识}
\begin{thm}{Abel 变换 (分部求和公式)}{}
令 $S_0=0$,$S_k=\sum_{i=1}^k a_i\ (1\leq k\leq n)$,则
\begin{equation}
    \sum_{k=1}^n a_k b_k = \sum_{k=1}^{n-1} S_k(b_k-b_{k+1}) + b_n S_n
\end{equation}
\end{thm}
\newpage
% \begin{thm}{钟开莱不等式}{}
% 设 $a_{1},a_{2},\dots,a_{n}$ 是实数, $b_{1}\ge b_{2}\ge\dots\ge b_{n}>0$,满足对任意 $1\le k\le n$,有 
% \begin{equation}
%     \sum_{i=1}^{k}a_{i}\ge\sum_{i=1}^{k}b_{i}.
% \end{equation}
% 求证:
% \begin{equation}
%     \sum_{i=1}^{n}a_{i}^{2}\ge\sum_{i=1}^{n}b_{i}^{2}.
% \end{equation}
% \end{thm}
% 钟开莱不等式
\begin{thm}{钟开莱不等式}{}
给定整数 $n\geq 2$,设 $a_1\geq a_2\geq\dots\geq a_n>0$ 且对任意 $1\leq k\leq n$ 有 $\sum_{i=1}^k a_i\leq\sum_{i=1}^k b_i$,求证:
\begin{align}
    \sum_{i=1}^n a_i^2 &\leq\sum_{i=1}^n b_i^2,\\
    \sum_{i=1}^n a_i^3 &\leq\sum_{i=1}^n a_i b_i^2.
\end{align}
\end{thm}
\newpage



% === 预习题部分 ===
% === 预习题部分 ===
% === 预习题部分 ===
\section{预习题}
\begin{preview}{}{}
给定正整数 $n$,对任意 $1\leq k\leq n$ 正实数 $a_1,a_2,\dots,a_n$ 满足 $a_1+a_2+\dots+a_k\leq k$,证明:
\begin{equation}
    \sum_{i=1}^n \frac{a_i}{i} \leq\sum_{i=1}^n \frac{1}{i}
\end{equation}
\end{preview}
\vspace{7cm}

\begin{preview}{1978 IMO}{}
已知 $a_1,a_2,\dots,a_n$ 是两两不同的正整数,证明:
\begin{equation}
    \sum_{i=1}^n \frac{a_i}{i^2} \geq\sum_{i=1}^n \frac{1}{i}
\end{equation}
\end{preview}












\newpage
% === 例题部分 ===
% === 例题部分 ===
% === 例题部分 ===
\section{例题}



\begin{example}{1994 USAMO}{}
设 $a_{1},a_{2},\dots,a_{n}$ 是正实数,满足对 $1\le k\le n$,有 $\sum_{i=1}^{k}a_{i}\ge\sqrt{k}.$
证明:
\begin{equation}
    \sum_{i=1}^{n}a_{i}^{2}\ge\frac{1}{4}\sum_{i=1}^{n}\frac{1}{i}.
\end{equation}
\end{example}
\newpage


\begin{example}{}{}
设 $a_{1},a_{2},\dots,a_{n}$ 是实数,求证:存在 $k\in\{1,2,\dots,n\}$,使得对任意 $1\ge b_{1}\ge b_{2}\ge\dots\ge b_{n}\ge0$ 都有
\begin{equation}
    \left|\sum_{i=1}^{n}a_{i}b_{i}\right|\le\left|\sum_{i=1}^{k}a_{i}\right|.
\end{equation}
\end{example}


\newpage 
\begin{example}{1999 APMO}{}
设 $\{a_n\}$ 是正项数列,满足对任意 $i,j\ge1$,有 $a_{i+j}\le a_{i}+a_{j}.$ 求证:对任意正整数 $n$,有
\begin{equation}
    \sum_{i=1}^{n}\frac{a_{i}}{i}\ge a_{n}.
\end{equation}
\end{example}
\newpage


\begin{example}{}{}
设正实数 $a_{1},a_{2},\dots,a_{n},b_{1},b_{2},\dots,b_{n}$ 满足 $b_{1}\ge b_{2}\ge\dots\ge b_{n}$,且对 $1\le k\le n$,有 $\prod_{i=1}^{k}a_{i}\ge\prod_{i=1}^{k}b_{i}.$ 
求证:
\begin{equation}
    \sum_{i=1}^{n}a_{i}\ge\sum_{i=1}^{n}b_{i}.
\end{equation}
\end{example}

\newpage
\begin{example}{加强形式的Chebyshev不等式}{}
设 $a_{1},a_{2},\dots,a_{n}$ 和 $b_{1},b_{2},\dots,b_{n}$ 是实数,满足
\begin{gather}
    a_{1}\ge\frac{a_{1}+a_{2}}{2}\ge\dots\ge\frac{a_{1}+a_{2}+\dots+a_{n}}{n}, \\
    b_{1}\ge\frac{b_{1}+b_{2}}{2}\ge\dots\ge\frac{b_{1}+b_{2}+\dots+b_{n}}{n}.
\end{gather}
求证:
\begin{equation}
    \sum_{i=1}^{n}a_{i}b_{i}\ge\frac{1}{n}\left(\sum_{i=1}^{n}a_{i}\right)\left(\sum_{i=1}^{n}b_{i}\right).
\end{equation}
\end{example}


\newpage
\begin{example}{}{}
已知实数 $a_1,a_2,\dots,a_n$,证明:
\begin{equation}
    \sum_{i=1}^n a_i^2 \geq\sum_{i=1}^{n-1} a_i a_{i+1} +\frac{3}{2(n+1)^3}\biggl(\sum_{i=1}^n a_i\biggr)^2
\end{equation}
\end{example}

\newpage
% 第 8 题
\begin{example}{2018 清华飞测}{}
给定正整数 $n$, 对任意 $1\leq k\leq n$ 正实数 $a_1,a_2,\dots,a_k$ 满足 $\sum_{i=1}^k a_i\leq k^2$,证明:
\begin{equation}
    \sum_{i=1}^n \frac{1}{a_i} >\frac{1}{4}\log_2 n
\end{equation}
\end{example}







\newpage
\begin{example}{}{}
给定整数 $n, k\ge2$. 设非负实数 $a_{1},a_{2},\dots,a_{n},c_{1},c_{2},\dots,c_{n}$ 满足:
\begin{enumerate}
    \item $a_{1}\ge a_{2}\ge\dots\ge a_{n},$ 且 $a_{1}+a_{2}+\dots+a_{n}=1$;
    \item 对 $1\le i\le n$,有 $\sum_{j=1}^{i}c_{j} \le i^k$.
\end{enumerate}
求 $c_{1}a_{1}^{k}+c_{2}a_{2}^{k}+\dots+c_{n}a_{n}^{k}$ 的最大值。
\end{example}











\newpage
% === 练习题部分 ===
\section{练习题}
\begin{homework}{}{}
给定整数 $n\ge2$ 以及不全为零的实数 $a_{1},a_{2},\dots,a_{n}.$ 求 $a_{1},a_{2},\dots,a_{n}$ 满足的充要条件,使得存在正整数 $x_{1}>x_{2}>\dots>x_n$,满足
\begin{equation}
    a_{1}x_{1}+a_{2}x_{2}+\dots+a_{n}x_{n}\ge0.
\end{equation}
\end{homework}



\newpage 
\begin{homework}{}{}
设整数 $n\ge2$,正实数 $a_{1},a_{2},\dots,a_{n}$ 满足 
\[
a_{1}\le a_{2},\quad a_{1}+a_{2}\le a_{3},\quad \dots,\quad a_{1}+a_{2}+\dots+a_{n-1}\le a_{n}.
\]
求证:
\begin{equation}
    \frac{a_{1}}{a_{2}}+\frac{a_{2}}{a_{3}}+\dots+\frac{a_{n-1}}{a_{n}}\le\frac{n}{2}.
\end{equation}
\end{homework}


\newpage
\begin{homework}{}{}
给定实数 $a_1,a_2,\dots,a_{n+1}$,记 $M=\max_{1\leq k\leq n}|a_k-a_{k+1}|$,证明:
\begin{equation}
    \left|\frac{1}{n}\sum_{k=1}^n a_k-\frac{1}{n+1}\sum_{k=1}^{n+1} a_k\right| \leq\frac{M}{2}
\end{equation}
\end{homework}


\newpage
\begin{homework}{}{}
设 $a_{1},a_{2},\dots,a_{n},b_{1},b_{2},\dots,b_{n}$ 是实数。求证:对任意实数 $x_{1}\le x_{2}\le \dots\le x_{n}$ 都有 $\sum_{i=1}^{n}a_{i}x_{i}\le\sum_{i=1}^{n}b_{i}x_{i}$ 的充要条件是:
对 $1\le k\le n-1$,有
\begin{equation}
    \sum_{i=1}^{k}a_{i}\ge\sum_{i=1}^{k}b_{i} \quad \text{且} \quad \sum_{i=1}^{n}a_{i}=\sum_{i=1}^{n}b_{i}.
\end{equation}
\end{homework}

\newpage
\begin{homework}{}{}
设正实数 $a_{1},a_{2},\dots,a_{n},b_{1},b_{2},\dots,b_{n}$ 满足 $a_{1}\le a_{2}\le\dots\le a_{n}$,且对 $1\le k\le n$,有 $\sum_{i=1}^{k}a_{i}\ge\sum_{i=1}^{k}b_{i}.$ 
求证:
\begin{equation}
    \sum_{i=1}^{n}\sqrt{a_{i}}\ge\sum_{i=1}^{n}\sqrt{b_{i}}.
\end{equation}
\end{homework}

\newpage
\begin{homework}{}{}
给定整数 $n\ge2$,设 $a_{1},a_{2},\dots,a_{n}$ 是正整数,满足对集合 $\{1,2,\dots,n\}$ 的任一非空子集 $I$,$\sum_{i\in I}a_{i}$ 互不相同。求:
\begin{enumerate}
    \item $\sum_{i=1}^{n}\sqrt{a_{i}}$ 的最小值;
    \item $\sum_{i=1}^{n}a_{i}^{2}$ 的最小值;
    \item $\sum_{i=1}^{n}\frac{1}{a_{i}}$ 的最大值。
\end{enumerate}
\end{homework}
\chapter{局部不等式}
% === 例题 ===
\section{例题}
% 1
\begin{example}{}{}
设整数 $n\ge3$,实数 $x_{1},x_{2},\dots,x_{n}\in[-1,1]$,且满足 $\sum_{i=1}^{n}x_{i}^{3}=0$。
求证:
\begin{equation}
    \sum_{i=1}^{n}x_{i}\le\frac{n}{3}.
\end{equation}
\end{example}
\newpage



% 2
\begin{example}{}{}
求所有的整数 $n\ge2$,使得存在实数 $x_{1},x_{2},\dots,x_{n}$,满足
\begin{equation}
    \sum_{k=1}^{n}x_{k}=0,\quad \sum_{k=1}^{n}x_{k}^{2}=1,\quad \sum_{k=1}^{n}x_{k}^{3}=2 \max_{1\le i\le n}x_{i}-\frac{2}{\sqrt{n}}.
\end{equation}
\end{example}
\newpage



% 3
\begin{example}{}{}
给定整数 $n\ge3$,设不全为 0 的实数 $x_{1},x_{2},\dots,x_{n}$ 满足 $\sum_{k=1}^{n}x_{k}=0$。记
\begin{equation}
    A=\Bigl(\sum_{k=1}^{n}x_{k}^{3}\Bigr)^{2},\quad B=\Bigl(\sum_{k=1}^{n}x_{k}^{2}\Bigr)^{3}.
\end{equation}
求 $\frac{A}{B}$ 的最大值。
\end{example}
\newpage

% 4
\begin{example}{}{}
设非负实数 $a_{1},a_{2},\dots,a_{100}$ 满足 $a_{i}+a_{i+1}+a_{i+2}\le1\ (1\le i\le100)$,其中 $a_{101}=a_{1},\ a_{102}=a_{2}$。求 $\sum_{i=1}^{100}a_{i}a_{i+2}$ 的最大值。
\end{example}
\newpage

% 5
\begin{example}{}{}
设整数 $n\ge2$,$z_1, z_2, \dots, z_n$ 是复数,求证:
\begin{equation}
    \Bigl(\sum_{1\le i<j\le n}|z_{i}-z_{j}|\Bigr)^{2}\ge(n-1)\sum_{1\le i<j\le n}|z_{i}-z_{j}|^{2}.
\end{equation}
\end{example}
\newpage






% bj
\begin{example}{}{}
设整数 $n\ge2$,正实数 $a_{1},a_{2},\dots,a_{n}$ 满足 $\sum_{i=1}^{n}a_{i}=1$。求证:
\begin{equation}
    \sum_{i=1}^{n}\frac{a_{i}a_{i+1}}{1-(a_{i}-a_{i+1})^{2}}\le\frac{1}{2},
\end{equation}
其中 $a_{n+1}=a_{1}$。
\end{example}
\newpage







% bj
\begin{example}{}{}
设整数 $n\ge3$,$a_{1},a_{2},\dots,a_{n}$ 是正实数,求证:
\begin{equation}
    \sum_{i=1}^{n}\frac{1}{a_{i}(a_{i}^{2}+a_{i-1}a_{i+1})}\le\sum_{i=1}^{n}\frac{1}{a_{i}a_{i+1}(a_{i}+a_{i+1})},
\end{equation}
其中 $a_{0}=a_{n},\ a_{n+1}=a_{1}$。
\end{example}
\newpage


% bj
\begin{example}{2012年APMO P5}{}
设整数 $n\ge2$,正实数 $a_{1},a_{2},\dots,a_{n}$ 满足 $\sum_{i=1}^{n}a_{i}^{2}=n$。求证:
\begin{equation}
    \sum_{1\le i<j\le n}\frac{1}{n-a_{i}a_{j}}\le\frac{n}{2}.
\end{equation}
\end{example}
\newpage


% % === 第 12 题 ===
% \begin{example}{}{p12}
% 给定整数 \( n \geq 3 \),且 \( a_i \geq 1(i=1,2, \cdots, n) \),证明:
% \begin{equation}
%     \sum_{i=1}^n a_i \cdot \sum_{i=1}^n \frac{1}{a_i} \leq n^2+\sum_{1 \leq i<j \leq n}\left|a_i-a_j\right|
% \end{equation}
% \end{example}
% \newpage


% bj
\begin{example}{}{}
设正实数 $x_{1},x_{2},\dots,x_{n}$ 满足 $x_1+x_2+\dots+x_n=n$。求证:
\begin{equation}
    \sum_{i=1}^{n}\frac{i}{1+x_{i}+\dots+x_{i}^{i-1}}\le\sum_{i=1}^{n}\frac{i+1}{1+x_{i}+\dots+x_{i}^{i}}.
\end{equation}
\end{example}
\newpage






% === 练习题 ===
% === 练习题 ===
% === 练习题 ===
\section{练习题}
% bj
\begin{homework}{}{}
设整数 $n\ge2$,实数 $x_{1},x_{2},\dots,x_{n}$ 满足 $\sum_{i=1}^{n}x_{i}=0,\ \sum_{i=1}^{n}x_{i}^{2}=1$。求证:存在 $a\in\{x_{1},x_{2},\dots,x_{n}\}$,使得对任意 $1\le j\le n$,都有
\begin{equation}
    \sum_{i=1}^{n}x_{i}^{3}\ge a+2x_{j}+nax_{j}^{2}.
\end{equation}
\end{homework}



% 第 14 题
\newpage
\begin{homework}{}{}
求最大的实数 $\lambda$,使得对任意正整数 $n$ 和任意实数 $a_{1},a_{2},\dots,a_{n}$,只要 $\sum_{i=1}^n a_i=0$,就有
\begin{equation}
    \sum_{i=1}^{n}a_{i}^{4}\ge\frac{\lambda}{n^{3}}\Bigl(\sum_{i=1}^{n}a_{i}^2\Bigr)^{2}.
\end{equation}
\end{homework}


\newpage
% bj
\begin{homework}{}{}
设整数 $n\ge3$,$x_{1},x_{2},\dots,x_{n}$ 是不小于 1 的实数,求证:
\begin{equation}
    \sum_{i=1}^{n}\frac{\sqrt{x_{i}x_{i+1}-1}}{x_{i+1}+x_{i+2}}\le\frac{1}{4}\sum_{i=1}^{n}x_{i},
\end{equation}
其中下标按模 $n$ 理解。
\end{homework}


% 第 11 题
\newpage
\begin{homework}{}{}
设实数 $a_{1}\ge a_{2}\ge\dots\ge a_{2n+1}$,求证:
\begin{equation}
    \Bigl(\sum_{i=1}^{2n+1}a_{i}\Bigr)^{2}\ge 4n\sum_{i=1}^{n+1}a_{i}a_{n+i}.
\end{equation}
\end{homework}
% \newpage

% 第 12 题
\newpage
\begin{homework}{}{}
设 $x_{1},x_{2},\dots,x_{n}$ 是正实数,求证:
\begin{equation}
    \frac{1}{1+x_{1}}+\frac{2}{1+x_{1}+x_{2}}+\dots+\frac{n}{1+x_{1}+\dots+x_{n}}\le\frac{n}{2}\sqrt{\frac{1}{x_{1}}+\frac{1}{x_{2}}+\dots+\frac{1}{x_{n}}}.
\end{equation}
\end{homework}
% \newpage



% example 60
\newpage
\begin{homework}{}{}
设整数 $n \ge 2$,且非负实数 $a_1, a_2, \dots, a_n$ 满足 $a_1 + a_2 + \dots + a_n = 3$,证明:
\begin{equation}
    \sum_{i=1}^n a_1a_2\dots a_{i-1}a_i \le 4
\end{equation}
\end{homework}
% \newpage

% example 59
\newpage
\begin{homework}{}{}
已知非负实数 $x_1, x_2, \dots, x_n$ 满足 $x_1 + x_2 + \dots + x_n = n$,证明:
\begin{equation}
    \frac{x_1}{1+x_1^2} + \frac{x_2}{1+x_2^2} + \dots + \frac{x_n}{1+x_n^2} \le \frac{1}{1+x_1} + \frac{1}{1+x_2} + \dots + \frac{1}{1+x_n}
\end{equation}
\end{homework}
% \newpage


% example 63
\newpage
\begin{homework}{}{}
已知正实数 $x_1, x_2, \dots, x_n$ 满足 $\sum_{i=1}^n \frac{1}{1+x_i} = 1$,证明:
\begin{equation}
    \frac{n-1}{n-1+x_1^2} + \frac{n-1}{n-1+x_2^2} + \dots + \frac{n-1}{n-1+x_n^2} \ge 1
\end{equation}
\end{homework}
% \newpage


% example 62
\newpage
\begin{homework}{2004 CTST}{}
已知非负实数 $x_1, x_2, \dots, x_n$ 满足 $\sum_{i=1}^n \frac{1}{1+x_i} = 1$,证明:
\begin{equation}
    \frac{x_1}{n-1+x_1^2} + \frac{x_2}{n-1+x_2^2} + \dots + \frac{x_n}{n-1+x_n^2} \le 1
\end{equation}
\end{homework}
% \newpage


% example 64
\newpage
\begin{homework}{}{}
设 $a_i \in [0, 1], i=1, 2, \dots, n$,记 $S = a_1^3 + a_2^3 + \dots + a_n^3$,证明:
\begin{equation}
    \frac{a_1}{2n+1+S-a_1^3} + \frac{a_2}{2n+1+S-a_2^3} + \dots + \frac{a_n}{2n+1+S-a_n^3} \le \frac{1}{3}
\end{equation}
\end{homework}




\newpage
% 第 2 题
\begin{homework}{}{}
设整数 $n\geq 3$,非负实数 $x_1,x_2,\dots,x_n$ 的和为 1,证明:
\begin{equation}
    \Biggl[\sum_{i=1}^n\frac{x_i}{1+(n-2)x_i}\Biggr]^2 \leq\sum_{i=1}^n\frac{x_i^2}{[1+(n-2)x_i]^2} +\frac{n}{4(n-1)}
\end{equation}
\end{homework}



% example 66
\newpage
\begin{homework}{}{}
设整数 $n \ge 2$,且正实数 $a_1, a_2, \dots, a_n$ 满足 $\sum_{i=1}^n a_i = n$,证明:
\begin{equation}
    \sum_{i=1}^n \sqrt[n+1]{a_i} \ge \frac{2n}{n+1} \cdot \left( \sum_{1 \le i < j \le n} a_i a_j - \frac{n^2 - 2n - 1}{2} \right)
\end{equation}
\end{homework}
% \newpage


% example 57
\newpage
\begin{homework}{2007 白俄罗斯}{}
已知 $x_1, x_2, \dots, x_{n+1}$ 均为正数,证明:
\begin{equation}
    \frac{1}{x_1} + \frac{x_1}{x_2} + \frac{x_1x_2}{x_3} + \dots + \frac{x_1x_2\dots x_n}{x_{n+1}} \ge 4(1 - x_1x_2\dots x_{n+1})
\end{equation}
\end{homework}



% example 71
\newpage
\begin{homework}{1993 圣彼得堡}{}
设 $a_i \in [-1, 1],\ a_i a_{i+1} \ne -1,\ i=1, 2, \dots, n$,且 $a_{n+1} = a_1$,证明:
\begin{equation}
    \frac{1}{1+a_1a_2} + \frac{1}{1+a_2a_3} + \dots + \frac{1}{1+a_na_{n+1}} \ge \frac{1}{1+a_1^2} + \frac{1}{1+a_2^2} + \dots + \frac{1}{1+a_n^2}
\end{equation}
\end{homework}



% example 65
\newpage
\begin{homework}{}{}
设整数 $n \ge 2$,且 $x_1, x_2, \dots, x_n$ 均为正实数,证明:
\begin{equation}
    \sum_{i=1}^n \left(\frac{x_i}{\sum_{j=1}^n x_j - x_i}\right)^{\frac{n-1}{n}} \ge \frac{n \cdot \sqrt[n]{n-1}}{n-1}
\end{equation}
\end{homework}
% \newpage
% chapters/11_adjustment_method.tex

\chapter{调整法}
独立变量:固定其余变量,考虑单一变量的影响,利用单调性。约束变量:固定其余变量,固定两个变量的和或积,考虑变化影响。  

\begin{thm}{}{}
\begin{enumerate}
\item 设 \( a_k \in \mathbb{R} \),\( \sum_{k=1}^n a_k = t \),若在 \( a_1 \) 最大、\( a_2 \) 最小的前提下能证明
\[ f(a_1,a_2,\cdots,a_n) \geq f\left(\frac{t}{n}, a_1+a_2-\frac{t}{n}, a_3,\cdots,a_n\right), \]
则可经过 \( n-1 \) 次调整将 \( a_1,a_2,\cdots,a_n \) 全调为 \( \frac{t}{n} \)。
\item 设 \( a_k \in \mathbb{R}^+ \),\( \prod_{k=1}^n a_k = t \),若在 \( a_1 \) 最大、\( a_2 \) 最小的前提下能证明
\[ f(a_1,a_2,\cdots,a_n) \geq f\left(\sqrt[n]{t}, \frac{a_1a_2}{\sqrt[n]{t}}, a_3,\cdots,a_n\right), \]
则可经过 \( n-1 \) 次调整将 \( a_1,a_2,\cdots,a_n \) 全调为 \( \sqrt[n]{t} \)。
\end{enumerate}
\end{thm}
\begin{thm}{端点调整}{}
若 \( a_k \in [a,b] \),将 \( f(a_1,a_2,\cdots,a_n) \) 视为 \( a_k \) 的一元函数 \( g(x) \),且 \( g'(x) \) 单调,则可将 \( a_k \) 调为 \( a \) 或 \( b \)。
% (单增最大值在右端点,单减最小值在左端点)注:有时不需 \( g'(x) \) 单调,只需 \( g'(x) \) 的某一部分因式单调,而这一部分与 \( g'(x) \) 同号即可!
\end{thm}

\begin{thm}{磨光变换法(无限调整法、SMV定理)}{}
\vspace{1cm}
\end{thm}
\begin{thm}{EV(Equal Variable)定理}{}
\vspace{1cm}
\end{thm}
\newpage

% \begin{thm}{}{}
% 设 \( a_k \geq 0 \),\( \sum_{k=1}^n a_k = t \),正整数 \( m \geq 2 \),若在 \( a_1+a_2 \leq \frac{2t}{m} \) 的前提下能证明
% \[ f(a_1,a_2,\cdots,a_n) \geq f(0, a_1+a_2, a_3,\cdots,a_n), \]
% 则可经过 \( n-m+1 \) 次调整将 \( a_1,a_2,\cdots,a_n \) 调成 \( n-m+1 \) 个 \( 0 \)。
% \end{thm}


% \begin{thm}{}{}
% 例1:\( a_k \geq 0 \),\( \sum_{k=1}^n a_k = 2 \),\( n \geq 3 \),求 \( 3a_1 + \sum_{k=2}^n 2^{k+1}a_2\cdots a_k \) 最大值。
%     简解:不妨设 \( a_1 \geq a_2 \geq \cdots \geq a_n \),令上式为 \( f(a_1,a_2,\cdots,a_n) \),则
%     \[ \text{易证 } f(a_1,a_2,\cdots,a_n) \leq f(a_1+a_4+a_5+\cdots+a_n, a_2, a_3, 0, 0,\cdots,0). \]
% \end{thm}

% \begin{thm}{}{}
% 例2:\( a_k \geq 0 \),\( \sum_{k=1}^n a_k = 2 \),求 \( \sum_{k=1}^n \frac{1}{\sum_{j=k}^n a_j^2} \) 最小值。
%     简解:不妨设 \( a_1 \geq a_2 \geq \cdots \geq a_n \),令上式为 \( f(a_1,a_2,\cdots,a_n) \),则易证
%     \[ f(a_1,a_2,\cdots,a_n) \geq f\left(a_1+\frac{a_3+a_4+\cdots+a_n}{2}, a_2+\frac{a_3+a_4+\cdots+a_n}{2}, 0, 0,\cdots,0\right). \]
% \end{thm}

% \begin{thm}{}{}
% (多次不同调整)例:若干个不同正整数之和为2026,求积最大值。
% 分6步调整:
% (1) 设这些数为 \( a_1 < a_2 < \cdots < a_n \),先说明 \( a_1 > 1 \);
% (2) 说明 \( a_1 < 5 \);
% (3) 说明 \( a_1 < 4 \);
% (4) 说明 \( a_{k+1}-a_k \leq 2 \);
% (5) 说明使 \( a_{k+1}-a_k = 2 \) 的 \( k \) 至多1个;
% (6) 说明 \( a_{k+1}-a_k = 2 \) 的 \( k \) 恰有1个。
% 最后分别在 \( a_1=2 \) 和 \( a_1=3 \) 时求出积进行比较即可。
% \end{thm}

% \begin{thm}{}{}
% (多次不同调整)例2:给定正整数 \( n \geq 3 \),设 \( a_k \in [-1,1] \),\( \sum_{k=1}^n a_k = 0 \),求 \( \sum_{k=1}^n a_k^{2025} \) 最大值。
% 分3步调整:
% (1) 若 \( a_1 > 0 > a_2 \) 且 \( a_1+a_2 \leq 0 \),则将 \( a_1,a_2 \) 调为 \( 0, a_1+a_2 \);
% (2) 若 \( a_1 > 0 > a_2 \) 且 \( a_1+a_2 > 0 \),将 \( a_1,a_2 \) 调为 \( 1, a_1+a_2-1 \);
% (3) 利用赫尔德不等式将所有负数调成相等。
% 如此正数全为1,负数全相等,再设出正数个数 \( x \),化为 \( x \) 的一元函数。
% \end{thm}


% \begin{thm}{}{}
%     (2个方向的调整)例:2025年CMO第5题:若 \( n \) 元对称不等式 \( f(a_1,a_2,\cdots,a_n) \leq 0 \) 中关于 \( a_1a_2 \) 为开口向上的二次函数,\( \sum_{k=1}^n a_k = nt \),\( a_k \geq 0 \)。
% 由于二次函数最大值在端点,若设 \( a_1 \) 最大、\( a_2 \) 最小,则
% \[ 0(a_1+a_2) \leq a_1a_2 \leq t(a_1+a_2 - t), \]
% 故有 \( f(a_1,a_2,\cdots,a_n) \leq f(0, a_1+a_2, a_3,\cdots,a_n) \) 或 \( f(t, a_1+a_2 - t, a_3,\cdots,a_n) \)。
% 从而经有限次调整后,\( a_1,a_2,\cdots,a_n \) 化为 \( n-m \) 个 \( 0 \) 和 \( m \) 个 \( \frac{nt}{m} \)。
% \end{thm}


% \begin{thm}{}{}
% (\( 2^n \) 元不等式的调整)若 \( 2^n \) 元对称不等式 \( f(a_1,a_2,\cdots,a_{2^n}) \) 中,可将 \( a_1,a_2 \) 调为 \( \frac{a_1+a_2}{2}, \frac{a_1+a_2}{2} \)(或 \( \sqrt{a_1a_2}, \sqrt{a_1a_2} \)),其中 \( a_1,a_2 \) 为任意2个数(不必最大或最小),则可经过 \( n \times 2^{n-1} \) 次调整后将所有数调至相等。
% \end{thm}

% \begin{thm}{}{}
% (部分调整)指仅调整 \( n \) 个变量中的 \( m \) 个变量(\( m < n \)),调整方式变化万千,不再列举。
% \end{thm}


% === 预习题 ===
\section{预习题}
\begin{preview}{}{}
已知 $0 \le a, b, c \le 1$,证明:
\begin{equation}
    \frac{a}{bc+1} + \frac{b}{ca+1} + \frac{c}{ab+1} \le 2
\end{equation}
\end{preview}


\begin{preview}{}{}
设 $x, y, z$ 都是非负实数且 $x+y+z=1$,证明:
    \begin{equation}
        yz+zx+xy-2xyz \le \frac{7}{27}
    \end{equation}
\end{preview}


\begin{preview}{}{}
设正实数 $a, b, c, d$ 满足 $abcd = 1$,证明:
\begin{equation}
        \frac{1}{a} + \frac{1}{b} + \frac{1}{c} + \frac{1}{d} + \frac{4}{a+b+c+d} \ge 5
\end{equation}
\end{preview}


\begin{preview}{}{}
设实数 $a, b, c$ 满足 $a+b+c=1,\ abc>0$,证明:
\begin{equation}
    ab+bc+ca < \frac{\sqrt{abc}}{2} + \frac{1}{4}
\end{equation}
\end{preview}
\newpage


% === 例题 ===
\section{例题}
\begin{example}{}{}
已知非负实数\( x_1, x_2, \dots, x_n \ (n \geq 3) \)满足不等式\( x_1 + x_2 + \dots + x_n \leq \frac{1}{2} \),求\( (1 - x_1)(1 - x_2)\cdots(1 - x_n) \)的最小值。
\end{example}
\newpage
\begin{example}{}{}
已知正实数 \( x_1, x_2, \dots, x_n \) 满足 \( x_1 + x_2 + \dots + x_n = 1 \),证明:
\begin{equation}
\frac{(1 - x_1)(1 - x_2)\cdots(1 - x_n)}{x_1x_2\cdots x_n} \geq (n - 1)^n.
\end{equation}
\end{example}
\newpage
\begin{example}{}{}
设整数 \( n \geq 2 \),正实数 \( x_1, x_2, \dots, x_n \) 满足 \( x_i x_j \geq 1 \)(其中 \( i \neq j \),\( 1 \leq i \leq n \),\( 1 \leq j \leq n \)),证明:
\begin{equation}
\frac{1}{1 + x_1} + \frac{1}{1 + x_2} + \dots + \frac{1}{1 + x_n} \geq \frac{n}{1 + \sqrt[n]{x_1x_2\cdots x_n}}.
\end{equation}
\end{example}
\newpage
\begin{example}{}{}
已知正实数 \( a_1, a_2, \dots, a_n \) 满足 \( \sum_{i=1}^{n} a_i = n \),证明:
\begin{equation}
a_1^2 + a_2^2 + \dots + a_n^2 + a_1a_2\cdots a_n \geq n + 1.
\end{equation}
\end{example}
\newpage




\begin{example}{}{}
设 \( x_i \in (0, 1] \)(\( 1 \leq i \leq n \)),\( 0 < \lambda \leq 2 \),证明:
\begin{equation}
\sum_{i=1}^{n} \left[1 + (i - 1)\lambda\right] \cdot x_i^2 \geq \frac{2 + (n - 1)\lambda}{2n} \cdot \left( \sum_{i=1}^{n} x_i \right)^2.
\end{equation}
\end{example}
\newpage



\begin{example}{}{}
已知 \( a_1, a_2, \dots, a_n \) 为正实数,证明:
\begin{equation}
\frac{\sum_{i=1}^{n} a_i}{n} - \sqrt[n]{\prod_{i=1}^{n} a_i} \leq \max_{1 \leq i < j \leq n} \left\{ (\sqrt{a_i} - \sqrt{a_j})^2 \right\}.
\end{equation}
\end{example}
\newpage



% % 例 5
% \begin{example}{}{}
% 给定不小于 3 的正整数 $n$。已知实数 $x_1, x_2, \dots, x_n$ 满足 $x_1+x_2+\dots+x_n=0$ 且 $x_1^2+x_2^2+\dots+x_n^2=1$,求 $(2+x_1)(2+x_2)\dots(2+x_n)$ 的最大值。
% \end{example}
% \newpage

% === 练习题 ===
\section{练习题}
\begin{homework}{}{}
给定正整数 \( n \geq 4 \),且 \( \sum_{i=1}^{n} x_i \geq n \),\( \sum_{i=1}^{n} x_i^2 \geq n^2 \),证明:这 \( n \) 个数中一定有一个数大于等于2。
\end{homework}

\newpage
\begin{homework}{}{}
设非负实数 \( a_1, a_2, \dots, a_n \) 中的最大数为 \( a \),证明:
\begin{equation}
\frac{a_1^2 + a_2^2 + \dots + a_n^2}{n} \leq \left( \frac{a_1 + a_2 + \dots + a_n}{n} \right)^2 + \frac{a^2}{4}.
\end{equation}
并确定不等式等号成立的条件。
\end{homework}


\newpage
\begin{homework}{}{}
设 $a_1, a_2, \dots, a_n$ 为 $n\ (n \ge 2)$ 个互不相同的实数,记
\begin{equation}
    S = a_1^2 + a_2^2 + \dots + a_n^2, \quad M = \min_{1 \le i < j \le n} (a_i - a_j)^2.
\end{equation}
证明:$12S \ge n(n-1)M$.
\end{homework}

\newpage
\begin{homework}{}{}
设整数 \( n \geq 2 \),\( \alpha_i \in \left(0, \frac{\pi}{2}\right) \)(\( i = 1, 2, \dots, n \)),证明:
\begin{equation}
\prod_{i=1}^{n} \cos\alpha_i \cdot \sum_{i=1}^{n} \tan\alpha_i \leq \frac{(n - 1)^{\frac{n - 1}{2}}}{n^{\frac{n - 2}{2}}}.
\end{equation}
\end{homework}



% \section{简单备选题}

% % 预 4
% \begin{example}{}{}
% 已知正实数 $x_1, x_2, \dots, x_n$,其中任意两个数 $x_i, x_j\ (i \ne j)$,都有 $x_i x_j \ge 1$。证明:
% \begin{equation}
%     \frac{1}{1+x_1} + \frac{1}{1+x_2} + \dots + \frac{1}{1+x_n} \ge \frac{n}{1+\sqrt[n]{x_1 x_2 \dots x_n}}
% \end{equation}
% \end{example}





% % 练 1
% \begin{problem}{}{}
% 设 $A, B, C$ 为 $\triangle ABC$ 的三个角,证明:
% \begin{equation}
%     \cos A + \cos B + \cos C > 1
% \end{equation}
% \end{problem}

% % 例 1
% \begin{example}{}{}
% 设 $n$ 为正整数,$x_1, x_2, \dots, x_n$ 为非负实数,且 $x_1 + x_2 + \dots + x_n = \pi$。记 $M$ 为 $\sin^2 x_1 + \sin^2 x_2 + \dots + \sin^2 x_n$ 的最大值,证明:当 $n > 3$ 时,$M$ 为定值。
% \end{example}
% \newpage

% % 习 4
% \begin{problem}{}{}
% \begin{enumerate}
%     \item 设 $a, b, c, d \ge 0$,且 $a+b+c+d=4$。证明:
%     \begin{equation}
%         bcd+cda+dab+abc-abcd \le \frac{1}{2} \cdot (ab+ac+ad+bc+bd+cd)
%     \end{equation}
%     \item 试求实数 $k$ 的取值范围,使不等式
%     \begin{equation}
%         0 \le xyz+yzw+zwx+wxy - kxyzw < \frac{16-k}{256}
%     \end{equation}
%     对一切满足 $x+y+z+w=1$ 的所有非负实数 $x, y, z, w$ 恒成立。
% \end{enumerate}
% \end{problem}



% % 例 6
% \begin{problem}{}{}
% \begin{enumerate}
%     \item 设 $a, b, c, d \ge 0$,且 $a+b+c+d=1$。证明:
%     \begin{equation}
%         bcd+cda+dab+abc \le \frac{1}{27} + \frac{176}{27}abcd
%     \end{equation}
%     \item 试求实数 $k$ 的取值范围,使不等式
%     \begin{equation}
%         0 < xyz+yzw+zwx+wxy - kxyzw \le \frac{4-k}{27}(xy+xz+xw+yz+yw+zw)
%     \end{equation}
%     对一切满足 $x+y+z+w=4$ 的所有非负实数 $x, y, z, w$ 恒成立。
% \end{enumerate}
% \end{problem}





% % 练 2
% \begin{problem}{}{}
% 已知实数 $a, b, c, d$ 满足 $a \ge c,\ b \ge d > 0$,试求
% \begin{equation}
%     S = \frac{a}{a+b} + \frac{b}{b+c} + \frac{c}{c+d} + \frac{d}{d+a}
% \end{equation}
% 的取值范围。
% \end{problem}


% % 练 3
% \begin{problem}{}{}
% \begin{enumerate}
%     \item 求实数 $k$ 的最大值,使得对任意实数 $x \ge 1, y \ge 1$,恒有
%     \begin{equation}
%         \frac{x^2}{1+x} + \frac{y^2}{1+y} + (x-1)(y-1) \ge kxy
%     \end{equation}
%     \item 已知正实数 $a, b, c$ 满足 $abc=1$,证明:
%     \begin{equation}
%         \frac{1}{a} + \frac{1}{b} + \frac{1}{c} + \frac{3}{a+b+c} \ge 4
%     \end{equation}
% \end{enumerate}
% \end{problem}


% % 练 4
% \begin{problem}{}{}
% 设正实数 $a, b, c$ 满足 $\min\{ab, bc, ca\} \ge 1$,证明:
% \begin{equation}
%     \sqrt[3]{(a^2+1)(b^2+1)(c^2+1)} \le \left(\frac{a+b+c}{3}\right)^2 + 1
% \end{equation}
% \end{problem}


% % 习 1
% \begin{problem}{}{}
% 试求最大的实数 $k$,满足:对任意正实数 $a, b, c\ (a^2 > bc)$,都有
% \begin{equation}
%     (a^2-bc)^2 > k \cdot (b^2-ca)(c^2-ab)
% \end{equation}
% \end{problem}

% % 习 3
% \begin{problem}{}{}
% 设正实数 $a, b, c, d$ 满足 $abcd=1$,证明:
% \begin{equation}
%     \frac{1}{a} + \frac{1}{b} + \frac{1}{c} + \frac{1}{d} + \frac{9}{a+b+c+d} \ge \frac{25}{4}
% \end{equation}
% \end{problem}
% \newpage

% === 备用题 ===

% % 备 1
% \begin{problem}{}{}
% 求所有的正实数 $t$,使得对任意正整数 $n \ge 2$ 和满足 $\sum_{i=1}^n a_i = n$ 的正实数 $a_1, a_2, \dots, a_n$,总有
% \begin{equation}
%     \sum_{i=1}^n \frac{1}{a_i} - t \cdot \prod_{i=1}^n \frac{1}{a_i} \le n-t
% \end{equation}
% \end{problem}
% \newpage

% % === 大显身手 (更多练习) ===

% % === 习题 ===

% % 习 2
% \begin{problem}{}{}
% 给定正整数 $n \ge 2$,非负实数 $x_1, x_2, \dots, x_n$ 满足 $x_1+x_2+\dots+x_n=1$。求 $\sum_{i=1}^n (x_i^4 - x_i^5)$ 的最大值。
% \end{problem}
% \newpage

% % 例 4
% \begin{example}{}{}
% 设 $a_1, a_2, \dots, a_n$ 是给定的不全为零的实数,$r_1, r_2, \dots, r_n$ 为实数,如果不等式
% \begin{equation}
%     r_1(x_1-a_1) + r_2(x_2-a_2) + \dots + r_n(x_n-a_n) \le \sqrt{x_1^2+x_2^2+\dots+x_n^2} - \sqrt{a_1^2+a_2^2+\dots+a_n^2}
% \end{equation}
% 对任何实数 $x_1, x_2, \dots, x_n$ 成立,求 $r_1, r_2, \dots, r_n$ 的值。
% \end{example}
% \newpage

\part{高级不等式}
% chapters/14_bernoulli.tex

\chapter{伯努利不等式}
\section{基础知识}

% \begin{thm}{伯努利不等式 (Bernoulli's Inequality)}{bernoulli_basic}
% 对任意实数 $x > -1$ 和整数 $n \ge 0$,有:
% \begin{equation}
%     (1+x)^n \ge 1+nx
% \end{equation}
% 当且仅当 $n=0, 1$ 或 $x=0$ 时取等号。
% \end{thm}
% % \vspace{2cm}

\begin{thm}{伯努利不等式 (Bernoulli's Inequality)}{bernoulli_gen}
对任意实数 $x > -1$ 和实数 $\alpha$,有:
\begin{enumerate}
    \item 当 $\alpha \in (-\infty, 0] \cup [1, +\infty)$ 时:
    \begin{equation}
        (1+x)^\alpha \ge 1+\alpha x
    \end{equation}
    \item 当 $\alpha \in (0, 1)$ 时:
    \begin{equation}
        (1+x)^\alpha \le 1+\alpha x
    \end{equation}
\end{enumerate}
当且仅当 $\alpha = 0, 1$ 或 $x=0$ 时取等号。
\end{thm}


\newpage 
\begin{thm}{广义伯努利不等式}{}
设 $x_1, x_2, \dots, x_n$ 为符号相同的实数,且 $x_i > -1$,则:
\begin{equation}
    (1+x_1)(1+x_2)\cdots(1+x_n) \ge 1 + x_1 + x_2 + \dots + x_n
\end{equation}
\end{thm}






% === 例题部分 ===

\newpage  
\section{例题}
% 例 1
\begin{example}{}{}
设正实数 $a_{1},a_{2},\dots,a_{n}\le1$,求证:
\begin{equation}
    (1+a_{1})^{\frac{1}{a_{2}}}(1+a_{2})^{\frac{1}{a_{3}}}\dots(1+a_{n})^{\frac{1}{a_{1}}}\ge2^{n}.
\end{equation}
\end{example}
\newpage



% 例 3
\begin{example}{}{}
设实数 $a_{1},a_{2},\dots,a_{n}\ge1$,求证:
\begin{equation}
    (1+a_{1})(1+a_{2})\dots(1+a_{n})\ge\frac{2^{n}}{n+1}(1+a_{1}+a_{2}+\dots+a_{n}).
\end{equation}
\end{example}
\newpage

% 例 4
\begin{example}{}{}
求最小的实数 $\lambda$,使得对任意正整数 $n$ 及任意满足 $\sum_{i=1}^{n}a_{i}=1$ 的正实数 $a_{1},a_{2},\dots,a_{n}$ 都有
\begin{equation}
    \lambda\prod_{i=1}^{n}(1-a_{i})\ge1-\sum_{i=1}^{n}a_{i}^{2}.
\end{equation}
\end{example}
\newpage

% 例 5
\begin{example}{}{}
设 $a_{1},a_{2},\dots,a_{n}$ 是正实数,求证:
\begin{equation}
    \prod_{i=1}^{n}(a_{i}^{2}+n-1)\ge n^{n-2}\Bigl(\sum_{i=1}^{n}a_{i}\Bigr)^{2}.
\end{equation}
\end{example}
\newpage

% 例 6
\begin{example}{}{}
设正实数 $a_{1},a_{2},\dots,a_{n}$ 满足 $\sum_{i=1}^{n}a_{i}=n$。求证:
\begin{equation}
    (a_{1}^{n}+1)(a_{2}^{n}+1)\dots(a_{n}^{n}+1)\ge2^{n}.
\end{equation}
\end{example}
\newpage

% 例 7
\begin{example}{}{}
设 $a_{1},a_{2},\dots,a_{n}$ 是不全为 1 的正整数,满足 $\frac{1}{a_{1}}+\frac{1}{a_{2}}+\dots+\frac{1}{a_{n}}=k$ 是整数。求证:多项式
\begin{equation}
    P(x)=a_{1}a_{2}\dots a_{n}(x+1)^{k}-(x+a_{1})(x+a_{2})\dots(x+a_{n})
\end{equation}
没有正根。
\end{example}
\newpage

% 例 8
\begin{example}{}{}
设整数 $n\ge3$,多项式 $P(x)=x^{n}+a_{n-1}x^{n-1}+a_{n-2}x^{n-2}+\dots+a_{1}x+a_0$ 有 $n$ 个实根,且都在区间 $(0,1)$ 内。求证:
\begin{equation}
    \sum_{i=1}^{n-2}ia_{i}>0.
\end{equation}
\end{example}
\newpage


% 例 2
\begin{thm}{Mitrinovic 不等式}{}
设整数 $n\ge2$,$a_{1},a_{2},\dots,a_{n}$ 是正实数,求证:对 $a_{1},a_{2},\dots,a_{n}$ 的任一排列 $b_{1},b_{2},\dots,b_{n}$,都有
\begin{equation}
    \sum_{i=1}^{n}a_{i}^{b_{i}}>1.
\end{equation}
\end{thm}
\newpage



% === 作业题部分 ===
\section{练习题}
% 作业题 1
\begin{homework}{}{}
设实数 $a_{1},a_{2},\dots,a_{n}$ 满足 $a_{i}\ge-1,\ i=1,2,\dots,n$ 且 $\sum_{i=1}^{n}a_{i}=0$。求证:
\begin{equation}
    \prod_{i=1}^{n}(1+a_{i})+\frac{n}{4}\sum_{i=1}^{n}a_{i}^{2}\ge1.
\end{equation}
\end{homework}
% 作业题 2
\newpage 
\begin{homework}{}{}
给定整数 $n\ge2$,求最小的实数 $\lambda$,使得对任意非负实数 $a_{1},a_{2},\dots,a_{n}$,都有
\begin{equation}
    \sum_{i=1}^{n}\sqrt{a_{i}}\le\sqrt{\prod_{i=1}^{n}(a_{i}+\lambda)}.
\end{equation}
\end{homework}
% chapters/15_other_famous.tex

\chapter{其他著名不等式}

\section{基础知识}

\begin{thm}{Jensen 不等式}{}
\begin{enumerate}
    \item 设 $f(x)$ 是 $[a,b]$ 上的下凸函数,则对任意 $x_{1},x_{2},\dots,x_{n}\in [a,b]$,有
    \begin{equation}
        f\left(\frac{1}{n}\sum_{i=1}^{n}x_{i}\right)\le\frac{1}{n}\sum_{i=1}^{n}f(x_{i})
    \end{equation}
    当且仅当 $x_{1}=x_{2}=\dots=x_{n}$ 时等号成立。
    
    \item 设 $f(x)$ 是 $[a,b]$ 上的下凸函数,正实数 $\lambda_{1},\lambda_{2},\dots,\lambda_{n}$ 满足 $\sum_{i=1}^{n}\lambda_{i}=1$。则对任意 $x_{1},x_{2},\dots,x_{n}\in[a,b]$,有
    \begin{equation}
        f\left(\sum_{i=1}^{n}\lambda_{i}x_{i}\right)\le\sum_{i=1}^{n}\lambda_{i}f(x_{i})
    \end{equation}
    当且仅当 $x_{1}=x_{2}=\dots=x_{n}$ 时等号成立。
\end{enumerate}
\end{thm}
% \vspace{2cm}
\newpage
\begin{thm}{加权均值不等式}{}
设正实数 $\lambda_{1},\lambda_{2},\dots,\lambda_{n}$ 满足 $\sum_{i=1}^{n}\lambda_{i}=1$,$a_{1},a_{2},\dots,a_{n}$ 是正实数,则
\begin{equation}
    \sum_{i=1}^{n}\lambda_{i}a_{i}\ge\prod_{i=1}^{n}a_{i}^{\lambda_{i}}
\end{equation}
当且仅当 $a_{1}=a_{2}=\dots=a_{n}$ 时等号成立。
\end{thm}
\vspace{6.5cm}

\begin{thm}{幂平均不等式}{}
设 $a_{1},a_{2},\dots,a_{n}$ 是正实数,非零实数 $\alpha<\beta$。则
\begin{equation}
    \left(\frac{1}{n}\sum_{i=1}^{n}a_{i}^{\alpha}\right)^{\frac{1}{\alpha}}\le\left(\frac{1}{n}\sum_{i=1}^{n}a_{i}^{\beta}\right)^{\frac{1}{\beta}}
\end{equation}
当且仅当 $a_{1}=a_{2}=\dots=a_{n}$ 时等号成立。
\end{thm}
% \vspace{5cm}
\newpage 

\begin{thm}{范数不等式}{}
设 $a_{1},a_{2},\dots,a_{n}$ 是非负实数,正实数 $\alpha<\beta$。则
\begin{equation}
    \left(\sum_{i=1}^{n}a_{i}^{\beta}\right)^{\frac{1}{\beta}}\le\left(\sum_{i=1}^{n}a_{i}^{\alpha}\right)^{\frac{1}{\alpha}}
\end{equation}
当且仅当 $a_{1},a_{2},\dots,a_{n}$ 中至少有 $n-1$ 个为 0 时等号成立。
\end{thm}
\newpage

\begin{thm}{Hölder 不等式}{}
设 $a_{1},a_{2},\dots,a_{n},b_{1},b_{2},\dots,b_{n}$ 是正实数,$p,q$ 是大于 1 的实数,满足 $\frac{1}{p}+\frac{1}{q}=1$。则
\begin{equation}
    \sum_{i=1}^{n}a_{i}b_{i}\le\left(\sum_{i=1}^{n}a_{i}^{p}\right)^{\frac{1}{p}}\left(\sum_{i=1}^{n}b_{i}^{q}\right)^{\frac{1}{q}}
\end{equation}
当且仅当 $\frac{a_{1}^{p}}{b_{1}^{q}}=\frac{a_{2}^{p}}{b_{2}^{q}}=\dots=\frac{a_{n}^{p}}{b_{n}^{q}}$ 时等号成立。
\end{thm}
\newpage

\begin{thm}{Minkowski 不等式}{}
设 $a_{1},a_{2},\dots,a_{n},b_{1},b_{2},\dots,b_{n}$ 是正实数,实数 $p\ge1$。则
\begin{equation}
    \left(\sum_{i=1}^{n}(a_{i}+b_{i})^{p}\right)^{\frac{1}{p}}\le\left(\sum_{i=1}^{n}a_{i}^{p}\right)^{\frac{1}{p}}+\left(\sum_{i=1}^{n}b_{i}^{p}\right)^{\frac{1}{p}}
\end{equation}
当且仅当 $\frac{a_{1}}{b_{1}}=\frac{a_{2}}{b_{2}}=\dots=\frac{a_{n}}{b_{n}}$ 时等号成立。
\end{thm}
\newpage

\section{例题}
% === 例题部分 ===


% 例 1
\begin{example}{樊畿不等式}{}
设 $a_{1},a_{2},\dots,a_{n}\in(0,\frac{1}{2}]$,求证:
\begin{equation}
    \frac{(\prod_{i=1}^{n}a_{i})^{\frac{1}{n}}}{\frac{1}{n}\sum_{i=1}^{n}a_{i}} \le \frac{(\prod_{i=1}^{n}(1-a_{i}))^{\frac{1}{n}}}{\frac{1}{n}\sum_{i=1}^{n}(1-a_{i})}.
\end{equation}
\end{example}
\newpage

% 例 2
\begin{example}{}{}
设 $n$ 是正整数,$x_{1},x_{2},\dots,x_{n+1},p,q$ 是正实数,满足 $p<q$,且 $x_{n+1}^{p}> x_{1}^{p}+x_{2}^{p}+\dots+x_{n}^{p}$。求证:
\begin{enumerate}
    \item $x_{n+1}^{q}>x_{1}^{q}+x_{2}^{q}+\dots+x_{n}^{q}$
    \item $(x_{n+1}^{p}-\sum_{i=1}^{n}x_{i}^{p})^{\frac{1}{p}}<(x_{n+1}^{q}-\sum_{i=1}^{n}x_{i}^{q})^{\frac{1}{q}}.$
\end{enumerate}
\end{example}
\newpage

% 例 3
\begin{example}{}{}
设整数 $n\ge2$,正实数 $a_{1},a_{2},\dots,a_{n}$ 满足 $\sum_{k=1}^{n}\frac{1}{a_{k}}=1$。求证:
\begin{equation}
    \sum_{k=1}^{n}\frac{a_{k}^{a_{k}-1}}{(a_{k}^{a_{k}}-1)^{n}}\ge\frac{a_{1}a_{2}\dots a_{n}}{(a_{1}a_{2}\dots a_{n}-1)^{n}}.
\end{equation}
\end{example}
\newpage

% === 作业题部分 ===
\section{作业题}
% 作业题 1
\begin{homework}{}{}
设 $a_{1},a_{2},\dots,a_{n}$ 是不小于 1 的实数,求证:
\begin{equation}
    \frac{1}{a_{1}+1}+\frac{1}{a_{2}+1}+\dots+\frac{1}{a_{n}+1}\ge\frac{n}{\sqrt[n]{a_{1}a_{2}\dots a_{n}}+1}.
\end{equation}
\end{homework}
\newpage

% 作业题 2
\begin{homework}{}{}
设正实数 $a_{1},a_{2},\dots,a_{n}$ 满足 $\sum_{i=1}^{n}a_{i}=1$。求证:
\begin{equation}
    \prod_{i=1}^{n}\frac{1+a_{i}}{a_{i}}\ge\prod_{i=1}^{n}\frac{n-a_{i}}{1-a_{i}}.
\end{equation}
\end{homework}
\newpage

% 作业题 3
\begin{homework}{}{}
设 $m,n$ 是给定的大于 1 的整数,$\alpha<\beta$ 是给定的正实数,对不全为 0 的非负实数 $a_{ij}\ (1\le i\le n,\ 1\le j\le m)$,求
\begin{equation}
    \frac{\left[\sum_{j=1}^{m}\left(\sum_{i=1}^{n}a_{ij}^{\alpha}\right)^{\frac{\beta}{\alpha}}\right]^{\frac{1}{\beta}}}{\left[\sum_{i=1}^{n}\left(\sum_{j=1}^{m}a_{ij}^{\beta}\right)^{\frac{\alpha}{\beta}}\right]^{\frac{1}{\alpha}}}
\end{equation}
的最大值。
\end{homework}
\newpage

% 作业题 4
\begin{homework}{}{}
给定正整数 $n$。求最大的实数 $\lambda$,使得对所有满足
\begin{equation}
    \frac{1}{2n}\sum_{i=1}^{2n}(x_{i}+2)^{n}\ge\prod_{i=1}^{2n}x_{i}
\end{equation}
的正实数 $x_{1},x_{2},\dots,x_{2n}$,都有
\begin{equation}
    \frac{1}{2n}\sum_{i=1}^{2n}(x_{i}+1)^{n}\ge\lambda\prod_{i=1}^{2n}x_{i}.
\end{equation}
\end{homework}
\newpage
% chapters/13_reverse_ineq.tex

\chapter{反向不等式}
\section{基础知识}
\begin{thm}{Kantorović 不等式}{}
设整数 $n\ge2$,正实数 $\lambda_{1},\lambda_{2},\dots,\lambda_{n}$ 满足 $\sum_{i=1}^{n}\lambda_{i}=1$。设正实数 $m<M$,实数 $a_{1},a_{2},\dots,a_{n}\in[m,M]$,则
\begin{equation}
    \left(\sum_{i=1}^{n}\lambda_{i}a_{i}\right)\left(\sum_{i=1}^{n}\frac{\lambda_{i}}{a_{i}}\right)\le\frac{(M+m)^{2}}{4Mm}.
\end{equation}
\end{thm}
\newpage


\begin{thm}{Polyá-Szegö 不等式}{}
设整数 $n\ge2$,正实数 $a<A,\ b<B$,实数 $a_{1},a_{2},\dots,a_{n}\in[a,A],\ b_{1},b_{2},\dots,b_{n}\in[b,B]$。则
\begin{equation}
    \left(\sum_{i=1}^{n}a_{i}^{2}\right)\left(\sum_{i=1}^{n}b_{i}^{2}\right)\le\frac{1}{4}\left(\sqrt{\frac{AB}{ab}}+\sqrt{\frac{ab}{AB}}\right)^{2}\left(\sum_{i=1}^{n}a_{i}b_{i}\right)^{2}.
\end{equation}
\end{thm}
\newpage


% \section{预习题}


\section{例题}
% example 69
\begin{example}{1978 苏联}{}
设 $x_1, x_2, \dots, x_n \in [a, b]$,其中 $0 < a < b$,证明:
\begin{equation}
    (x_1 + x_2 + \dots + x_n) \cdot \left( \frac{1}{x_1} + \frac{1}{x_2} + \dots + \frac{1}{x_n} \right) \le \frac{(a+b)^2}{4ab} \cdot n^2
\end{equation}
\end{example}
\newpage


% example 67
\begin{example}{2016 克罗地亚}{}
已知 $x_1, x_2, \dots, x_n$ 均为非负实数,证明:
\begin{equation}
    \left( x_1 + \frac{x_2}{2} + \dots + \frac{x_n}{n} \right) \cdot \left( x_1 + 2x_2 + \dots + nx_n \right) \le \frac{(n+1)^2}{4n} \cdot (x_1 + x_2 + \dots + x_n)^2
\end{equation}
\end{example}
\newpage

% 作业题 3
\begin{example}{}{}
设整数 $n\ge2$,正实数 $a_{1},a_{2},\dots,a_{n}$ 满足
\begin{equation}
    (a_{1}+a_{2}+\dots+a_{n})\left(\frac{1}{a_{1}}+\frac{1}{a_{2}}+\dots+\frac{1}{a_{n}}\right)\le\left(n+\frac{1}{2}\right)^{2}.
\end{equation}
求证:$\max\{a_{1},\dots,a_{n}\}\le4 \min\{a_{1},\dots,a_{n}\}$。
\end{example}
\newpage

 
% 例 3
\begin{example}{}{}
给定整数 $n\ge2$。设实数 $a_{1},a_{2},\dots,a_{n}\in[1,2]$,求
\begin{equation}
    \left(\sum_{i=1}^{n}a_{i}\right)\left(\sum_{i=1}^{n}\frac{1}{a_{i}}\right)^{2}
\end{equation}
的最大值。
\end{example}
\newpage

% 例 4
\begin{example}{1998年高中联赛}{}
设整数 $n\ge2$,实数 $a_{1},a_{2},\dots,a_{n},b_{1},b_{2},\dots,b_{n}\in[1,2]$ 且满足 $\sum_{i=1}^{n}a_{i}^{2}=\sum_{i=1}^{n}b_{i}^{2}$。
求证:
\begin{equation}
    \sum_{i=1}^{n}\frac{a_{i}^{3}}{b_{i}}\le\frac{17}{10}\sum_{i=1}^{n}a_{i}^{2}.
\end{equation}
\end{example}
\newpage

% 例 5
\begin{example}{}{}
设整数 $n\ge3$,正实数 $a_{1},a_{2},\dots,a_{n}$ 满足
\begin{equation}
    (a_{1}+a_{2}+\dots+a_{n})\left(\frac{1}{a_{1}}+\frac{1}{a_{2}}+\dots+\frac{1}{a_{n}}\right)<n^{2}+1.
\end{equation}
求证:$a_{1},a_{2},\dots,a_{n}$ 中任意三个数均能构成三角形的三边长。
\end{example}
\newpage






% 例 8
\begin{example}{}{}
设整数 $n\ge2$,正实数 $a<b$,实数 $a_{1},a_{2},\dots,a_{n}\in[a,b]$。求证:
\begin{equation}
    \frac{a_{1}+a_{2}+\dots+a_{n}}{n\sqrt[n]{a_{1}a_{2}\dots a_{n}}}\le\left(\frac{M}{2}\right)^{2-\frac{2}{n}},
\end{equation}
其中 $M=\sqrt{\frac{a}{b}}+\sqrt{\frac{b}{a}}$。
\end{example}
\newpage

% 例 9
\begin{example}{}{}
给定整数 $n\ge2$。求最小的实数 $\lambda$,使得对任意正实数 $a_{1},a_{2},\dots,a_{n}$ 都有
\begin{equation}
    \sqrt[n]{\prod_{i=1}^{n}a_{i}}+\lambda\sum_{1\le i<j\le n}|a_{i}-a_{j}|\ge\frac{1}{n}\sum_{i=1}^{n}a_{i}.
\end{equation}
\end{example}
\newpage



% 例 10
\begin{example}{}{}
给定整数 $n\ge2$。求最小的实数 $\lambda$,使得对任意实数 $a_{1},a_{2},\dots,a_{n}$,都有
\begin{equation}
    \left|\sum_{i=1}^{n}a_{i}\right|+\lambda\sum_{1\le i<j\le n}|a_{i}-a_{j}|\ge\sum_{i=1}^{n}|a_{i}|.
\end{equation}
\end{example}



\newpage 
\section{练习题}



% example 68
\begin{problem}{}{}
已知非负实数 $a_1, a_2, \dots, a_n$ 满足 $a_1 + a_2 + \dots + a_n = 1$,其中 $n \ge 3$,求 
\begin{equation}
    \sum_{i=1}^n i a_i \cdot (\sum_{i=1}^n \frac{a_i}{i})^2
\end{equation}
的最大值。
\end{problem}



% example 70
\begin{problem}{}{}
已知非负实数 $x_1, x_2, \dots, x_n$ 满足 $\sum_{i=1}^n x_i = 1$,证明:
\begin{equation}
    1 \le \sum_{i=1}^n (2i-1)x_i \cdot \sum_{i=1}^n \frac{x_i}{2i-1} \le \frac{n^2}{2n-1}
\end{equation}
\end{problem}

\begin{problem}{}{}
设整数 $n\ge2$,正实数 $a_{1},a_{2},\dots,a_{n}$ 满足 $a_{1}\le a_{2}\le\dots\le a_{n}$,且 $a_{1}\ge\frac{a_{2}}{2}\ge\dots\ge\frac{a_{n}}{n}$。
求证:
\begin{equation}
    \frac{a_{1}+a_{2}+\dots+a_{n}}{n\sqrt[n]{a_{1}a_{2}\dots a_{n}}}\le\frac{n+1}{2\sqrt[n]{n!}}.
\end{equation}
\end{problem}
% 作业题 1
\begin{problem}{}{}
设整数 $n\ge2$,正实数 $\lambda_{1},\lambda_{2},\dots,\lambda_{n}$ 满足 $\sum_{i=1}^{n}\lambda_{i}=1$。设正实数 $m<M$,实数 $a_{1},a_{2},\dots,a_{n}\in[m,M]$。求证:
\begin{equation}
    \sum_{i=1}^{n}\lambda_{i}a_{i}-\frac{1}{\sum_{i=1}^{n}\frac{\lambda_{i}}{a_{i}}}\le(\sqrt{M}-\sqrt{m})^{2}.
\end{equation}
\end{problem}
% 例 7


% 作业题 2
\begin{problem}{}{}
设整数 $n\ge2$,正实数 $m<M$,实数 $a_{1},a_{2},\dots,a_{n}\in[m,M]$。求证:
\begin{equation}
    \sum_{i=1}^{n}\frac{a_{i}^{2}}{a_{i+1}}\le\frac{M^{2}-Mm+m^{2}}{Mm}\sum_{i=1}^{n}a_{i},
\end{equation}
其中 $a_{n+1}=a_{1}$。
\end{problem}

% 例 6
\begin{problem}{}{}
设整数 $n\ge3$,正实数 $a_{1},a_{2},\dots,a_{n}$ 满足
\begin{equation}
    (n-1)(a_{1}^{4}+a_{2}^{4}+\dots+a_{n}^{4})<(a_{1}^{2}+a_{2}^{2}+\dots+a_{n}^{2})^{2}.
\end{equation}
求证:$a_{1},a_{2},\dots,a_{n}$ 中任意三个数均能构成三角形的三边长。
\end{problem}

% 作业题 4
\begin{problem}{}{}
给定整数 $n\ge3$。求最大的实数 $\lambda$,使得只要正实数 $a_{1},a_{2},\dots,a_{n}$ 满足
\begin{equation}
    a_{1}^{2}+a_{2}^{2}+\dots+a_{n}^{2}<\lambda(a_{1}+a_{2}+\dots+a_{n})^{2},
\end{equation}
那么 $a_{1},a_{2},\dots,a_{n}$ 中任意三个数便能构成三角形的三边长。
\end{problem}


% chapters/16_convex_func.tex

\chapter{凸函数与不等式}

本讲介绍凸函数的两个应用,一是用于处理变量有界的情形;二是卡拉玛特 (Karamata) 不等式,这两个都是强有力的工具,有了它们会对之前的很多问题有新的理解。

\section{重要定理}

\begin{thm}{优超关系 (Majorization)}{majorization_def}
设 $x_{1},x_{2},\dots,x_{n}$ 和 $y_{1},y_{2},\dots,y_{n}$ 是实数,如果满足:
\begin{enumerate}
    \item $x_{1}\ge x_{2}\ge\dots\ge x_{n}$ 且 $y_{1}\ge y_{2}\ge\dots\ge y_{n}$;
    \item 对任意 $1\le k\le n-1$,有 $\sum_{i=1}^k x_i \ge \sum_{i=1}^k y_i$;
    \item $\sum_{i=1}^n x_i = \sum_{i=1}^n y_i$.
\end{enumerate}
则称 $(x_{1},x_{2},\dots,x_{n})$ 优超于 $(y_{1},y_{2},\dots,y_{n})$,记作 $(x)\succ(y)$。
\end{thm}


\newpage 
\begin{thm}{卡拉玛特不等式 (Karamata's Inequality)}{karamata}
设 $(x_{1},x_{2},\dots,x_{n}) \succ (y_{1},y_{2},\dots,y_{n})$。
\begin{enumerate}
    \item 若 $f(x)$ 是区间 $I$ 上的**下凸函数** (Convex),则
    \begin{equation}
        \sum_{i=1}^{n}f(x_{i})\ge\sum_{i=1}^{n}f(y_{i})
    \end{equation}
    \item 若 $f(x)$ 是区间 $I$ 上的**上凸函数** (Concave),则
    \begin{equation}
        \sum_{i=1}^{n}f(x_{i})\le\sum_{i=1}^{n}f(y_{i})
    \end{equation}
\end{enumerate}
\end{thm}

\newpage 
\begin{thm}{Popoviciu 不等式}{popoviciu}
设 $f(x)$ 是 $I$ 上的下凸函数,则对任意实数 $a_{1},a_{2},\dots,a_{n} \in I$,都有
\begin{equation}
    \sum_{i=1}^{n}f(a_{i}) + n(n-2)f(a) \ge (n-1)\sum_{i=1}^{n}f(b_{i})
\end{equation}
其中 $a = \frac{1}{n}\sum_{i=1}^n a_i$,$b_{i} = \frac{1}{n-1}\sum_{j\ne i}a_{j}$。
\end{thm}



% === 例题部分 ===
\newpage 
\section{例题}
% 例 1
\begin{example}{}{}
给定整数 $n\ge3$,设 $a_{1},a_{2},\dots,a_{n}\in[0,1]$,求 $\sum_{i=1}^{n}a_{i}-\sum_{i=1}^{n}a_{i}a_{i+1}$ 的最值,其中 $a_{n+1}=a_{1}$。
\end{example}
\newpage

% 例 2
\begin{example}{}{}
设整数 $n\ge2$,正实数 $m<M$,实数 $a_{1},a_{2},\dots,a_{n}\in[m,M]$。求证:
\begin{equation}
    \left(\sum_{i=1}^{n}a_{i}\right)\left(\sum_{i=1}^{n}\frac{1}{a_{i}}\right)\le n^{2}+\left[\frac{n^{2}}{4}\right]\frac{(M-m)^{2}}{Mm}.
\end{equation}
\end{example}
\newpage

% 例 3
\begin{example}{}{}
设整数 $n\ge2$,$a_{1},a_{2},\dots,a_{n}\in[0,1]$。记 $S=\sum_{i=1}^{n}a_{i}$,求证:
\begin{equation}
    \sum_{i=1}^{n}\frac{a_{i}}{1+S-a_{i}}+\prod_{i=1}^{n}(1-a_{i})\le1.
\end{equation}
\end{example}
\newpage

% 例 4
\begin{example}{}{}
给定整数 $n>2$。设非负实数 $x_{1},x_{2},\dots,x_{n}$ 满足 $\sum_{i=1}^{n}x_{i}=1$,求
\begin{equation}
    x_{1}^{2}+x_{2}^{2}+\dots+x_{n}^{2}+2\sqrt{x_{1}x_{2}\dots x_{n}}
\end{equation}
的最大值和最小值。
\end{example}
\newpage

% 例 5
\begin{example}{}{}
设整数 $n\ge2$,$a_{1},a_{2},\dots,a_{n},b_{1},b_{2},\dots,b_{n}$ 是非负实数,求证:
\begin{equation}
    \left(\frac{n}{n-1}\right)^{n-1}\left(\frac{1}{n}\sum_{i=1}^{n}a_{i}^{2}\right)+\left(\frac{1}{n}\sum_{i=1}^{n}b_{i}\right)^{2}\ge\prod_{i=1}^{n}(a_{i}^{2}+b_{i}^{2})^{\frac{1}{n}}.
\end{equation}
\end{example}
\newpage

% 例 6
\begin{example}{}{}
设实数 $a_{1},a_{2},\dots,a_{1997}\in[-\frac{1}{\sqrt{3}},\sqrt{3}]$,且满足 $\sum_{i=1}^{1997}a_{i}=-318\sqrt{3}$。求 $\sum_{i=1}^{1997}a_{i}^{12}$ 的最大值。
\end{example}
\newpage

% 例 7
\begin{example}{}{}
设整数 $n\ge2$,$a_{1},a_{2},\dots,a_{n}\in(n-1,n)$。记 $S=\sum_{i=1}^{n}a_{i}$,求证:
\begin{equation}
    \prod_{i=1}^{n}a_{i}\ge\prod_{i=1}^{n}(S-(n-1)a_{i}).
\end{equation}
\end{example}
\newpage

% 例 8
\begin{example}{}{}
设整数 $n\ge3$,正实数 $a_{1}\ge a_{2}\ge\dots\ge a_{n}$。求证:
\begin{equation}
    \sum_{i=1}^{n}\sqrt{2a_{i}+a_{i+1}}\le\sum_{i=1}^{n}\sqrt{a_{i}+a_{i+1}+a_{i+2}},
\end{equation}
其中 $a_{n+1}=a_{1},\ a_{n+2}=a_{2}$。
\end{example}




% === 作业题部分 ===
\newpage
\section{作业题}
% 作业题 1
\begin{homework}{Ozeki 不等式}{}
设 $a_{1},a_{2},\dots,a_{n}$ 和 $b_{1},b_{2},\dots,b_{n}$ 满足 $0\le m_{1}\le a_{i}\le M_{1},\ 0\le m_{2}\le b_{i}\le M_{2}$。求证:
\begin{equation}
    \left(\sum_{i=1}^{n}a_{i}^{2}\right)\left(\sum_{i=1}^{n}b_{i}^{2}\right)-\left(\sum_{i=1}^{n}a_{i}b_{i}\right)^{2}\le\left[\frac{n^{2}}{3}\right](M_{1}M_{2}-m_{1}m_{2})^{2}.
\end{equation}
\end{homework}
\newpage
% 作业题 2
\begin{homework}{}{}
给定整数 $n\ge2$。设实数 $0\le a_{1}\le a_{2}\le\dots\le a_{n}\le1$,求
\begin{equation}
    \sum_{1\le i<j\le n}(a_{j}-a_{i}+1)^{2}+4\sum_{i=1}^{n}a_{i}^{2}
\end{equation}
的最大值。
\end{homework}
\newpage
% 作业题 3
\begin{homework}{}{}
给定正整数 $n$ 和正实数 $a$。设 $k, x_{1},x_{2},\dots,x_{k}$ 是正整数,且满足 $x_{1}+x_{2}+\dots+x_{k}=n$。求 $a^{x_{1}}+a^{x_{2}}+\dots+a^{x_{k}}$ 的最大值。
\end{homework}
\newpage
% 作业题 4
\begin{homework}{}{}
设整数 $n\ge3$,正实数 $a_{1}\ge a_{2}\ge\dots\ge a_{n}$。求证:
\begin{equation}
    \prod_{i=1}^{n}\frac{a_{i}+a_{i+1}}{2}\le\prod_{i=1}^{n}\frac{a_{i}+a_{i+1}+a_{i+2}}{3},
\end{equation}
其中 $a_{n+1}=a_{1},\ a_{n+2}=a_{2}$。
\end{homework}
\newpage

\part{常见问题与方法}
% chapters/09_abs_ineq_1.tex

\chapter{绝对值不等式(一)}
本讲介绍含绝对值的不等式,最常用的方法是三角不等式和正负分离。


% \section{预习题}
\section{例题}
% 例 1
\begin{example}{}{}
设整数 $n\ge2$,$a_{1},a_{2},\dots,a_{n},b_{1},b_{2},\dots,b_{n}$ 是实数,求证:存在 $1\le k\le n$,使得
\begin{equation}
    \sum_{i=1}^{n}|a_{i}-a_{k}|\le\sum_{i=1}^{n}|b_{i}-a_{k}|.
\end{equation}
\end{example}
\newpage

% 例 2
\begin{example}{}{}
设整数 $n\ge2$,实数 $a_{1},a_{2},\dots,a_{n}$ 满足 $\sum_{i=1}^{n-1}|a_{i}-a_{i+1}|=1$。对 $1\le k\le n$,记 $A_{k}=\frac{1}{k}\sum_{i=1}^{k}a_{i}$,求证:
\begin{equation}
    \sum_{i=1}^{n-1}|A_{i}-A_{i+1}|\le1-\frac{1}{n}.
\end{equation}
\end{example}
\newpage

% 例 3
\begin{example}{}{}
设整数 $n\ge2$,$a_{0},a_{1},\dots,a_{n}$ 是实数,满足 $a_{1}=a_{n-1}=0$。求证:对任意实数 $t$,
\begin{equation}
    |a_{0}|-|a_{n}|\le\sum_{i=0}^{n-2}|a_{i}-ta_{i+1}-a_{i+2}|.
\end{equation}
\end{example}
\newpage

% 例 4
\begin{example}{}{}
给定整数 $n\ge2$。设实数 $a_{1},a_{2},\dots,a_{n}$ 满足 $\sum_{i=1}^{n}a_{i}=0$ 且 $\sum_{i=1}^{n}a_{i}^{2}=1$。求:
\begin{enumerate}
    \item $\sum_{i=1}^{n}|a_{i}|$ 的最小值和最大值;
    \item $\max_{1\le i\le n}|a_{i}|$ 的最小值和最大值。
\end{enumerate}
\end{example}
\newpage

% 例 5
\begin{example}{}{}
给定整数 $n\ge2$。设非负实数 $a_{1}\le a_{2}\le\dots\le a_{n},\ b_{1}\le b_{2}\le\dots\le b_{n}$ 满足 $\sum_{i=1}^{n}a_{i}=\sum_{i=1}^{n}b_{i}=1$。求:
\begin{enumerate}
    \item $\min_{1\le i\le n}|a_{i}-b_{i}|$ 的最大值;
    \item $\sum_{i=1}^{n}|a_{i}-b_{i}|$ 的最大值。
\end{enumerate}
\end{example}
\newpage

% 例 6
\begin{example}{}{}
给定整数 $n \ge 2$。设实数 $-1\le a_{1}\le a_{2}\le\dots\le a_{n}\le1,\ -1\le b_{1}\le b_{2}\le\dots\le b_{n}\le1$ 满足 $\sum_{i=1}^{n}a_{i}=\sum_{i=1}^{n}b_{i}$。求 $\sum_{i=1}^{n}|a_{i}-b_{i}|$ 的最大值。
\end{example}
\newpage

% 例 7
\begin{example}{}{}
设整数 $n\ge3$,非零实数 $x_1, x_2, \dots, x_n$ 满足 $\frac{x_{1}}{x_{2}}+\frac{x_{2}}{x_{3}}+\dots+\frac{x_{n}}{x_{1}}=0$。求证:
\begin{equation}
    |x_{1}x_{2}+x_{2}x_{3}+\dots+x_{n}x_{1}|\le\sum_{i=1}^{n}|x_{i}|\cdot(\max_{1\le i\le n}|x_{i}|-\min_{1\le i\le n}|x_{i}|).
\end{equation}
\end{example}
\newpage

% 例 8
\begin{example}{}{}
设实数 $a<b,\ \lambda_{1},\lambda_{2},\dots,\lambda_{n}\in[a,b]$。设实数 $x_{1},x_{2},\dots,x_{n},y_{1},y_{2},\dots,y_{n}$ 满足 $\sum_{i=1}^{n}x_{i}^{2}=\sum_{i=1}^{n}y_{i}^{2}=1$。求证:
\begin{equation}
    \left|\sum_{i=1}^{n}\lambda_{i}(x_{i}^{2}-y_{i}^{2})\right|\le(b-a)\sqrt{1-(\sum_{i=1}^{n}x_{i}y_{i})^{2}}.
\end{equation}
\end{example}



\newpage
\section{作业题}
% 作业题 1
\begin{homework}{}{}
设整数 $n\ge2$,非零实数 $a_{1},a_{2},\dots,a_{n}$ 满足 $a_{1}+a_{2}+\dots+a_{n}=0$。求证:存在 $1\le i<j\le n$,使得
\begin{equation}
    \frac{1}{2}\le\left|\frac{a_{i}}{a_{j}}\right|\le2.
\end{equation}
\end{homework}

\newpage
% 作业题 2
\begin{homework}{}{}
设整数 $n\ge3$,正实数 $a_{1},a_{2},\dots,a_{n}$ 满足 $a_{i}\le1\ (i=1,2,\dots,n)$。对 $1\le k\le n$,记 $A_{k}=\frac{1}{k}\sum_{i=1}^{k}a_{i}$。求证:
\begin{equation}
    \left|\sum_{i=1}^{n}a_{i}-\sum_{i=1}^{n}A_{i}\right|<\frac{n-1}{2}.
\end{equation}
\end{homework}

\newpage
% 作业题 3
\begin{homework}{}{}
给定整数 $n\ge2$。求最大的实数 $\lambda$,使得对任意和为 0 的实数 $a_{1},a_{2},\dots,a_{n}$,都有
\begin{equation}
    \sum_{i=1}^{n}a_{i}^{2}+1\ge\lambda\sum_{i=1}^{n}|a_{i}|.
\end{equation}
\end{homework}


\newpage
% 作业题 4
\begin{homework}{}{}
设正实数 $a_{1},a_{2},\dots,a_{n},b_{1},b_{2},\dots,b_{n}$ 满足 $\sum_{i=1}^{n}a_{i}=\sum_{i=1}^{n}b_{i}=1$。求证:
\begin{equation}
    \sum_{i=1}^{n}|a_{i}-b_{i}|\le2-\min_{1\le i\le n}\frac{a_{i}}{b_{i}}-\min_{1\le i\le n}\frac{b_{i}}{a_{i}}.
\end{equation}
\end{homework}
\chapter{绝对值不等式 (二)}

本讲继续介绍含绝对值的不等式,包括设序和离散介值原理等方法,以及几个综合性的问题。
\section{例题}
% 例 1
\begin{example}{}{}
给定整数 $n\ge2$。设实数 $a_{1},a_{2},\dots,a_{n}\in[0,1]$,求
\begin{equation}
    \sum_{1\le i<j\le n}|a_{i}-a_{j}|
\end{equation}
的最大值。
\end{example}
\newpage



% 例 2
\begin{example}{}{}
给定整数 $n\ge2$。设实数 $a_{1},a_{2},\dots,a_{n}\in[-1,1]$,求
\begin{equation}
    \left|a_{1}-\frac{a_{2}+a_{3}+\dots+a_{n}}{n}\right| + \left|a_{2}-\frac{a_{1}+a_{3}+\dots+a_{n}}{n}\right| + \dots + \left|a_{n}-\frac{a_{1}+a_{2}+\dots+a_{n-1}}{n}\right|
\end{equation}
的最大值。
\end{example}
\newpage



% 例 3
\begin{example}{}{}
给定整数 $n\ge2$。设实数 $a_{1},a_{2},\dots,a_{n}$ 满足:
\begin{enumerate}
    \item $\sum_{i=1}^{n}a_{i}=0$;
    \item $|a_{i}|\le1,\ i=1,2,\dots,n$.
\end{enumerate}
求 $\min_{1\le i\le n-1}|a_{i}-a_{i+1}|$ 的最大值。
\end{example}
\newpage



% 例 4
\begin{example}{}{}
设整数 $n\ge3$,实数 $a_{1},a_{2},\dots,a_{n}$ 满足 $\sum_{i=1}^{n}a_{i}>1,\ |a_{i}|\le1\ (i=1,2,\dots,n)$。求证:存在正整数 $k<n$,使得
\begin{equation}
    \left|\sum_{i=1}^{k}a_{i}-\sum_{i=k+1}^{n}a_{i}\right|\le1.
\end{equation}
\end{example}
\newpage


% 例 5
\begin{example}{}{}
设实数 $a_{1},a_{2},\dots,a_{40}$ 满足 $\sum_{i=1}^{40}a_{i}=0$ 且对 $1\le i\le40$,都有 $|a_{i}-a_{i+1}|\le1$,这里 $a_{41}=a_{1}$。记 $a=a_{10},\ b=a_{20},\ c=a_{30},\ d=a_{40}$。
\begin{enumerate}
    \item 求 $a+b+c+d$ 的最大值;
    \item 求 $ab+cd$ 的最大值。
\end{enumerate}
\end{example}
\newpage

% 例 6
\begin{example}{}{}
设实数 $a_1, a_2, \dots, a_{1001}$ 满足 $a_{1}=a_{1001},\ |a_{i}+a_{i+2}-2a_{i+1}|\le1\ (i=1,2,\dots,999)$。求
\begin{equation}
    \max_{1\le i<j\le 1001}|a_{i}-a_{j}|
\end{equation}
的最大值。
\end{example}
\newpage

\section{作业题}
% 作业题 1
\begin{homework}{}{}
给定整数 $n\ge2$。设实数 $a_{1},a_{2},\dots,a_{n}$ 满足:
\begin{enumerate}
    \item $\sum_{i=1}^{n}a_{i}=0$;
    \item $\max_{1\le i\le n}|a_{i}|=1$.
\end{enumerate}
求 $\max_{1\le i\le n}|a_{i}-a_{i+1}|$ 的最小值,其中 $a_{n+1}=a_{1}$。
\end{homework}

\newpage
% 作业题 2
\begin{homework}{}{}
设实数 $x_{1},x_{2},\dots,x_{n}$ 满足 $|x_{1}+x_{2}+\dots+x_{n}|=1$,且 $|x_{i}|\le\frac{n+1}{2}\ (1\le i\le n)$。求证:存在 $x_{1},x_{2},\dots,x_{n}$ 的一个排列 $y_{1},y_{2},\dots,y_{n}$,使得
\begin{equation}
    |y_{1}+2y_{2}+\dots+ny_{n}|\le\frac{n+1}{2}.
\end{equation}
\end{homework}

\newpage
% 作业题 3
\begin{homework}{}{}
设实数 $a_1, a_2, \dots, a_{2018}$ 满足 $|a_{i+1}-a_{i}|\le1\ (1\le i\le 2018)$,其中 $a_{2019}=a_{1}$。求
\begin{equation}
    \sum_{i=1}^{2018}|a_{i}|-\left|\sum_{i=1}^{2018}a_{i}\right|
\end{equation}
的最大值。
\end{homework}
\newpage






\section{其他练习题}
\begin{problem}{}{}
已知 \( n \) 个实数 \( x_1, x_2, \dots, x_n \) 的算术平均值为 \( a \),证明:
\[
\sum_{k=1}^{n} (x_k - a)^2 \leq \frac{1}{2} \cdot \left( \sum_{k=1}^{n} |x_k - a| \right)^2.
\]
\end{problem}

\begin{problem}{}{}
设 \( a_0 = 0 \),\( a_1, a_2, \dots, a_n \in \mathbb{R} \),证明:
\[
\sum_{k=1}^{n} |a_k(a_k - a_{k-1})| \leq \frac{n+1}{2} \cdot \sum_{k=1}^{n} (a_k - a_{k-1})^2.
\]
\end{problem}

\begin{problem}{}{}
设整数 \( n \geq 3 \),实数 \( a_1, a_2, \dots, a_n \) 均大于1,且 \( |a_{k+1} - a_k| < 1 \ (1 \leq k \leq n-1) \),证明:
\[
\frac{a_1}{a_2} + \frac{a_2}{a_3} + \dots + \frac{a_{n-1}}{a_n} + \frac{a_n}{a_1} < 2n - 1.
\]
\end{problem}

\begin{problem}{}{}
(18浙江预赛)将 \( 2n \ (n \geq 2) \) 个不同的整数分成两组 \( a_1, a_2, \dots, a_n, b_1, b_2, \dots, b_n \),证明:
\[
\sum_{\substack{1 \leq i \leq n \\ 1 \leq j \leq n}} |a_i - b_j| - \sum_{1 \leq i < j \leq n} \left( |a_j - a_i| + |b_j - b_i| \right) \geq n.
\]
\end{problem}

\begin{problem}{}{}
已知非负实数 \( x_1, x_2, \dots, x_n \) 均不超过1,证明:
\[
2 \cdot \sum_{i=1}^{n} \sum_{j=1}^{n} |x_i - x_j| \leq n^2.
\]
\end{problem}

\begin{problem}{}{}
已知实数 \( a_1, a_2, \dots, a_n \) 满足 \( a_1^2 + a_2^2 + \dots + a_n^2 = 1 \),求 \( |a_1 - a_2| + |a_2 - a_3| + \dots + |a_{n-1} - a_n| + |a_n - a_1| \) 的最大值。
\end{problem}

\begin{problem}{}{}
对每一个整数 \( n \geq 2 \),求最大的常数 \( c_n \),使得不等式
\[
c_n \cdot \sum_{i=1}^{n} |a_i| \leq \sum_{1 \leq i < j \leq n} |a_i - a_j|
\]
对任意满足 \( \sum_{i=1}^{n} a_i = 0 \) 的实数 \( a_1, a_2, \dots, a_n \) 成立。
\end{problem}

% chapters/17_mean_value_principle.tex

\chapter{平均值原理与不等式}

平均值原理是一种整体思想,本讲介绍两类能用平均值原理处理的问题,一类是存在性问题,一类与最大最小有关。

% === 例题部分 ===

% 例 1
\begin{example}{}{}
设非负实数 $x_{1},x_{2},\dots,x_{n}$ 满足 $x_{1}+x_{2}+\dots+x_{n}=1$。求证:存在 $x_{1},x_{2},\dots,x_{n}$ 的一个排列 $a_{1},a_{2},\dots,a_{n}$,使得
\begin{equation}
    a_{1}a_{2}+a_{2}a_{3}+\dots+a_{n}a_{1}\le\frac{1}{n}.
\end{equation}
\end{example}
\newpage

% 例 2
\begin{example}{}{}
设 $x_{1},x_{2},\dots,x_{n}$ 是实数,求证:存在 $x_{1},x_{2},\dots,x_{n}$ 的一个排列 $a_{1},a_{2},\dots,a_{n}$,使得
\begin{equation}
    \left|\sum_{i=1}^{n}ia_{i}\right|\ge\frac{n-1}{2}\max_{1\le i<j\le n}|x_{i}-x_{j}|.
\end{equation}
\end{example}
\newpage

% 例 3
\begin{example}{}{}
设实数 $x_{1},x_{2},\dots,x_{n}$ 满足 $x_{1}^{2}+x_{2}^{2}+\dots+x_{n}^{2}=1$,整数 $k\ge2$。求证:存在不全为 0 的整数 $a_{1},a_{2},\dots,a_{n}$,使得其中每一个的绝对值都不超过 $k-1$,且
\begin{equation}
    |a_{1}x_{1}+a_{2}x_{2}+\dots+a_{n}x_{n}|\le\frac{(k-1)\sqrt{n}}{k^{n}-1}.
\end{equation}
\end{example}
\newpage

% 例 4
\begin{example}{}{}
设实数 $a_{1},a_{2},\dots,a_{n}$ 满足 $0\le a_{1}<a_{2}<\dots<a_{n}\le1$。求证:存在 $x\in[0,1]$ 使得
\begin{equation}
    \sum_{i=1}^{n}\frac{1}{|x-a_{i}|}\le 8n\left(1+\frac{1}{3}+\dots+\frac{1}{2\lceil\frac{n}{2}\rceil-1}\right).
\end{equation}
\end{example}
\newpage

% 例 5
\begin{example}{}{}
设 $a_{1},a_{2},\dots,a_{n}$ 是小于 1 的正实数,$k$ 是正整数,求证:
\begin{equation}
    \min\{a_{1}(1-a_{2})^{k},a_{2}(1-a_{3})^{k},\dots,a_{n}(1-a_{1})^{k}\}\le\frac{k^{k}}{(k+1)^{k+1}}.
\end{equation}
\end{example}
\newpage

% 例 6
\begin{example}{}{}
给定正实数 $a,b$,整数 $n\ge2$。设函数 $f(x)=(x+a)(x+b)$,非负实数 $x_{1},x_{2},\dots,x_{n}$ 满足 $x_{1}+x_{2}+\dots+x_{n}=1$。求 $\sum_{1\le i<j\le n}\min\{f(x_{i}),f(x_{j})\}$ 的最大值。
\end{example}
\newpage

% 例 7
\begin{example}{}{}
设非负实数 $a_{1},a_{2},\dots,a_{9}$ 满足 $\sum_{i=1}^{9}a_{i}=1$。记
\begin{align}
    S &= \min\{a_{1},a_{2}\}+2\min\{a_{2},a_{3}\}+\dots+9\min\{a_{9},a_{1}\}, \\
    T &= \max\{a_{1},a_{2}\}+2\max\{a_{2},a_{3}\}+\dots+9\max\{a_{9},a_{1}\}.
\end{align}
当 $S$ 取最大值 $S_{0}$ 时,求 $T$ 的所有可能值。
\end{example}
\newpage

% 例 8
\begin{example}{}{}
设整数 $n\ge2$,$a_{1},a_{2},\dots,a_{n}$ 是正实数,求证:
\begin{equation}
    (\max_{1\le i\le n}a_{i})\left(\sum_{i=1}^{n}ia_{i}\right)\ge\frac{n+1}{n-1}\sum_{1\le i<j\le n}a_{i}a_{j}.
\end{equation}
\end{example}
\newpage

% 例 9
\begin{example}{}{}
给定整数 $n\ge2$。求最小的正实数 $\lambda$,使得对任意正实数 $a_{1},a_{2},\dots,a_{n}$,都有
\begin{equation}
    \sum_{i=1}^{n}\max\{a_{1},\dots,a_{i}\}\cdot \min\{a_{i},\dots,a_{n}\}\le\lambda\sum_{i=1}^{n}a_{i}^{2}.
\end{equation}
\end{example}




\newpage


% === 作业题部分 ===
\section{作业题}
% 作业题 1
\begin{homework}{}{}
设 $a_{1},a_{2},\dots,a_{n}$ 是实数,求证:存在实数 $x$,使得
\begin{equation}
    \{x-a_{1}\}+\{x-a_{2}\}+\dots+\{x-a_{n}\}\le\frac{n-1}{2},
\end{equation}
其中 $\{x\}$ 表示实数 $x$ 的小数部分。
\end{homework}


\newpage
% 作业题 2
\begin{homework}{}{}
设整数 $n\ge3$,$a_{1},a_{2},\dots,a_{n}$ 是实数,求证:存在 $\{1,2,\dots,n\}$ 的子集 $S$,满足对任意 $1\le i\le n-2$,有 $1\le|S\cap\{i,i+1,i+2\}|\le2$ 且
\begin{equation}
    \left|\sum_{i\in S}a_{i}\right|\ge\frac{1}{6}\sum_{i=1}^{n}|a_{i}|.
\end{equation}
\end{homework}

\newpage
% 作业题 3
\begin{homework}{}{}
给定整数 $n\ge4$。求最大的实数 $\lambda$,使得对任意满足 $a_{1}+a_{2}+\dots+a_{n}=0$ 的实数 $a_{1},a_{2},\dots,a_{n}$,都有
\begin{equation}
    \lambda \min\{a_{1},a_{2},\dots,a_{n}\}\ge \min\{a_{1},a_{2}\}+\min\{a_{2},a_{3}\}+\dots+\min\{a_{n},a_{1}\}.
\end{equation}
\end{homework}
\newpage
% 作业题 4
\begin{homework}{}{}
设 $a_{1},a_{2},\dots,a_{n}$ 是实数。对 $1\le i\le n$,定义
\begin{equation}
    d_{i}=\max_{1\le j\le i}a_{j}-\min_{i\le j\le n}a_{j}.
\end{equation}
令 $d=\max_{1\le i\le n}d_{i}$。
\begin{enumerate}
    \item 求证:对任意实数 $x_{1}\le x_{2}\le\dots\le x_{n}$,$\max_{1\le i\le n}|x_i - a_i| \ge \frac{d}{2}$.
    \item 求证:存在实数 $x_{1}\le x_{2}\le\dots\le x_{n}$ 使得 (1) 中等号成立。
\end{enumerate}
\end{homework}

% chapters/18_induction_1.tex

\chapter{归纳法与不等式 (一)}
凡是与正整数有关的命题都可以尝试用归纳法证明,在不等式中也是如此,本讲主要介绍第一数学归纳法的一些例子。

% === 例题部分 ===
\section{例题}
% 例 1
\begin{example}{}{}
设整数 $n\ge2$,$a_{1},a_{2},\dots,a_{n}\in[0,1]$。求证:
\begin{equation}
    \sum_{i=1}^{n}a_{i}-\sum_{1\le i<j\le n}a_{i}a_{j}\le1.
\end{equation}
\end{example}
\newpage

% 例 2
\begin{example}{}{}
设整数 $n\ge2$,实数 $a_{1},a_{2},\dots,a_{n}\in[0,1]$。求证:
\begin{equation}
    \prod_{i=1}^{n}(a_{i}^{2}-a_{i}a_{i+1}+1)\ge1,
\end{equation}
其中 $a_{n+1}=a_{1}$。
\end{example}
\newpage

% 例 3
\begin{example}{}{}
设整数 $n\ge2$,实数 $a_{1},a_{2},\dots,a_{n}\ge1$,且满足 $|a_{i}-a_{i+1}|\le1\ (1\le i\le n-1)$。求证:
\begin{equation}
    \sum_{i=1}^{n}\frac{a_{i}}{a_{i+1}}\le 2n-H_{n},
\end{equation}
其中 $a_{n+1}=a_{1}$,$H_{n}=1+\frac{1}{2}+\dots+\frac{1}{n}$。
\end{example}
\newpage

% 例 4
\begin{example}{}{}
设整数 $n\ge2$,正实数 $a_{1}\ge a_{2}\ge\dots\ge a_{n}$。求证:
\begin{equation}
    \sum_{i=1}^{n}\frac{a_{i}}{a_{i+1}}-n\le\frac{1}{2a_{1}a_{n}}\sum_{i=1}^{n}(a_{i}-a_{i+1})^{2},
\end{equation}
其中 $a_{n+1}=a_{1}$。
\end{example}
\newpage

% 例 5
\begin{example}{}{}
设整数 $n\ge2$,正实数 $a_{1}\le a_{2}\le\dots\le a_{n}$。求证:
\begin{equation}
    \sum_{1\le i<j\le n}(a_{i}+a_{j})^{2}\left(\frac{1}{i^{2}}+\frac{1}{j^{2}}\right)\ge4(n-1)\sum_{i=1}^{n}\frac{a_{i}^{2}}{i^{2}}.
\end{equation}
\end{example}
\newpage

% 例 6
\begin{example}{}{}
设整数 $n\ge3$,非负实数 $a_{1},a_{2},\dots,a_{n}$ 满足 $a_{1}+a_{2}+\dots+a_{n}=1$。求证:
\begin{equation}
    a_{1}^{2}a_{2}+a_{2}^{2}a_{3}+\dots+a_{n}^{2}a_{1}\le\frac{4}{27}.
\end{equation}
\end{example}
\newpage

% 例 7
\begin{example}{}{}
设正实数 $a_{1}\le a_{2}\le\dots\le a_{n}$ 满足 $a_{1}a_{2}\dots a_{n}=1$。求证:
\begin{equation}
    \sum_{i=1}^{n}\frac{1}{2^{i}(1+a_{i}^{2^{i}})}\ge\frac{1}{2}-\frac{1}{2^{n+1}}.
\end{equation}
\end{example}
\newpage

% 例 8
\begin{example}{}{}
设整数 $n\ge2$,正实数 $a_{1},a_{2},\dots,a_{n}$ 满足 $a_{i}a_{j}\le t^{|i-j|}$ 对任意 $1\le i,j\le n$ 成立,其中 $t\in(0,1)$。求证:
\begin{equation}
    a_{1}+a_{2}+\dots+a_{n}<\frac{1}{1-\sqrt{t}}.
\end{equation}
\end{example}



\newpage
\section{作业题}
% === 作业题部分 ===

% 作业题 1
\begin{homework}{}{}
设整数 $n\ge2$,$a_{1},a_{2},\dots,a_{n}\in[0,1]$。求证:
\begin{equation}
    \sum_{i=1}^{n}a_{i}-\sum_{i=1}^{n}a_{i}a_{i+1}\le\left[\frac{n}{2}\right],
\end{equation}
其中 $a_{n+1}=a_{1}$。
\end{homework}
\newpage

% 作业题 2
\begin{homework}{}{}
设整数 $n\ge2$,$a_{1},a_{2},\dots,a_{n}\in(0,1)$。求证:
\begin{equation}
    \frac{\sqrt{1-a_{1}}}{a_{1}}+\frac{\sqrt{1-a_{2}}}{a_{2}}+\dots+\frac{\sqrt{1-a_{n}}}{a_{n}}<\frac{\sqrt{n-1}}{a_{1}a_{2}\dots a_{n}}.
\end{equation}
\end{homework}
\newpage

% 作业题 3
\begin{homework}{}{}
设整数 $n\ge4$,正实数 $a_{1},a_{2},\dots,a_{n}$ 满足 $a_{1}a_{2}\dots a_{n}=1$。求证:
\begin{equation}
    \frac{1}{\sqrt{1+a_{1}}}+\frac{1}{\sqrt{1+a_{2}}}+\dots+\frac{1}{\sqrt{1+a_{n}}}<n-1.
\end{equation}
\end{homework}
\newpage

% 作业题 4
\begin{homework}{}{}
设整数 $n\ge2$,正实数 $a_{1},a_{2},\dots,a_{n}$ 满足 $a_{i}a_{j}\le 4^{-|i-j|}$ 对任意 $1\le i,j\le n$ 成立。求证:
\begin{equation}
    a_{1}+a_{2}+\dots+a_{n}<\frac{5}{3}.
\end{equation}
\end{homework}
\newpage


\chapter{归纳法与不等式 (二)}

本讲继续介绍用归纳法证明不等式,包括第二数学归纳法、加强数学归纳法、反向数学归纳法,以及几个经典的不等式。


\section{例题}
% === 例题部分 ===

% 例 1
\begin{example}{}{}
设实数 $a_{i},b_{i}\ (i=0,1,\dots,2n)$ 满足:
\begin{enumerate}
    \item 对 $i=0,1,\dots,2n-1$,有 $a_{i}+a_{i+1}\ge0$;
    \item 对 $j=0,1,\dots,n-1$,有 $a_{2j+1}\le0$;
    \item 对 $0\le p\le q\le n$,有 $\sum_{k=2p}^{2q}b_{k}>0$.
\end{enumerate}
求证:$\sum_{i=0}^{2n}(-1)^{i}a_{i}b_{i}\ge0$.
\end{example}
\newpage

% 例 2
\begin{example}{}{}
设 $a_{1},a_{2},\dots,a_{n}$ 是不全为零的非负实数,对 $1\le k\le n$,记
\begin{equation}
    m_{k}=\max_{1\le l\le k}\frac{a_{k-l+1}+a_{k-l+2}+\dots+a_{k}}{l}.
\end{equation}
求证:对任意正实数 $\alpha$,满足 $m_{k}>\alpha$ 的 $k$ 少于 $\frac{a_{1}+a_{2}+\dots+a_{n}}{\alpha}$ 个。
\end{example}
\newpage

% 例 3
\begin{example}{}{}
给定整数 $n\ge2$。设非负实数 $a_{1},a_{2},\dots,a_{n}$ 满足 $a_{1}\ge a_{2}\ge\dots\ge a_{n},\ a_{1}+a_{2}+\dots+a_{n}\le n$。求
\begin{equation}
    a_{1}+a_{1}a_{2}+a_{1}a_{2}a_{3}+\dots+a_{1}a_{2}\dots a_{n}
\end{equation}
的最小值。
\end{example}
\newpage

% 例 4
\begin{example}{}{}
设 $a_{1},a_{2},a_{3},\dots$ 是实数列,满足存在正整数 $N$,使得对任意 $n\ge N$ 都有 $a_{n}=1$。已知对任意整数 $n\ge2$ 都有 $a_{n}\le a_{n-1}+\frac{1}{2^{n}}a_{2n}$。
求证:对任意正整数 $k$,都有 $a_{k}>1-\frac{1}{2^{k}}$。
\end{example}
\newpage

% 例 5
\begin{example}{}{}
对实数列 $\{a_{n}\}$,定义数列 $\{b_{n}\}$ 如下:
\begin{equation}
    b_{1}=a_{1},\quad b_{n+1}=a_{n+1}-\left(\sum_{i=1}^{n}a_{i}^{2}\right)^{\frac{1}{2}},\quad n\ge1.
\end{equation}
求最小的正实数 $\lambda$,使得对任意实数列 $\{a_{n}\}$ 以及任意正整数 $n$,都有
\begin{equation}
    \frac{1}{n}\sum_{i=1}^{n}a_{i}^{2}\le\sum_{i=1}^{n}\lambda^{n-i}b_{i}^{2}.
\end{equation}
\end{example}
\newpage

% 例 6
\begin{example}{}{}
求最大的正实数 $\lambda$,使得对任意正整数 $n$ 以及任意正实数 $a_{1},a_{2},\dots,a_{n}$,都有
\begin{equation}
    1+\sum_{k=1}^{n}\frac{1}{a_{k}^{2}}\ge\lambda\sum_{k=1}^{n}\frac{1}{(1+\sum_{i=1}^{k}a_{i})^{2}}.
\end{equation}
\end{example}
\newpage

% 例 7
\begin{example}{牛顿不等式 (Newton's Inequality)}{}
设 $a_{1},a_{2},\dots,a_{n}$ 是实数,对 $1\le k\le n$ 记
\begin{equation}
    S_{k}=\frac{\sum_{1\le i_{1}<i_{2}<\dots<i_{k}\le n}a_{i_{1}}a_{i_{2}}\dots a_{i_{k}}}{C_{n}^{k}}.
\end{equation}
则 $S_{k-1}S_{k+1}\le S_{k}^{2}$,其中 $S_{0}=1$。当且仅当 $a_{1}=a_{2}=\dots=a_{n}$ 时等号成立。
\end{example}
\newpage

% 例 8
\begin{example}{麦克劳林不等式 (Maclaurin's Inequality)}{}
设 $a_{1},a_{2},\dots,a_{n}$ 是正实数,对 $1\le k\le n$,记
\begin{equation}
    S_{k}=\frac{\sum_{1\le i_{1}<i_{2}<\dots<i_{k}\le n}a_{i_{1}}a_{i_{2}}\dots a_{i_{k}}}{C_{n}^{k}}.
\end{equation}
则
\begin{equation}
    S_{1}\ge\sqrt{S_{2}}\ge\sqrt[3]{S_{3}}\ge\dots\ge\sqrt[n]{S_{n}}.
\end{equation}
当且仅当 $a_{1}=a_{2}=\dots=a_{n}$ 时等号成立。
\end{example}
\newpage


\section{作业题}

% === 作业题部分 ===

% 作业题 1
\begin{homework}{}{}
设 $n$ 是正整数,$a_{1},a_{2},\dots,a_{n},b_{1},b_{2},\dots,b_{n},A,B$ 是正实数,满足 $b_i \le a_i \le A,\ i = 1,2,\dots,n$,且 $\frac{b_{1}b_{2}\dots b_{n}}{a_{1}a_{2}\dots a_{n}}\le\frac{B}{A}$。
求证:
\begin{equation}
    \frac{(b_{1}+1)(b_{2}+1)\dots(b_{n}+1)}{(a_{1}+1)(a_{2}+1)\dots(a_{n}+1)}\le\frac{B+1}{A+1}.
\end{equation}
\end{homework}
\newpage

% 作业题 2
\begin{homework}{}{}
设 $x_1, x_2, \dots, x_n$ 和 $y_1, y_2, \dots, y_n$ 均为不减的正数数列,满足 $\sum_{i=1}^{n}x_{i}=\sum_{i=1}^{n}y_{i}$。
求证:
\begin{equation}
    \sum_{\emptyset \ne S\subseteq\{1,2,\dots,n\}}\frac{\sum_{i\in S}x_{i}}{\sum_{i\in S}y_{i}}\le 2^{n}-1.
\end{equation}
\end{homework}
\newpage

% 作业题 3
\begin{homework}{}{}
设 $0<a_{1}\le a_{2}\le\dots\le a_{n},\ b_{1}\ge b_{2}\ge\dots\ge b_{n}>0$,且对 $1 \le i \le n-1$,有 $\frac{a_{i+1}}{a_{i}}\le\frac{b_{i}}{b_{i+1}}$。
求证:
\begin{equation}
    \frac{A_{n}(a)}{G_{n}(a)}\le\left(\frac{A_{n}(b)}{G_{n}(b)}\right)^{n-1},
\end{equation}
其中 $A_n, G_n$ 分别表示算术平均值和几何平均值。
\end{homework}
\newpage

% 作业题 4
\begin{homework}{Suranyi 不等式}{}
设 $a_1, a_2, \dots, a_n$ 是正实数,则
\begin{equation}
    (n-1)\sum_{i=1}^{n}a_{i}^{n}+n\prod_{i=1}^{n}a_{i}\ge\left(\sum_{i=1}^{n}a_{i}\right)\left(\sum_{i=1}^{n}a_{i}^{n-1}\right).
\end{equation}
\end{homework}

\end{document}