% chapters/09_abs_ineq_1.tex

\chapter{绝对值不等式}
\section{绝对值不等式 (一)}
本讲介绍含绝对值的不等式,最常用的方法是三角不等式和正负分离。


% \section{预习题}
\section{例题}
% 例 1
\begin{example}{}{}
设整数 $n\ge2$,$a_{1},a_{2},\dots,a_{n},b_{1},b_{2},\dots,b_{n}$ 是实数,求证:存在 $1\le k\le n$,使得
\begin{equation}
    \sum_{i=1}^{n}|a_{i}-a_{k}|\le\sum_{i=1}^{n}|b_{i}-a_{k}|.
\end{equation}
\end{example}
\newpage

% 例 2
\begin{example}{}{}
设整数 $n\ge2$,实数 $a_{1},a_{2},\dots,a_{n}$ 满足 $\sum_{i=1}^{n-1}|a_{i}-a_{i+1}|=1$。对 $1\le k\le n$,记 $A_{k}=\frac{1}{k}\sum_{i=1}^{k}a_{i}$,求证:
\begin{equation}
    \sum_{i=1}^{n-1}|A_{i}-A_{i+1}|\le1-\frac{1}{n}.
\end{equation}
\end{example}
\newpage

% 例 3
\begin{example}{}{}
设整数 $n\ge2$,$a_{0},a_{1},\dots,a_{n}$ 是实数,满足 $a_{1}=a_{n-1}=0$。求证:对任意实数 $t$,
\begin{equation}
    |a_{0}|-|a_{n}|\le\sum_{i=0}^{n-2}|a_{i}-ta_{i+1}-a_{i+2}|.
\end{equation}
\end{example}
\newpage

% 例 4
\begin{example}{}{}
给定整数 $n\ge2$。设实数 $a_{1},a_{2},\dots,a_{n}$ 满足 $\sum_{i=1}^{n}a_{i}=0$ 且 $\sum_{i=1}^{n}a_{i}^{2}=1$。求:
\begin{enumerate}
    \item $\sum_{i=1}^{n}|a_{i}|$ 的最小值和最大值;
    \item $\max_{1\le i\le n}|a_{i}|$ 的最小值和最大值。
\end{enumerate}
\end{example}
\newpage

% 例 5
\begin{example}{}{}
给定整数 $n\ge2$。设非负实数 $a_{1}\le a_{2}\le\dots\le a_{n},\ b_{1}\le b_{2}\le\dots\le b_{n}$ 满足 $\sum_{i=1}^{n}a_{i}=\sum_{i=1}^{n}b_{i}=1$。求:
\begin{enumerate}
    \item $\min_{1\le i\le n}|a_{i}-b_{i}|$ 的最大值;
    \item $\sum_{i=1}^{n}|a_{i}-b_{i}|$ 的最大值。
\end{enumerate}
\end{example}
\newpage

% 例 6
\begin{example}{}{}
给定整数 $n \ge 2$。设实数 $-1\le a_{1}\le a_{2}\le\dots\le a_{n}\le1,\ -1\le b_{1}\le b_{2}\le\dots\le b_{n}\le1$ 满足 $\sum_{i=1}^{n}a_{i}=\sum_{i=1}^{n}b_{i}$。求 $\sum_{i=1}^{n}|a_{i}-b_{i}|$ 的最大值。
\end{example}
\newpage

% 例 7
\begin{example}{}{}
设整数 $n\ge3$,非零实数 $x_1, x_2, \dots, x_n$ 满足 $\frac{x_{1}}{x_{2}}+\frac{x_{2}}{x_{3}}+\dots+\frac{x_{n}}{x_{1}}=0$。求证:
\begin{equation}
    |x_{1}x_{2}+x_{2}x_{3}+\dots+x_{n}x_{1}|\le\sum_{i=1}^{n}|x_{i}|\cdot(\max_{1\le i\le n}|x_{i}|-\min_{1\le i\le n}|x_{i}|).
\end{equation}
\end{example}
\newpage

% 例 8
\begin{example}{}{}
设实数 $a<b,\ \lambda_{1},\lambda_{2},\dots,\lambda_{n}\in[a,b]$。设实数 $x_{1},x_{2},\dots,x_{n},y_{1},y_{2},\dots,y_{n}$ 满足 $\sum_{i=1}^{n}x_{i}^{2}=\sum_{i=1}^{n}y_{i}^{2}=1$。求证:
\begin{equation}
    \left|\sum_{i=1}^{n}\lambda_{i}(x_{i}^{2}-y_{i}^{2})\right|\le(b-a)\sqrt{1-(\sum_{i=1}^{n}x_{i}y_{i})^{2}}.
\end{equation}
\end{example}
\newpage



\section{练习题}
% 作业题 1
\begin{problem}{}{}
设整数 $n\ge2$,非零实数 $a_{1},a_{2},\dots,a_{n}$ 满足 $a_{1}+a_{2}+\dots+a_{n}=0$。求证:存在 $1\le i<j\le n$,使得
\begin{equation}
    \frac{1}{2}\le\left|\frac{a_{i}}{a_{j}}\right|\le2.
\end{equation}
\end{problem}
% 作业题 2
\begin{problem}{}{}
设整数 $n\ge3$,正实数 $a_{1},a_{2},\dots,a_{n}$ 满足 $a_{i}\le1\ (i=1,2,\dots,n)$。对 $1\le k\le n$,记 $A_{k}=\frac{1}{k}\sum_{i=1}^{k}a_{i}$。求证:
\begin{equation}
    \left|\sum_{i=1}^{n}a_{i}-\sum_{i=1}^{n}A_{i}\right|<\frac{n-1}{2}.
\end{equation}
\end{problem}

% 作业题 3
\begin{problem}{}{}
给定整数 $n\ge2$。求最大的实数 $\lambda$,使得对任意和为 0 的实数 $a_{1},a_{2},\dots,a_{n}$,都有
\begin{equation}
    \sum_{i=1}^{n}a_{i}^{2}+1\ge\lambda\sum_{i=1}^{n}|a_{i}|.
\end{equation}
\end{problem}

% 作业题 4
\begin{problem}{}{}
设正实数 $a_{1},a_{2},\dots,a_{n},b_{1},b_{2},\dots,b_{n}$ 满足 $\sum_{i=1}^{n}a_{i}=\sum_{i=1}^{n}b_{i}=1$。求证:
\begin{equation}
    \sum_{i=1}^{n}|a_{i}-b_{i}|\le2-\min_{1\le i\le n}\frac{a_{i}}{b_{i}}-\min_{1\le i\le n}\frac{b_{i}}{a_{i}}.
\end{equation}
\end{problem}
% chapters/10_abs_ineq_2.tex


\newpage 
\section{绝对值不等式 (二)}

本讲继续介绍含绝对值的不等式,包括设序和离散介值原理等方法,以及几个综合性的问题。
\section{例题}

% 例 1
\begin{example}{}{}
给定整数 $n\ge2$。设实数 $a_{1},a_{2},\dots,a_{n}\in[0,1]$,求
\begin{equation}
    \sum_{1\le i<j\le n}|a_{i}-a_{j}|
\end{equation}
的最大值。
\end{example}
\newpage

% 例 2
\begin{example}{}{}
给定整数 $n\ge2$。设实数 $a_{1},a_{2},\dots,a_{n}\in[-1,1]$,求
\begin{equation}
    \left|a_{1}-\frac{a_{2}+a_{3}+\dots+a_{n}}{n}\right| + \left|a_{2}-\frac{a_{1}+a_{3}+\dots+a_{n}}{n}\right| + \dots + \left|a_{n}-\frac{a_{1}+a_{2}+\dots+a_{n-1}}{n}\right|
\end{equation}
的最大值。
\end{example}
\newpage

% 例 3
\begin{example}{}{}
给定整数 $n\ge2$。设实数 $a_{1},a_{2},\dots,a_{n}$ 满足:
\begin{enumerate}
    \item $\sum_{i=1}^{n}a_{i}=0$;
    \item $|a_{i}|\le1,\ i=1,2,\dots,n$.
\end{enumerate}
求 $\min_{1\le i\le n-1}|a_{i}-a_{i+1}|$ 的最大值。
\end{example}
\newpage

% 例 4
\begin{example}{}{}
设整数 $n\ge3$,实数 $a_{1},a_{2},\dots,a_{n}$ 满足 $\sum_{i=1}^{n}a_{i}>1,\ |a_{i}|\le1\ (i=1,2,\dots,n)$。求证:存在正整数 $k<n$,使得
\begin{equation}
    \left|\sum_{i=1}^{k}a_{i}-\sum_{i=k+1}^{n}a_{i}\right|\le1.
\end{equation}
\end{example}
\newpage

% 例 5
\begin{example}{}{}
设实数 $a_{1},a_{2},\dots,a_{40}$ 满足 $\sum_{i=1}^{40}a_{i}=0$ 且对 $1\le i\le40$,都有 $|a_{i}-a_{i+1}|\le1$,这里 $a_{41}=a_{1}$。记 $a=a_{10},\ b=a_{20},\ c=a_{30},\ d=a_{40}$。
\begin{enumerate}
    \item 求 $a+b+c+d$ 的最大值;
    \item 求 $ab+cd$ 的最大值。
\end{enumerate}
\end{example}
\newpage

% 例 6
\begin{example}{}{}
设实数 $a_1, a_2, \dots, a_{1001}$ 满足 $a_{1}=a_{1001},\ |a_{i}+a_{i+2}-2a_{i+1}|\le1\ (i=1,2,\dots,999)$。求
\begin{equation}
    \max_{1\le i<j\le 1001}|a_{i}-a_{j}|
\end{equation}
的最大值。
\end{example}
\newpage

\section{作业题}
% 作业题 1
\begin{problem}{}{}
给定整数 $n\ge2$。设实数 $a_{1},a_{2},\dots,a_{n}$ 满足:
\begin{enumerate}
    \item $\sum_{i=1}^{n}a_{i}=0$;
    \item $\max_{1\le i\le n}|a_{i}|=1$.
\end{enumerate}
求 $\max_{1\le i\le n}|a_{i}-a_{i+1}|$ 的最小值,其中 $a_{n+1}=a_{1}$。
\end{problem}


% 作业题 2
\begin{problem}{}{}
设实数 $x_{1},x_{2},\dots,x_{n}$ 满足 $|x_{1}+x_{2}+\dots+x_{n}|=1$,且 $|x_{i}|\le\frac{n+1}{2}\ (1\le i\le n)$。求证:存在 $x_{1},x_{2},\dots,x_{n}$ 的一个排列 $y_{1},y_{2},\dots,y_{n}$,使得
\begin{equation}
    |y_{1}+2y_{2}+\dots+ny_{n}|\le\frac{n+1}{2}.
\end{equation}
\end{problem}


% 作业题 3
\begin{problem}{}{}
设实数 $a_1, a_2, \dots, a_{2018}$ 满足 $|a_{i+1}-a_{i}|\le1\ (1\le i\le 2018)$,其中 $a_{2019}=a_{1}$。求
\begin{equation}
    \sum_{i=1}^{2018}|a_{i}|-\left|\sum_{i=1}^{2018}a_{i}\right|
\end{equation}
的最大值。
\end{problem}
\newpage

\section{其他练习题}
\begin{example}{}{}
已知 \( n \) 个实数 \( x_1, x_2, \dots, x_n \) 的算术平均值为 \( a \),证明:
\[
\sum_{k=1}^{n} (x_k - a)^2 \leq \frac{1}{2} \cdot \left( \sum_{k=1}^{n} |x_k - a| \right)^2.
\]
\end{example}

\begin{example}{}{}
设 \( a_0 = 0 \),\( a_1, a_2, \dots, a_n \in \mathbb{R} \),证明:
\[
\sum_{k=1}^{n} |a_k(a_k - a_{k-1})| \leq \frac{n+1}{2} \cdot \sum_{k=1}^{n} (a_k - a_{k-1})^2.
\]
\end{example}

\begin{example}{}{}
设整数 \( n \geq 3 \),实数 \( a_1, a_2, \dots, a_n \) 均大于1,且 \( |a_{k+1} - a_k| < 1 \ (1 \leq k \leq n-1) \),证明:
\[
\frac{a_1}{a_2} + \frac{a_2}{a_3} + \dots + \frac{a_{n-1}}{a_n} + \frac{a_n}{a_1} < 2n - 1.
\]
\end{example}

\begin{example}{}{}
(18浙江预赛)将 \( 2n \ (n \geq 2) \) 个不同的整数分成两组 \( a_1, a_2, \dots, a_n, b_1, b_2, \dots, b_n \),证明:
\[
\sum_{\substack{1 \leq i \leq n \\ 1 \leq j \leq n}} |a_i - b_j| - \sum_{1 \leq i < j \leq n} \left( |a_j - a_i| + |b_j - b_i| \right) \geq n.
\]
\end{example}

\begin{example}{}{}
已知非负实数 \( x_1, x_2, \dots, x_n \) 均不超过1,证明:
\[
2 \cdot \sum_{i=1}^{n} \sum_{j=1}^{n} |x_i - x_j| \leq n^2.
\]
\end{example}

\begin{example}{}{}
已知实数 \( a_1, a_2, \dots, a_n \) 满足 \( a_1^2 + a_2^2 + \dots + a_n^2 = 1 \),求 \( |a_1 - a_2| + |a_2 - a_3| + \dots + |a_{n-1} - a_n| + |a_n - a_1| \) 的最大值。
\end{example}

\begin{example}{}{}
对每一个整数 \( n \geq 2 \),求最大的常数 \( c_n \),使得不等式
\[
c_n \cdot \sum_{i=1}^{n} |a_i| \leq \sum_{1 \leq i < j \leq n} |a_i - a_j|
\]
对任意满足 \( \sum_{i=1}^{n} a_i = 0 \) 的实数 \( a_1, a_2, \dots, a_n \) 成立。
\end{example}
