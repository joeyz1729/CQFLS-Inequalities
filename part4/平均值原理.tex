% chapters/17_mean_value_principle.tex

\chapter{平均值原理与不等式}

平均值原理是一种整体思想,本讲介绍两类能用平均值原理处理的问题,一类是存在性问题,一类与最大最小有关。

% === 例题部分 ===

% 例 1
\begin{example}{}{}
设非负实数 $x_{1},x_{2},\dots,x_{n}$ 满足 $x_{1}+x_{2}+\dots+x_{n}=1$。求证:存在 $x_{1},x_{2},\dots,x_{n}$ 的一个排列 $a_{1},a_{2},\dots,a_{n}$,使得
\begin{equation}
    a_{1}a_{2}+a_{2}a_{3}+\dots+a_{n}a_{1}\le\frac{1}{n}.
\end{equation}
\end{example}
\newpage

% 例 2
\begin{example}{}{}
设 $x_{1},x_{2},\dots,x_{n}$ 是实数,求证:存在 $x_{1},x_{2},\dots,x_{n}$ 的一个排列 $a_{1},a_{2},\dots,a_{n}$,使得
\begin{equation}
    \left|\sum_{i=1}^{n}ia_{i}\right|\ge\frac{n-1}{2}\max_{1\le i<j\le n}|x_{i}-x_{j}|.
\end{equation}
\end{example}
\newpage

% 例 3
\begin{example}{}{}
设实数 $x_{1},x_{2},\dots,x_{n}$ 满足 $x_{1}^{2}+x_{2}^{2}+\dots+x_{n}^{2}=1$,整数 $k\ge2$。求证:存在不全为 0 的整数 $a_{1},a_{2},\dots,a_{n}$,使得其中每一个的绝对值都不超过 $k-1$,且
\begin{equation}
    |a_{1}x_{1}+a_{2}x_{2}+\dots+a_{n}x_{n}|\le\frac{(k-1)\sqrt{n}}{k^{n}-1}.
\end{equation}
\end{example}
\newpage

% 例 4
\begin{example}{}{}
设实数 $a_{1},a_{2},\dots,a_{n}$ 满足 $0\le a_{1}<a_{2}<\dots<a_{n}\le1$。求证:存在 $x\in[0,1]$ 使得
\begin{equation}
    \sum_{i=1}^{n}\frac{1}{|x-a_{i}|}\le 8n\left(1+\frac{1}{3}+\dots+\frac{1}{2\lceil\frac{n}{2}\rceil-1}\right).
\end{equation}
\end{example}
\newpage

% 例 5
\begin{example}{}{}
设 $a_{1},a_{2},\dots,a_{n}$ 是小于 1 的正实数,$k$ 是正整数,求证:
\begin{equation}
    \min\{a_{1}(1-a_{2})^{k},a_{2}(1-a_{3})^{k},\dots,a_{n}(1-a_{1})^{k}\}\le\frac{k^{k}}{(k+1)^{k+1}}.
\end{equation}
\end{example}
\newpage

% 例 6
\begin{example}{}{}
给定正实数 $a,b$,整数 $n\ge2$。设函数 $f(x)=(x+a)(x+b)$,非负实数 $x_{1},x_{2},\dots,x_{n}$ 满足 $x_{1}+x_{2}+\dots+x_{n}=1$。求 $\sum_{1\le i<j\le n}\min\{f(x_{i}),f(x_{j})\}$ 的最大值。
\end{example}
\newpage

% 例 7
\begin{example}{}{}
设非负实数 $a_{1},a_{2},\dots,a_{9}$ 满足 $\sum_{i=1}^{9}a_{i}=1$。记
\begin{align}
    S &= \min\{a_{1},a_{2}\}+2\min\{a_{2},a_{3}\}+\dots+9\min\{a_{9},a_{1}\}, \\
    T &= \max\{a_{1},a_{2}\}+2\max\{a_{2},a_{3}\}+\dots+9\max\{a_{9},a_{1}\}.
\end{align}
当 $S$ 取最大值 $S_{0}$ 时,求 $T$ 的所有可能值。
\end{example}
\newpage

% 例 8
\begin{example}{}{}
设整数 $n\ge2$,$a_{1},a_{2},\dots,a_{n}$ 是正实数,求证:
\begin{equation}
    (\max_{1\le i\le n}a_{i})\left(\sum_{i=1}^{n}ia_{i}\right)\ge\frac{n+1}{n-1}\sum_{1\le i<j\le n}a_{i}a_{j}.
\end{equation}
\end{example}
\newpage

% 例 9
\begin{example}{}{}
给定整数 $n\ge2$。求最小的正实数 $\lambda$,使得对任意正实数 $a_{1},a_{2},\dots,a_{n}$,都有
\begin{equation}
    \sum_{i=1}^{n}\max\{a_{1},\dots,a_{i}\}\cdot \min\{a_{i},\dots,a_{n}\}\le\lambda\sum_{i=1}^{n}a_{i}^{2}.
\end{equation}
\end{example}




\newpage


% === 作业题部分 ===
\section{作业题}
% 作业题 1
\begin{homework}{}{}
设 $a_{1},a_{2},\dots,a_{n}$ 是实数,求证:存在实数 $x$,使得
\begin{equation}
    \{x-a_{1}\}+\{x-a_{2}\}+\dots+\{x-a_{n}\}\le\frac{n-1}{2},
\end{equation}
其中 $\{x\}$ 表示实数 $x$ 的小数部分。
\end{homework}


\newpage
% 作业题 2
\begin{homework}{}{}
设整数 $n\ge3$,$a_{1},a_{2},\dots,a_{n}$ 是实数,求证:存在 $\{1,2,\dots,n\}$ 的子集 $S$,满足对任意 $1\le i\le n-2$,有 $1\le|S\cap\{i,i+1,i+2\}|\le2$ 且
\begin{equation}
    \left|\sum_{i\in S}a_{i}\right|\ge\frac{1}{6}\sum_{i=1}^{n}|a_{i}|.
\end{equation}
\end{homework}

\newpage
% 作业题 3
\begin{homework}{}{}
给定整数 $n\ge4$。求最大的实数 $\lambda$,使得对任意满足 $a_{1}+a_{2}+\dots+a_{n}=0$ 的实数 $a_{1},a_{2},\dots,a_{n}$,都有
\begin{equation}
    \lambda \min\{a_{1},a_{2},\dots,a_{n}\}\ge \min\{a_{1},a_{2}\}+\min\{a_{2},a_{3}\}+\dots+\min\{a_{n},a_{1}\}.
\end{equation}
\end{homework}
\newpage
% 作业题 4
\begin{homework}{}{}
设 $a_{1},a_{2},\dots,a_{n}$ 是实数。对 $1\le i\le n$,定义
\begin{equation}
    d_{i}=\max_{1\le j\le i}a_{j}-\min_{i\le j\le n}a_{j}.
\end{equation}
令 $d=\max_{1\le i\le n}d_{i}$。
\begin{enumerate}
    \item 求证:对任意实数 $x_{1}\le x_{2}\le\dots\le x_{n}$,$\max_{1\le i\le n}|x_i - a_i| \ge \frac{d}{2}$.
    \item 求证:存在实数 $x_{1}\le x_{2}\le\dots\le x_{n}$ 使得 (1) 中等号成立。
\end{enumerate}
\end{homework}
