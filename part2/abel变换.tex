% chapters/03_abel.tex

\chapter{Abel 变换}

\section{基础知识}
\begin{thm}{Abel 变换 (分部求和公式)}{}
令 $S_0=0$,$S_k=\sum_{i=1}^k a_i\ (1\leq k\leq n)$,则
\begin{equation}
    \sum_{k=1}^n a_k b_k = \sum_{k=1}^{n-1} S_k(b_k-b_{k+1}) + b_n S_n
\end{equation}
\end{thm}
\newpage
% \begin{thm}{钟开莱不等式}{}
% 设 $a_{1},a_{2},\dots,a_{n}$ 是实数, $b_{1}\ge b_{2}\ge\dots\ge b_{n}>0$,满足对任意 $1\le k\le n$,有 
% \begin{equation}
%     \sum_{i=1}^{k}a_{i}\ge\sum_{i=1}^{k}b_{i}.
% \end{equation}
% 求证:
% \begin{equation}
%     \sum_{i=1}^{n}a_{i}^{2}\ge\sum_{i=1}^{n}b_{i}^{2}.
% \end{equation}
% \end{thm}
% 钟开莱不等式
\begin{thm}{钟开莱不等式}{}
给定整数 $n\geq 2$,设 $a_1\geq a_2\geq\dots\geq a_n>0$ 且对任意 $1\leq k\leq n$ 有 $\sum_{i=1}^k a_i\leq\sum_{i=1}^k b_i$,求证:
\begin{align}
    \sum_{i=1}^n a_i^2 &\leq\sum_{i=1}^n b_i^2,\\
    \sum_{i=1}^n a_i^3 &\leq\sum_{i=1}^n a_i b_i^2.
\end{align}
\end{thm}
\newpage



% === 预习题部分 ===
% === 预习题部分 ===
% === 预习题部分 ===
\section{预习题}
\begin{preview}{}{}
给定正整数 $n$,对任意 $1\leq k\leq n$ 正实数 $a_1,a_2,\dots,a_n$ 满足 $a_1+a_2+\dots+a_k\leq k$,证明:
\begin{equation}
    \sum_{i=1}^n \frac{a_i}{i} \leq\sum_{i=1}^n \frac{1}{i}
\end{equation}
\end{preview}
\vspace{7cm}

\begin{preview}{1978 IMO}{}
已知 $a_1,a_2,\dots,a_n$ 是两两不同的正整数,证明:
\begin{equation}
    \sum_{i=1}^n \frac{a_i}{i^2} \geq\sum_{i=1}^n \frac{1}{i}
\end{equation}
\end{preview}












\newpage
% === 例题部分 ===
% === 例题部分 ===
% === 例题部分 ===
\section{例题}



\begin{example}{1994 USAMO}{}
设 $a_{1},a_{2},\dots,a_{n}$ 是正实数,满足对 $1\le k\le n$,有 $\sum_{i=1}^{k}a_{i}\ge\sqrt{k}.$
证明:
\begin{equation}
    \sum_{i=1}^{n}a_{i}^{2}\ge\frac{1}{4}\sum_{i=1}^{n}\frac{1}{i}.
\end{equation}
\end{example}
\newpage


\begin{example}{}{}
设 $a_{1},a_{2},\dots,a_{n}$ 是实数,求证:存在 $k\in\{1,2,\dots,n\}$,使得对任意 $1\ge b_{1}\ge b_{2}\ge\dots\ge b_{n}\ge0$ 都有
\begin{equation}
    \left|\sum_{i=1}^{n}a_{i}b_{i}\right|\le\left|\sum_{i=1}^{k}a_{i}\right|.
\end{equation}
\end{example}


\newpage 
\begin{example}{1999 APMO}{}
设 $\{a_n\}$ 是正项数列,满足对任意 $i,j\ge1$,有 $a_{i+j}\le a_{i}+a_{j}.$ 求证:对任意正整数 $n$,有
\begin{equation}
    \sum_{i=1}^{n}\frac{a_{i}}{i}\ge a_{n}.
\end{equation}
\end{example}
\newpage


\begin{example}{}{}
设正实数 $a_{1},a_{2},\dots,a_{n},b_{1},b_{2},\dots,b_{n}$ 满足 $b_{1}\ge b_{2}\ge\dots\ge b_{n}$,且对 $1\le k\le n$,有 $\prod_{i=1}^{k}a_{i}\ge\prod_{i=1}^{k}b_{i}.$ 
求证:
\begin{equation}
    \sum_{i=1}^{n}a_{i}\ge\sum_{i=1}^{n}b_{i}.
\end{equation}
\end{example}

\newpage
\begin{example}{加强形式的Chebyshev不等式}{}
设 $a_{1},a_{2},\dots,a_{n}$ 和 $b_{1},b_{2},\dots,b_{n}$ 是实数,满足
\begin{gather}
    a_{1}\ge\frac{a_{1}+a_{2}}{2}\ge\dots\ge\frac{a_{1}+a_{2}+\dots+a_{n}}{n}, \\
    b_{1}\ge\frac{b_{1}+b_{2}}{2}\ge\dots\ge\frac{b_{1}+b_{2}+\dots+b_{n}}{n}.
\end{gather}
求证:
\begin{equation}
    \sum_{i=1}^{n}a_{i}b_{i}\ge\frac{1}{n}\left(\sum_{i=1}^{n}a_{i}\right)\left(\sum_{i=1}^{n}b_{i}\right).
\end{equation}
\end{example}


\newpage
\begin{example}{}{}
已知实数 $a_1,a_2,\dots,a_n$,证明:
\begin{equation}
    \sum_{i=1}^n a_i^2 \geq\sum_{i=1}^{n-1} a_i a_{i+1} +\frac{3}{2(n+1)^3}\biggl(\sum_{i=1}^n a_i\biggr)^2
\end{equation}
\end{example}

\newpage
% 第 8 题
\begin{example}{2018 清华飞测}{}
给定正整数 $n$, 对任意 $1\leq k\leq n$ 正实数 $a_1,a_2,\dots,a_k$ 满足 $\sum_{i=1}^k a_i\leq k^2$,证明:
\begin{equation}
    \sum_{i=1}^n \frac{1}{a_i} >\frac{1}{4}\log_2 n
\end{equation}
\end{example}







\newpage
\begin{example}{}{}
给定整数 $n, k\ge2$. 设非负实数 $a_{1},a_{2},\dots,a_{n},c_{1},c_{2},\dots,c_{n}$ 满足:
\begin{enumerate}
    \item $a_{1}\ge a_{2}\ge\dots\ge a_{n},$ 且 $a_{1}+a_{2}+\dots+a_{n}=1$;
    \item 对 $1\le i\le n$,有 $\sum_{j=1}^{i}c_{j} \le i^k$.
\end{enumerate}
求 $c_{1}a_{1}^{k}+c_{2}a_{2}^{k}+\dots+c_{n}a_{n}^{k}$ 的最大值。
\end{example}











\newpage
% === 练习题部分 ===
\section{练习题}
\begin{homework}{}{}
给定整数 $n\ge2$ 以及不全为零的实数 $a_{1},a_{2},\dots,a_{n}.$ 求 $a_{1},a_{2},\dots,a_{n}$ 满足的充要条件,使得存在正整数 $x_{1}>x_{2}>\dots>x_n$,满足
\begin{equation}
    a_{1}x_{1}+a_{2}x_{2}+\dots+a_{n}x_{n}\ge0.
\end{equation}
\end{homework}



\newpage 
\begin{homework}{}{}
设整数 $n\ge2$,正实数 $a_{1},a_{2},\dots,a_{n}$ 满足 
\[
a_{1}\le a_{2},\quad a_{1}+a_{2}\le a_{3},\quad \dots,\quad a_{1}+a_{2}+\dots+a_{n-1}\le a_{n}.
\]
求证:
\begin{equation}
    \frac{a_{1}}{a_{2}}+\frac{a_{2}}{a_{3}}+\dots+\frac{a_{n-1}}{a_{n}}\le\frac{n}{2}.
\end{equation}
\end{homework}


\newpage
\begin{homework}{}{}
给定实数 $a_1,a_2,\dots,a_{n+1}$,记 $M=\max_{1\leq k\leq n}|a_k-a_{k+1}|$,证明:
\begin{equation}
    \left|\frac{1}{n}\sum_{k=1}^n a_k-\frac{1}{n+1}\sum_{k=1}^{n+1} a_k\right| \leq\frac{M}{2}
\end{equation}
\end{homework}


\newpage
\begin{homework}{}{}
设 $a_{1},a_{2},\dots,a_{n},b_{1},b_{2},\dots,b_{n}$ 是实数。求证:对任意实数 $x_{1}\le x_{2}\le \dots\le x_{n}$ 都有 $\sum_{i=1}^{n}a_{i}x_{i}\le\sum_{i=1}^{n}b_{i}x_{i}$ 的充要条件是:
对 $1\le k\le n-1$,有
\begin{equation}
    \sum_{i=1}^{k}a_{i}\ge\sum_{i=1}^{k}b_{i} \quad \text{且} \quad \sum_{i=1}^{n}a_{i}=\sum_{i=1}^{n}b_{i}.
\end{equation}
\end{homework}

\newpage
\begin{homework}{}{}
设正实数 $a_{1},a_{2},\dots,a_{n},b_{1},b_{2},\dots,b_{n}$ 满足 $a_{1}\le a_{2}\le\dots\le a_{n}$,且对 $1\le k\le n$,有 $\sum_{i=1}^{k}a_{i}\ge\sum_{i=1}^{k}b_{i}.$ 
求证:
\begin{equation}
    \sum_{i=1}^{n}\sqrt{a_{i}}\ge\sum_{i=1}^{n}\sqrt{b_{i}}.
\end{equation}
\end{homework}

\newpage
\begin{homework}{}{}
给定整数 $n\ge2$,设 $a_{1},a_{2},\dots,a_{n}$ 是正整数,满足对集合 $\{1,2,\dots,n\}$ 的任一非空子集 $I$,$\sum_{i\in I}a_{i}$ 互不相同。求:
\begin{enumerate}
    \item $\sum_{i=1}^{n}\sqrt{a_{i}}$ 的最小值;
    \item $\sum_{i=1}^{n}a_{i}^{2}$ 的最小值;
    \item $\sum_{i=1}^{n}\frac{1}{a_{i}}$ 的最大值。
\end{enumerate}
\end{homework}