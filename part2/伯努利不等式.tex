% chapters/14_bernoulli.tex

\chapter{伯努利不等式}
\section{基础知识}

% \begin{thm}{伯努利不等式 (Bernoulli's Inequality)}{bernoulli_basic}
% 对任意实数 $x > -1$ 和整数 $n \ge 0$,有:
% \begin{equation}
%     (1+x)^n \ge 1+nx
% \end{equation}
% 当且仅当 $n=0, 1$ 或 $x=0$ 时取等号。
% \end{thm}
% % \vspace{2cm}

\begin{thm}{伯努利不等式 (Bernoulli's Inequality)}{bernoulli_gen}
对任意实数 $x > -1$ 和实数 $\alpha$,有:
\begin{enumerate}
    \item 当 $\alpha \in (-\infty, 0] \cup [1, +\infty)$ 时:
    \begin{equation}
        (1+x)^\alpha \ge 1+\alpha x
    \end{equation}
    \item 当 $\alpha \in (0, 1)$ 时:
    \begin{equation}
        (1+x)^\alpha \le 1+\alpha x
    \end{equation}
\end{enumerate}
当且仅当 $\alpha = 0, 1$ 或 $x=0$ 时取等号。
\end{thm}


\newpage 
\begin{thm}{广义伯努利不等式}{}
设 $x_1, x_2, \dots, x_n$ 为符号相同的实数,且 $x_i > -1$,则:
\begin{equation}
    (1+x_1)(1+x_2)\cdots(1+x_n) \ge 1 + x_1 + x_2 + \dots + x_n
\end{equation}
\end{thm}






% === 例题部分 ===

\newpage  
\section{例题}
% 例 1
\begin{example}{}{}
设正实数 $a_{1},a_{2},\dots,a_{n}\le1$,求证:
\begin{equation}
    (1+a_{1})^{\frac{1}{a_{2}}}(1+a_{2})^{\frac{1}{a_{3}}}\dots(1+a_{n})^{\frac{1}{a_{1}}}\ge2^{n}.
\end{equation}
\end{example}
\newpage



% 例 3
\begin{example}{}{}
设实数 $a_{1},a_{2},\dots,a_{n}\ge1$,求证:
\begin{equation}
    (1+a_{1})(1+a_{2})\dots(1+a_{n})\ge\frac{2^{n}}{n+1}(1+a_{1}+a_{2}+\dots+a_{n}).
\end{equation}
\end{example}
\newpage

% 例 4
\begin{example}{}{}
求最小的实数 $\lambda$,使得对任意正整数 $n$ 及任意满足 $\sum_{i=1}^{n}a_{i}=1$ 的正实数 $a_{1},a_{2},\dots,a_{n}$ 都有
\begin{equation}
    \lambda\prod_{i=1}^{n}(1-a_{i})\ge1-\sum_{i=1}^{n}a_{i}^{2}.
\end{equation}
\end{example}
\newpage

% 例 5
\begin{example}{}{}
设 $a_{1},a_{2},\dots,a_{n}$ 是正实数,求证:
\begin{equation}
    \prod_{i=1}^{n}(a_{i}^{2}+n-1)\ge n^{n-2}\Bigl(\sum_{i=1}^{n}a_{i}\Bigr)^{2}.
\end{equation}
\end{example}
\newpage

% 例 6
\begin{example}{}{}
设正实数 $a_{1},a_{2},\dots,a_{n}$ 满足 $\sum_{i=1}^{n}a_{i}=n$。求证:
\begin{equation}
    (a_{1}^{n}+1)(a_{2}^{n}+1)\dots(a_{n}^{n}+1)\ge2^{n}.
\end{equation}
\end{example}
\newpage

% 例 7
\begin{example}{}{}
设 $a_{1},a_{2},\dots,a_{n}$ 是不全为 1 的正整数,满足 $\frac{1}{a_{1}}+\frac{1}{a_{2}}+\dots+\frac{1}{a_{n}}=k$ 是整数。求证:多项式
\begin{equation}
    P(x)=a_{1}a_{2}\dots a_{n}(x+1)^{k}-(x+a_{1})(x+a_{2})\dots(x+a_{n})
\end{equation}
没有正根。
\end{example}
\newpage

% 例 8
\begin{example}{}{}
设整数 $n\ge3$,多项式 $P(x)=x^{n}+a_{n-1}x^{n-1}+a_{n-2}x^{n-2}+\dots+a_{1}x+a_0$ 有 $n$ 个实根,且都在区间 $(0,1)$ 内。求证:
\begin{equation}
    \sum_{i=1}^{n-2}ia_{i}>0.
\end{equation}
\end{example}
\newpage


% 例 2
\begin{thm}{Mitrinovic 不等式}{}
设整数 $n\ge2$,$a_{1},a_{2},\dots,a_{n}$ 是正实数,求证:对 $a_{1},a_{2},\dots,a_{n}$ 的任一排列 $b_{1},b_{2},\dots,b_{n}$,都有
\begin{equation}
    \sum_{i=1}^{n}a_{i}^{b_{i}}>1.
\end{equation}
\end{thm}
\newpage



% === 作业题部分 ===
\section{练习题}
% 作业题 1
\begin{homework}{}{}
设实数 $a_{1},a_{2},\dots,a_{n}$ 满足 $a_{i}\ge-1,\ i=1,2,\dots,n$ 且 $\sum_{i=1}^{n}a_{i}=0$。求证:
\begin{equation}
    \prod_{i=1}^{n}(1+a_{i})+\frac{n}{4}\sum_{i=1}^{n}a_{i}^{2}\ge1.
\end{equation}
\end{homework}
% 作业题 2
\newpage 
\begin{homework}{}{}
给定整数 $n\ge2$,求最小的实数 $\lambda$,使得对任意非负实数 $a_{1},a_{2},\dots,a_{n}$,都有
\begin{equation}
    \sum_{i=1}^{n}\sqrt{a_{i}}\le\sqrt{\prod_{i=1}^{n}(a_{i}+\lambda)}.
\end{equation}
\end{homework}