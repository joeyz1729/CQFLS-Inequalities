% chapters/11_adjustment_method.tex

\chapter{调整法}
独立变量:固定其余变量,考虑单一变量的影响,利用单调性。约束变量:固定其余变量,固定两个变量的和或积,考虑变化影响。  

\begin{thm}{}{}
\begin{enumerate}
\item 设 \( a_k \in \mathbb{R} \),\( \sum_{k=1}^n a_k = t \),若在 \( a_1 \) 最大、\( a_2 \) 最小的前提下能证明
\[ f(a_1,a_2,\cdots,a_n) \geq f\left(\frac{t}{n}, a_1+a_2-\frac{t}{n}, a_3,\cdots,a_n\right), \]
则可经过 \( n-1 \) 次调整将 \( a_1,a_2,\cdots,a_n \) 全调为 \( \frac{t}{n} \)。
\item 设 \( a_k \in \mathbb{R}^+ \),\( \prod_{k=1}^n a_k = t \),若在 \( a_1 \) 最大、\( a_2 \) 最小的前提下能证明
\[ f(a_1,a_2,\cdots,a_n) \geq f\left(\sqrt[n]{t}, \frac{a_1a_2}{\sqrt[n]{t}}, a_3,\cdots,a_n\right), \]
则可经过 \( n-1 \) 次调整将 \( a_1,a_2,\cdots,a_n \) 全调为 \( \sqrt[n]{t} \)。
\end{enumerate}
\end{thm}
\begin{thm}{端点调整}{}
若 \( a_k \in [a,b] \),将 \( f(a_1,a_2,\cdots,a_n) \) 视为 \( a_k \) 的一元函数 \( g(x) \),且 \( g'(x) \) 单调,则可将 \( a_k \) 调为 \( a \) 或 \( b \)。
% (单增最大值在右端点,单减最小值在左端点)注:有时不需 \( g'(x) \) 单调,只需 \( g'(x) \) 的某一部分因式单调,而这一部分与 \( g'(x) \) 同号即可!
\end{thm}

\begin{thm}{磨光变换法(无限调整法、SMV定理)}{}
\vspace{1cm}
\end{thm}
\begin{thm}{EV(Equal Variable)定理}{}
\vspace{1cm}
\end{thm}
\newpage

% \begin{thm}{}{}
% 设 \( a_k \geq 0 \),\( \sum_{k=1}^n a_k = t \),正整数 \( m \geq 2 \),若在 \( a_1+a_2 \leq \frac{2t}{m} \) 的前提下能证明
% \[ f(a_1,a_2,\cdots,a_n) \geq f(0, a_1+a_2, a_3,\cdots,a_n), \]
% 则可经过 \( n-m+1 \) 次调整将 \( a_1,a_2,\cdots,a_n \) 调成 \( n-m+1 \) 个 \( 0 \)。
% \end{thm}


% \begin{thm}{}{}
% 例1:\( a_k \geq 0 \),\( \sum_{k=1}^n a_k = 2 \),\( n \geq 3 \),求 \( 3a_1 + \sum_{k=2}^n 2^{k+1}a_2\cdots a_k \) 最大值。
%     简解:不妨设 \( a_1 \geq a_2 \geq \cdots \geq a_n \),令上式为 \( f(a_1,a_2,\cdots,a_n) \),则
%     \[ \text{易证 } f(a_1,a_2,\cdots,a_n) \leq f(a_1+a_4+a_5+\cdots+a_n, a_2, a_3, 0, 0,\cdots,0). \]
% \end{thm}

% \begin{thm}{}{}
% 例2:\( a_k \geq 0 \),\( \sum_{k=1}^n a_k = 2 \),求 \( \sum_{k=1}^n \frac{1}{\sum_{j=k}^n a_j^2} \) 最小值。
%     简解:不妨设 \( a_1 \geq a_2 \geq \cdots \geq a_n \),令上式为 \( f(a_1,a_2,\cdots,a_n) \),则易证
%     \[ f(a_1,a_2,\cdots,a_n) \geq f\left(a_1+\frac{a_3+a_4+\cdots+a_n}{2}, a_2+\frac{a_3+a_4+\cdots+a_n}{2}, 0, 0,\cdots,0\right). \]
% \end{thm}

% \begin{thm}{}{}
% (多次不同调整)例:若干个不同正整数之和为2026,求积最大值。
% 分6步调整:
% (1) 设这些数为 \( a_1 < a_2 < \cdots < a_n \),先说明 \( a_1 > 1 \);
% (2) 说明 \( a_1 < 5 \);
% (3) 说明 \( a_1 < 4 \);
% (4) 说明 \( a_{k+1}-a_k \leq 2 \);
% (5) 说明使 \( a_{k+1}-a_k = 2 \) 的 \( k \) 至多1个;
% (6) 说明 \( a_{k+1}-a_k = 2 \) 的 \( k \) 恰有1个。
% 最后分别在 \( a_1=2 \) 和 \( a_1=3 \) 时求出积进行比较即可。
% \end{thm}

% \begin{thm}{}{}
% (多次不同调整)例2:给定正整数 \( n \geq 3 \),设 \( a_k \in [-1,1] \),\( \sum_{k=1}^n a_k = 0 \),求 \( \sum_{k=1}^n a_k^{2025} \) 最大值。
% 分3步调整:
% (1) 若 \( a_1 > 0 > a_2 \) 且 \( a_1+a_2 \leq 0 \),则将 \( a_1,a_2 \) 调为 \( 0, a_1+a_2 \);
% (2) 若 \( a_1 > 0 > a_2 \) 且 \( a_1+a_2 > 0 \),将 \( a_1,a_2 \) 调为 \( 1, a_1+a_2-1 \);
% (3) 利用赫尔德不等式将所有负数调成相等。
% 如此正数全为1,负数全相等,再设出正数个数 \( x \),化为 \( x \) 的一元函数。
% \end{thm}


% \begin{thm}{}{}
%     (2个方向的调整)例:2025年CMO第5题:若 \( n \) 元对称不等式 \( f(a_1,a_2,\cdots,a_n) \leq 0 \) 中关于 \( a_1a_2 \) 为开口向上的二次函数,\( \sum_{k=1}^n a_k = nt \),\( a_k \geq 0 \)。
% 由于二次函数最大值在端点,若设 \( a_1 \) 最大、\( a_2 \) 最小,则
% \[ 0(a_1+a_2) \leq a_1a_2 \leq t(a_1+a_2 - t), \]
% 故有 \( f(a_1,a_2,\cdots,a_n) \leq f(0, a_1+a_2, a_3,\cdots,a_n) \) 或 \( f(t, a_1+a_2 - t, a_3,\cdots,a_n) \)。
% 从而经有限次调整后,\( a_1,a_2,\cdots,a_n \) 化为 \( n-m \) 个 \( 0 \) 和 \( m \) 个 \( \frac{nt}{m} \)。
% \end{thm}


% \begin{thm}{}{}
% (\( 2^n \) 元不等式的调整)若 \( 2^n \) 元对称不等式 \( f(a_1,a_2,\cdots,a_{2^n}) \) 中,可将 \( a_1,a_2 \) 调为 \( \frac{a_1+a_2}{2}, \frac{a_1+a_2}{2} \)(或 \( \sqrt{a_1a_2}, \sqrt{a_1a_2} \)),其中 \( a_1,a_2 \) 为任意2个数(不必最大或最小),则可经过 \( n \times 2^{n-1} \) 次调整后将所有数调至相等。
% \end{thm}

% \begin{thm}{}{}
% (部分调整)指仅调整 \( n \) 个变量中的 \( m \) 个变量(\( m < n \)),调整方式变化万千,不再列举。
% \end{thm}


% === 预习题 ===
\section{预习题}
\begin{preview}{}{}
已知 $0 \le a, b, c \le 1$,证明:
\begin{equation}
    \frac{a}{bc+1} + \frac{b}{ca+1} + \frac{c}{ab+1} \le 2
\end{equation}
\end{preview}


\begin{preview}{}{}
设 $x, y, z$ 都是非负实数且 $x+y+z=1$,证明:
    \begin{equation}
        yz+zx+xy-2xyz \le \frac{7}{27}
    \end{equation}
\end{preview}


\begin{preview}{}{}
设正实数 $a, b, c, d$ 满足 $abcd = 1$,证明:
\begin{equation}
        \frac{1}{a} + \frac{1}{b} + \frac{1}{c} + \frac{1}{d} + \frac{4}{a+b+c+d} \ge 5
\end{equation}
\end{preview}


\begin{preview}{}{}
设实数 $a, b, c$ 满足 $a+b+c=1,\ abc>0$,证明:
\begin{equation}
    ab+bc+ca < \frac{\sqrt{abc}}{2} + \frac{1}{4}
\end{equation}
\end{preview}
\newpage


% === 例题 ===
\section{例题}
\begin{example}{}{}
已知非负实数\( x_1, x_2, \dots, x_n \ (n \geq 3) \)满足不等式\( x_1 + x_2 + \dots + x_n \leq \frac{1}{2} \),求\( (1 - x_1)(1 - x_2)\cdots(1 - x_n) \)的最小值。
\end{example}
\newpage
\begin{example}{}{}
已知正实数 \( x_1, x_2, \dots, x_n \) 满足 \( x_1 + x_2 + \dots + x_n = 1 \),证明:
\begin{equation}
\frac{(1 - x_1)(1 - x_2)\cdots(1 - x_n)}{x_1x_2\cdots x_n} \geq (n - 1)^n.
\end{equation}
\end{example}
\newpage
\begin{example}{}{}
设整数 \( n \geq 2 \),正实数 \( x_1, x_2, \dots, x_n \) 满足 \( x_i x_j \geq 1 \)(其中 \( i \neq j \),\( 1 \leq i \leq n \),\( 1 \leq j \leq n \)),证明:
\begin{equation}
\frac{1}{1 + x_1} + \frac{1}{1 + x_2} + \dots + \frac{1}{1 + x_n} \geq \frac{n}{1 + \sqrt[n]{x_1x_2\cdots x_n}}.
\end{equation}
\end{example}
\newpage
\begin{example}{}{}
已知正实数 \( a_1, a_2, \dots, a_n \) 满足 \( \sum_{i=1}^{n} a_i = n \),证明:
\begin{equation}
a_1^2 + a_2^2 + \dots + a_n^2 + a_1a_2\cdots a_n \geq n + 1.
\end{equation}
\end{example}
\newpage




\begin{example}{}{}
设 \( x_i \in (0, 1] \)(\( 1 \leq i \leq n \)),\( 0 < \lambda \leq 2 \),证明:
\begin{equation}
\sum_{i=1}^{n} \left[1 + (i - 1)\lambda\right] \cdot x_i^2 \geq \frac{2 + (n - 1)\lambda}{2n} \cdot \left( \sum_{i=1}^{n} x_i \right)^2.
\end{equation}
\end{example}
\newpage



\begin{example}{}{}
已知 \( a_1, a_2, \dots, a_n \) 为正实数,证明:
\begin{equation}
\frac{\sum_{i=1}^{n} a_i}{n} - \sqrt[n]{\prod_{i=1}^{n} a_i} \leq \max_{1 \leq i < j \leq n} \left\{ (\sqrt{a_i} - \sqrt{a_j})^2 \right\}.
\end{equation}
\end{example}
\newpage



% % 例 5
% \begin{example}{}{}
% 给定不小于 3 的正整数 $n$。已知实数 $x_1, x_2, \dots, x_n$ 满足 $x_1+x_2+\dots+x_n=0$ 且 $x_1^2+x_2^2+\dots+x_n^2=1$,求 $(2+x_1)(2+x_2)\dots(2+x_n)$ 的最大值。
% \end{example}
% \newpage

% === 练习题 ===
\section{练习题}
\begin{homework}{}{}
给定正整数 \( n \geq 4 \),且 \( \sum_{i=1}^{n} x_i \geq n \),\( \sum_{i=1}^{n} x_i^2 \geq n^2 \),证明:这 \( n \) 个数中一定有一个数大于等于2。
\end{homework}

\newpage
\begin{homework}{}{}
设非负实数 \( a_1, a_2, \dots, a_n \) 中的最大数为 \( a \),证明:
\begin{equation}
\frac{a_1^2 + a_2^2 + \dots + a_n^2}{n} \leq \left( \frac{a_1 + a_2 + \dots + a_n}{n} \right)^2 + \frac{a^2}{4}.
\end{equation}
并确定不等式等号成立的条件。
\end{homework}


\newpage
\begin{homework}{}{}
设 $a_1, a_2, \dots, a_n$ 为 $n\ (n \ge 2)$ 个互不相同的实数,记
\begin{equation}
    S = a_1^2 + a_2^2 + \dots + a_n^2, \quad M = \min_{1 \le i < j \le n} (a_i - a_j)^2.
\end{equation}
证明:$12S \ge n(n-1)M$.
\end{homework}

\newpage
\begin{homework}{}{}
设整数 \( n \geq 2 \),\( \alpha_i \in \left(0, \frac{\pi}{2}\right) \)(\( i = 1, 2, \dots, n \)),证明:
\begin{equation}
\prod_{i=1}^{n} \cos\alpha_i \cdot \sum_{i=1}^{n} \tan\alpha_i \leq \frac{(n - 1)^{\frac{n - 1}{2}}}{n^{\frac{n - 2}{2}}}.
\end{equation}
\end{homework}



% \section{简单备选题}

% % 预 4
% \begin{example}{}{}
% 已知正实数 $x_1, x_2, \dots, x_n$,其中任意两个数 $x_i, x_j\ (i \ne j)$,都有 $x_i x_j \ge 1$。证明:
% \begin{equation}
%     \frac{1}{1+x_1} + \frac{1}{1+x_2} + \dots + \frac{1}{1+x_n} \ge \frac{n}{1+\sqrt[n]{x_1 x_2 \dots x_n}}
% \end{equation}
% \end{example}





% % 练 1
% \begin{problem}{}{}
% 设 $A, B, C$ 为 $\triangle ABC$ 的三个角,证明:
% \begin{equation}
%     \cos A + \cos B + \cos C > 1
% \end{equation}
% \end{problem}

% % 例 1
% \begin{example}{}{}
% 设 $n$ 为正整数,$x_1, x_2, \dots, x_n$ 为非负实数,且 $x_1 + x_2 + \dots + x_n = \pi$。记 $M$ 为 $\sin^2 x_1 + \sin^2 x_2 + \dots + \sin^2 x_n$ 的最大值,证明:当 $n > 3$ 时,$M$ 为定值。
% \end{example}
% \newpage

% % 习 4
% \begin{problem}{}{}
% \begin{enumerate}
%     \item 设 $a, b, c, d \ge 0$,且 $a+b+c+d=4$。证明:
%     \begin{equation}
%         bcd+cda+dab+abc-abcd \le \frac{1}{2} \cdot (ab+ac+ad+bc+bd+cd)
%     \end{equation}
%     \item 试求实数 $k$ 的取值范围,使不等式
%     \begin{equation}
%         0 \le xyz+yzw+zwx+wxy - kxyzw < \frac{16-k}{256}
%     \end{equation}
%     对一切满足 $x+y+z+w=1$ 的所有非负实数 $x, y, z, w$ 恒成立。
% \end{enumerate}
% \end{problem}



% % 例 6
% \begin{problem}{}{}
% \begin{enumerate}
%     \item 设 $a, b, c, d \ge 0$,且 $a+b+c+d=1$。证明:
%     \begin{equation}
%         bcd+cda+dab+abc \le \frac{1}{27} + \frac{176}{27}abcd
%     \end{equation}
%     \item 试求实数 $k$ 的取值范围,使不等式
%     \begin{equation}
%         0 < xyz+yzw+zwx+wxy - kxyzw \le \frac{4-k}{27}(xy+xz+xw+yz+yw+zw)
%     \end{equation}
%     对一切满足 $x+y+z+w=4$ 的所有非负实数 $x, y, z, w$ 恒成立。
% \end{enumerate}
% \end{problem}





% % 练 2
% \begin{problem}{}{}
% 已知实数 $a, b, c, d$ 满足 $a \ge c,\ b \ge d > 0$,试求
% \begin{equation}
%     S = \frac{a}{a+b} + \frac{b}{b+c} + \frac{c}{c+d} + \frac{d}{d+a}
% \end{equation}
% 的取值范围。
% \end{problem}


% % 练 3
% \begin{problem}{}{}
% \begin{enumerate}
%     \item 求实数 $k$ 的最大值,使得对任意实数 $x \ge 1, y \ge 1$,恒有
%     \begin{equation}
%         \frac{x^2}{1+x} + \frac{y^2}{1+y} + (x-1)(y-1) \ge kxy
%     \end{equation}
%     \item 已知正实数 $a, b, c$ 满足 $abc=1$,证明:
%     \begin{equation}
%         \frac{1}{a} + \frac{1}{b} + \frac{1}{c} + \frac{3}{a+b+c} \ge 4
%     \end{equation}
% \end{enumerate}
% \end{problem}


% % 练 4
% \begin{problem}{}{}
% 设正实数 $a, b, c$ 满足 $\min\{ab, bc, ca\} \ge 1$,证明:
% \begin{equation}
%     \sqrt[3]{(a^2+1)(b^2+1)(c^2+1)} \le \left(\frac{a+b+c}{3}\right)^2 + 1
% \end{equation}
% \end{problem}


% % 习 1
% \begin{problem}{}{}
% 试求最大的实数 $k$,满足:对任意正实数 $a, b, c\ (a^2 > bc)$,都有
% \begin{equation}
%     (a^2-bc)^2 > k \cdot (b^2-ca)(c^2-ab)
% \end{equation}
% \end{problem}

% % 习 3
% \begin{problem}{}{}
% 设正实数 $a, b, c, d$ 满足 $abcd=1$,证明:
% \begin{equation}
%     \frac{1}{a} + \frac{1}{b} + \frac{1}{c} + \frac{1}{d} + \frac{9}{a+b+c+d} \ge \frac{25}{4}
% \end{equation}
% \end{problem}
% \newpage

% === 备用题 ===

% % 备 1
% \begin{problem}{}{}
% 求所有的正实数 $t$,使得对任意正整数 $n \ge 2$ 和满足 $\sum_{i=1}^n a_i = n$ 的正实数 $a_1, a_2, \dots, a_n$,总有
% \begin{equation}
%     \sum_{i=1}^n \frac{1}{a_i} - t \cdot \prod_{i=1}^n \frac{1}{a_i} \le n-t
% \end{equation}
% \end{problem}
% \newpage

% % === 大显身手 (更多练习) ===

% % === 习题 ===

% % 习 2
% \begin{problem}{}{}
% 给定正整数 $n \ge 2$,非负实数 $x_1, x_2, \dots, x_n$ 满足 $x_1+x_2+\dots+x_n=1$。求 $\sum_{i=1}^n (x_i^4 - x_i^5)$ 的最大值。
% \end{problem}
% \newpage

% % 例 4
% \begin{example}{}{}
% 设 $a_1, a_2, \dots, a_n$ 是给定的不全为零的实数,$r_1, r_2, \dots, r_n$ 为实数,如果不等式
% \begin{equation}
%     r_1(x_1-a_1) + r_2(x_2-a_2) + \dots + r_n(x_n-a_n) \le \sqrt{x_1^2+x_2^2+\dots+x_n^2} - \sqrt{a_1^2+a_2^2+\dots+a_n^2}
% \end{equation}
% 对任何实数 $x_1, x_2, \dots, x_n$ 成立,求 $r_1, r_2, \dots, r_n$ 的值。
% \end{example}
% \newpage