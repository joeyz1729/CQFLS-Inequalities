\chapter{局部不等式}
% === 例题 ===
\section{例题}
% 1
\begin{example}{}{}
设整数 $n\ge3$,实数 $x_{1},x_{2},\dots,x_{n}\in[-1,1]$,且满足 $\sum_{i=1}^{n}x_{i}^{3}=0$。
求证:
\begin{equation}
    \sum_{i=1}^{n}x_{i}\le\frac{n}{3}.
\end{equation}
\end{example}
\newpage



% 2
\begin{example}{}{}
求所有的整数 $n\ge2$,使得存在实数 $x_{1},x_{2},\dots,x_{n}$,满足
\begin{equation}
    \sum_{k=1}^{n}x_{k}=0,\quad \sum_{k=1}^{n}x_{k}^{2}=1,\quad \sum_{k=1}^{n}x_{k}^{3}=2 \max_{1\le i\le n}x_{i}-\frac{2}{\sqrt{n}}.
\end{equation}
\end{example}
\newpage



% 3
\begin{example}{}{}
给定整数 $n\ge3$,设不全为 0 的实数 $x_{1},x_{2},\dots,x_{n}$ 满足 $\sum_{k=1}^{n}x_{k}=0$。记
\begin{equation}
    A=\Bigl(\sum_{k=1}^{n}x_{k}^{3}\Bigr)^{2},\quad B=\Bigl(\sum_{k=1}^{n}x_{k}^{2}\Bigr)^{3}.
\end{equation}
求 $\frac{A}{B}$ 的最大值。
\end{example}
\newpage

% 4
\begin{example}{}{}
设非负实数 $a_{1},a_{2},\dots,a_{100}$ 满足 $a_{i}+a_{i+1}+a_{i+2}\le1\ (1\le i\le100)$,其中 $a_{101}=a_{1},\ a_{102}=a_{2}$。求 $\sum_{i=1}^{100}a_{i}a_{i+2}$ 的最大值。
\end{example}
\newpage

% 5
\begin{example}{}{}
设整数 $n\ge2$,$z_1, z_2, \dots, z_n$ 是复数,求证:
\begin{equation}
    \Bigl(\sum_{1\le i<j\le n}|z_{i}-z_{j}|\Bigr)^{2}\ge(n-1)\sum_{1\le i<j\le n}|z_{i}-z_{j}|^{2}.
\end{equation}
\end{example}
\newpage






% bj
\begin{example}{}{}
设整数 $n\ge2$,正实数 $a_{1},a_{2},\dots,a_{n}$ 满足 $\sum_{i=1}^{n}a_{i}=1$。求证:
\begin{equation}
    \sum_{i=1}^{n}\frac{a_{i}a_{i+1}}{1-(a_{i}-a_{i+1})^{2}}\le\frac{1}{2},
\end{equation}
其中 $a_{n+1}=a_{1}$。
\end{example}
\newpage







% bj
\begin{example}{}{}
设整数 $n\ge3$,$a_{1},a_{2},\dots,a_{n}$ 是正实数,求证:
\begin{equation}
    \sum_{i=1}^{n}\frac{1}{a_{i}(a_{i}^{2}+a_{i-1}a_{i+1})}\le\sum_{i=1}^{n}\frac{1}{a_{i}a_{i+1}(a_{i}+a_{i+1})},
\end{equation}
其中 $a_{0}=a_{n},\ a_{n+1}=a_{1}$。
\end{example}
\newpage


% bj
\begin{example}{2012年APMO P5}{}
设整数 $n\ge2$,正实数 $a_{1},a_{2},\dots,a_{n}$ 满足 $\sum_{i=1}^{n}a_{i}^{2}=n$。求证:
\begin{equation}
    \sum_{1\le i<j\le n}\frac{1}{n-a_{i}a_{j}}\le\frac{n}{2}.
\end{equation}
\end{example}
\newpage


% % === 第 12 题 ===
% \begin{example}{}{p12}
% 给定整数 \( n \geq 3 \),且 \( a_i \geq 1(i=1,2, \cdots, n) \),证明:
% \begin{equation}
%     \sum_{i=1}^n a_i \cdot \sum_{i=1}^n \frac{1}{a_i} \leq n^2+\sum_{1 \leq i<j \leq n}\left|a_i-a_j\right|
% \end{equation}
% \end{example}
% \newpage


% bj
\begin{example}{}{}
设正实数 $x_{1},x_{2},\dots,x_{n}$ 满足 $x_1+x_2+\dots+x_n=n$。求证:
\begin{equation}
    \sum_{i=1}^{n}\frac{i}{1+x_{i}+\dots+x_{i}^{i-1}}\le\sum_{i=1}^{n}\frac{i+1}{1+x_{i}+\dots+x_{i}^{i}}.
\end{equation}
\end{example}
\newpage






% === 练习题 ===
% === 练习题 ===
% === 练习题 ===
\section{练习题}
% bj
\begin{homework}{}{}
设整数 $n\ge2$,实数 $x_{1},x_{2},\dots,x_{n}$ 满足 $\sum_{i=1}^{n}x_{i}=0,\ \sum_{i=1}^{n}x_{i}^{2}=1$。求证:存在 $a\in\{x_{1},x_{2},\dots,x_{n}\}$,使得对任意 $1\le j\le n$,都有
\begin{equation}
    \sum_{i=1}^{n}x_{i}^{3}\ge a+2x_{j}+nax_{j}^{2}.
\end{equation}
\end{homework}



% 第 14 题
\newpage
\begin{homework}{}{}
求最大的实数 $\lambda$,使得对任意正整数 $n$ 和任意实数 $a_{1},a_{2},\dots,a_{n}$,只要 $\sum_{i=1}^n a_i=0$,就有
\begin{equation}
    \sum_{i=1}^{n}a_{i}^{4}\ge\frac{\lambda}{n^{3}}\Bigl(\sum_{i=1}^{n}a_{i}^2\Bigr)^{2}.
\end{equation}
\end{homework}


\newpage
% bj
\begin{homework}{}{}
设整数 $n\ge3$,$x_{1},x_{2},\dots,x_{n}$ 是不小于 1 的实数,求证:
\begin{equation}
    \sum_{i=1}^{n}\frac{\sqrt{x_{i}x_{i+1}-1}}{x_{i+1}+x_{i+2}}\le\frac{1}{4}\sum_{i=1}^{n}x_{i},
\end{equation}
其中下标按模 $n$ 理解。
\end{homework}


% 第 11 题
\newpage
\begin{homework}{}{}
设实数 $a_{1}\ge a_{2}\ge\dots\ge a_{2n+1}$,求证:
\begin{equation}
    \Bigl(\sum_{i=1}^{2n+1}a_{i}\Bigr)^{2}\ge 4n\sum_{i=1}^{n+1}a_{i}a_{n+i}.
\end{equation}
\end{homework}
% \newpage

% 第 12 题
\newpage
\begin{homework}{}{}
设 $x_{1},x_{2},\dots,x_{n}$ 是正实数,求证:
\begin{equation}
    \frac{1}{1+x_{1}}+\frac{2}{1+x_{1}+x_{2}}+\dots+\frac{n}{1+x_{1}+\dots+x_{n}}\le\frac{n}{2}\sqrt{\frac{1}{x_{1}}+\frac{1}{x_{2}}+\dots+\frac{1}{x_{n}}}.
\end{equation}
\end{homework}
% \newpage



% example 60
\newpage
\begin{homework}{}{}
设整数 $n \ge 2$,且非负实数 $a_1, a_2, \dots, a_n$ 满足 $a_1 + a_2 + \dots + a_n = 3$,证明:
\begin{equation}
    \sum_{i=1}^n a_1a_2\dots a_{i-1}a_i \le 4
\end{equation}
\end{homework}
% \newpage

% example 59
\newpage
\begin{homework}{}{}
已知非负实数 $x_1, x_2, \dots, x_n$ 满足 $x_1 + x_2 + \dots + x_n = n$,证明:
\begin{equation}
    \frac{x_1}{1+x_1^2} + \frac{x_2}{1+x_2^2} + \dots + \frac{x_n}{1+x_n^2} \le \frac{1}{1+x_1} + \frac{1}{1+x_2} + \dots + \frac{1}{1+x_n}
\end{equation}
\end{homework}
% \newpage


% example 63
\newpage
\begin{homework}{}{}
已知正实数 $x_1, x_2, \dots, x_n$ 满足 $\sum_{i=1}^n \frac{1}{1+x_i} = 1$,证明:
\begin{equation}
    \frac{n-1}{n-1+x_1^2} + \frac{n-1}{n-1+x_2^2} + \dots + \frac{n-1}{n-1+x_n^2} \ge 1
\end{equation}
\end{homework}
% \newpage


% example 62
\newpage
\begin{homework}{2004 CTST}{}
已知非负实数 $x_1, x_2, \dots, x_n$ 满足 $\sum_{i=1}^n \frac{1}{1+x_i} = 1$,证明:
\begin{equation}
    \frac{x_1}{n-1+x_1^2} + \frac{x_2}{n-1+x_2^2} + \dots + \frac{x_n}{n-1+x_n^2} \le 1
\end{equation}
\end{homework}
% \newpage


% example 64
\newpage
\begin{homework}{}{}
设 $a_i \in [0, 1], i=1, 2, \dots, n$,记 $S = a_1^3 + a_2^3 + \dots + a_n^3$,证明:
\begin{equation}
    \frac{a_1}{2n+1+S-a_1^3} + \frac{a_2}{2n+1+S-a_2^3} + \dots + \frac{a_n}{2n+1+S-a_n^3} \le \frac{1}{3}
\end{equation}
\end{homework}




\newpage
% 第 2 题
\begin{homework}{}{}
设整数 $n\geq 3$,非负实数 $x_1,x_2,\dots,x_n$ 的和为 1,证明:
\begin{equation}
    \Biggl[\sum_{i=1}^n\frac{x_i}{1+(n-2)x_i}\Biggr]^2 \leq\sum_{i=1}^n\frac{x_i^2}{[1+(n-2)x_i]^2} +\frac{n}{4(n-1)}
\end{equation}
\end{homework}



% example 66
\newpage
\begin{homework}{}{}
设整数 $n \ge 2$,且正实数 $a_1, a_2, \dots, a_n$ 满足 $\sum_{i=1}^n a_i = n$,证明:
\begin{equation}
    \sum_{i=1}^n \sqrt[n+1]{a_i} \ge \frac{2n}{n+1} \cdot \left( \sum_{1 \le i < j \le n} a_i a_j - \frac{n^2 - 2n - 1}{2} \right)
\end{equation}
\end{homework}
% \newpage


% example 57
\newpage
\begin{homework}{2007 白俄罗斯}{}
已知 $x_1, x_2, \dots, x_{n+1}$ 均为正数,证明:
\begin{equation}
    \frac{1}{x_1} + \frac{x_1}{x_2} + \frac{x_1x_2}{x_3} + \dots + \frac{x_1x_2\dots x_n}{x_{n+1}} \ge 4(1 - x_1x_2\dots x_{n+1})
\end{equation}
\end{homework}



% example 71
\newpage
\begin{homework}{1993 圣彼得堡}{}
设 $a_i \in [-1, 1],\ a_i a_{i+1} \ne -1,\ i=1, 2, \dots, n$,且 $a_{n+1} = a_1$,证明:
\begin{equation}
    \frac{1}{1+a_1a_2} + \frac{1}{1+a_2a_3} + \dots + \frac{1}{1+a_na_{n+1}} \ge \frac{1}{1+a_1^2} + \frac{1}{1+a_2^2} + \dots + \frac{1}{1+a_n^2}
\end{equation}
\end{homework}



% example 65
\newpage
\begin{homework}{}{}
设整数 $n \ge 2$,且 $x_1, x_2, \dots, x_n$ 均为正实数,证明:
\begin{equation}
    \sum_{i=1}^n \left(\frac{x_i}{\sum_{j=1}^n x_j - x_i}\right)^{\frac{n-1}{n}} \ge \frac{n \cdot \sqrt[n]{n-1}}{n-1}
\end{equation}
\end{homework}
% \newpage