% chapters/13_reverse_ineq.tex

\chapter{反向不等式}
\section{基础知识}
\begin{thm}{Kantorović 不等式}{}
设整数 $n\ge2$,正实数 $\lambda_{1},\lambda_{2},\dots,\lambda_{n}$ 满足 $\sum_{i=1}^{n}\lambda_{i}=1$。设正实数 $m<M$,实数 $a_{1},a_{2},\dots,a_{n}\in[m,M]$,则
\begin{equation}
    \left(\sum_{i=1}^{n}\lambda_{i}a_{i}\right)\left(\sum_{i=1}^{n}\frac{\lambda_{i}}{a_{i}}\right)\le\frac{(M+m)^{2}}{4Mm}.
\end{equation}
\end{thm}
\newpage


\begin{thm}{Polyá-Szegö 不等式}{}
设整数 $n\ge2$,正实数 $a<A,\ b<B$,实数 $a_{1},a_{2},\dots,a_{n}\in[a,A],\ b_{1},b_{2},\dots,b_{n}\in[b,B]$。则
\begin{equation}
    \left(\sum_{i=1}^{n}a_{i}^{2}\right)\left(\sum_{i=1}^{n}b_{i}^{2}\right)\le\frac{1}{4}\left(\sqrt{\frac{AB}{ab}}+\sqrt{\frac{ab}{AB}}\right)^{2}\left(\sum_{i=1}^{n}a_{i}b_{i}\right)^{2}.
\end{equation}
\end{thm}
\newpage


% \section{预习题}


\section{例题}
% example 69
\begin{example}{1978 苏联}{}
设 $x_1, x_2, \dots, x_n \in [a, b]$,其中 $0 < a < b$,证明:
\begin{equation}
    (x_1 + x_2 + \dots + x_n) \cdot \left( \frac{1}{x_1} + \frac{1}{x_2} + \dots + \frac{1}{x_n} \right) \le \frac{(a+b)^2}{4ab} \cdot n^2
\end{equation}
\end{example}
\newpage


% example 67
\begin{example}{2016 克罗地亚}{}
已知 $x_1, x_2, \dots, x_n$ 均为非负实数,证明:
\begin{equation}
    \left( x_1 + \frac{x_2}{2} + \dots + \frac{x_n}{n} \right) \cdot \left( x_1 + 2x_2 + \dots + nx_n \right) \le \frac{(n+1)^2}{4n} \cdot (x_1 + x_2 + \dots + x_n)^2
\end{equation}
\end{example}
\newpage

% 作业题 3
\begin{example}{}{}
设整数 $n\ge2$,正实数 $a_{1},a_{2},\dots,a_{n}$ 满足
\begin{equation}
    (a_{1}+a_{2}+\dots+a_{n})\left(\frac{1}{a_{1}}+\frac{1}{a_{2}}+\dots+\frac{1}{a_{n}}\right)\le\left(n+\frac{1}{2}\right)^{2}.
\end{equation}
求证:$\max\{a_{1},\dots,a_{n}\}\le4 \min\{a_{1},\dots,a_{n}\}$。
\end{example}
\newpage

 
% 例 3
\begin{example}{}{}
给定整数 $n\ge2$。设实数 $a_{1},a_{2},\dots,a_{n}\in[1,2]$,求
\begin{equation}
    \left(\sum_{i=1}^{n}a_{i}\right)\left(\sum_{i=1}^{n}\frac{1}{a_{i}}\right)^{2}
\end{equation}
的最大值。
\end{example}
\newpage

% 例 4
\begin{example}{1998年高中联赛}{}
设整数 $n\ge2$,实数 $a_{1},a_{2},\dots,a_{n},b_{1},b_{2},\dots,b_{n}\in[1,2]$ 且满足 $\sum_{i=1}^{n}a_{i}^{2}=\sum_{i=1}^{n}b_{i}^{2}$。
求证:
\begin{equation}
    \sum_{i=1}^{n}\frac{a_{i}^{3}}{b_{i}}\le\frac{17}{10}\sum_{i=1}^{n}a_{i}^{2}.
\end{equation}
\end{example}
\newpage

% 例 5
\begin{example}{}{}
设整数 $n\ge3$,正实数 $a_{1},a_{2},\dots,a_{n}$ 满足
\begin{equation}
    (a_{1}+a_{2}+\dots+a_{n})\left(\frac{1}{a_{1}}+\frac{1}{a_{2}}+\dots+\frac{1}{a_{n}}\right)<n^{2}+1.
\end{equation}
求证:$a_{1},a_{2},\dots,a_{n}$ 中任意三个数均能构成三角形的三边长。
\end{example}
\newpage






% 例 8
\begin{example}{}{}
设整数 $n\ge2$,正实数 $a<b$,实数 $a_{1},a_{2},\dots,a_{n}\in[a,b]$。求证:
\begin{equation}
    \frac{a_{1}+a_{2}+\dots+a_{n}}{n\sqrt[n]{a_{1}a_{2}\dots a_{n}}}\le\left(\frac{M}{2}\right)^{2-\frac{2}{n}},
\end{equation}
其中 $M=\sqrt{\frac{a}{b}}+\sqrt{\frac{b}{a}}$。
\end{example}
\newpage

% 例 9
\begin{example}{}{}
给定整数 $n\ge2$。求最小的实数 $\lambda$,使得对任意正实数 $a_{1},a_{2},\dots,a_{n}$ 都有
\begin{equation}
    \sqrt[n]{\prod_{i=1}^{n}a_{i}}+\lambda\sum_{1\le i<j\le n}|a_{i}-a_{j}|\ge\frac{1}{n}\sum_{i=1}^{n}a_{i}.
\end{equation}
\end{example}
\newpage



% 例 10
\begin{example}{}{}
给定整数 $n\ge2$。求最小的实数 $\lambda$,使得对任意实数 $a_{1},a_{2},\dots,a_{n}$,都有
\begin{equation}
    \left|\sum_{i=1}^{n}a_{i}\right|+\lambda\sum_{1\le i<j\le n}|a_{i}-a_{j}|\ge\sum_{i=1}^{n}|a_{i}|.
\end{equation}
\end{example}



\newpage 
\section{练习题}



% example 68
\begin{problem}{}{}
已知非负实数 $a_1, a_2, \dots, a_n$ 满足 $a_1 + a_2 + \dots + a_n = 1$,其中 $n \ge 3$,求 
\begin{equation}
    \sum_{i=1}^n i a_i \cdot (\sum_{i=1}^n \frac{a_i}{i})^2
\end{equation}
的最大值。
\end{problem}



% example 70
\begin{problem}{}{}
已知非负实数 $x_1, x_2, \dots, x_n$ 满足 $\sum_{i=1}^n x_i = 1$,证明:
\begin{equation}
    1 \le \sum_{i=1}^n (2i-1)x_i \cdot \sum_{i=1}^n \frac{x_i}{2i-1} \le \frac{n^2}{2n-1}
\end{equation}
\end{problem}

\begin{problem}{}{}
设整数 $n\ge2$,正实数 $a_{1},a_{2},\dots,a_{n}$ 满足 $a_{1}\le a_{2}\le\dots\le a_{n}$,且 $a_{1}\ge\frac{a_{2}}{2}\ge\dots\ge\frac{a_{n}}{n}$。
求证:
\begin{equation}
    \frac{a_{1}+a_{2}+\dots+a_{n}}{n\sqrt[n]{a_{1}a_{2}\dots a_{n}}}\le\frac{n+1}{2\sqrt[n]{n!}}.
\end{equation}
\end{problem}
% 作业题 1
\begin{problem}{}{}
设整数 $n\ge2$,正实数 $\lambda_{1},\lambda_{2},\dots,\lambda_{n}$ 满足 $\sum_{i=1}^{n}\lambda_{i}=1$。设正实数 $m<M$,实数 $a_{1},a_{2},\dots,a_{n}\in[m,M]$。求证:
\begin{equation}
    \sum_{i=1}^{n}\lambda_{i}a_{i}-\frac{1}{\sum_{i=1}^{n}\frac{\lambda_{i}}{a_{i}}}\le(\sqrt{M}-\sqrt{m})^{2}.
\end{equation}
\end{problem}
% 例 7


% 作业题 2
\begin{problem}{}{}
设整数 $n\ge2$,正实数 $m<M$,实数 $a_{1},a_{2},\dots,a_{n}\in[m,M]$。求证:
\begin{equation}
    \sum_{i=1}^{n}\frac{a_{i}^{2}}{a_{i+1}}\le\frac{M^{2}-Mm+m^{2}}{Mm}\sum_{i=1}^{n}a_{i},
\end{equation}
其中 $a_{n+1}=a_{1}$。
\end{problem}

% 例 6
\begin{problem}{}{}
设整数 $n\ge3$,正实数 $a_{1},a_{2},\dots,a_{n}$ 满足
\begin{equation}
    (n-1)(a_{1}^{4}+a_{2}^{4}+\dots+a_{n}^{4})<(a_{1}^{2}+a_{2}^{2}+\dots+a_{n}^{2})^{2}.
\end{equation}
求证:$a_{1},a_{2},\dots,a_{n}$ 中任意三个数均能构成三角形的三边长。
\end{problem}

% 作业题 4
\begin{problem}{}{}
给定整数 $n\ge3$。求最大的实数 $\lambda$,使得只要正实数 $a_{1},a_{2},\dots,a_{n}$ 满足
\begin{equation}
    a_{1}^{2}+a_{2}^{2}+\dots+a_{n}^{2}<\lambda(a_{1}+a_{2}+\dots+a_{n})^{2},
\end{equation}
那么 $a_{1},a_{2},\dots,a_{n}$ 中任意三个数便能构成三角形的三边长。
\end{problem}

